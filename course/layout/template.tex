\usepackage[T1]{fontenc}
\usepackage{import}
\usepackage{media9} 
\usepackage{amssymb} % for changing item bullets
\usepackage{bbding} % dingbats (for bullets)
\usepackage{textcomp}
\usepackage{color,calc,graphicx,soul}
\usepackage{geometry}
\usepackage{layout}
\RequirePackage[calcwidth]{titlesec}
\usepackage{bookman}
\usepackage{fix-cm}
\let\footruleskip\undefined %will be redefined in fancyhdr
\usepackage{fancyhdr}
\usepackage[bookmarks, colorlinks=true]{hyperref}
\usepackage{makeidx}
\usepackage{hyphenat}
\usepackage{parskip}
\usepackage[most]{tcolorbox}
\usetikzlibrary{shadows}
\usepackage{environ}
\usepackage{polyglossia}
\usepackage{listings,fontspec}


\geometry{
	includeheadfoot,
	margin=2.54cm
}

%% colors 
\definecolor{primaryBack}{HTML}{4782D3}
\definecolor{primaryDark}{HTML}{14478D}
\definecolor{primaryLight}{HTML}{98BEF4}
\definecolor{primaryIntense}{HTML}{2965B9}

\definecolor{secondaryBack}{HTML}{FF8145}
\definecolor{secondaryDark}{HTML}{D64E0D}
\definecolor{secondaryLight}{HTML}{FFB795}
\definecolor{secondaryIntense}{HTML}{FF6C25}

\definecolor{thirdBack}{HTML}{FFDC45}
\definecolor{thirdDark}{HTML}{D6B00D}
\definecolor{thirdLight}{HTML}{FFEB95}
\definecolor{thirdIntense}{HTML}{FFD625}

\definecolor{complementBack}{HTML}{FFB945}
\definecolor{complementDark}{HTML}{D68A0D}
\definecolor{complementLight}{HTML}{FFD795}
\definecolor{complementIntense}{HTML}{FFAD25}

\definecolor{hidden}{HTML}{FFD7B5}

\definecolor{red}{HTML}{FF0000}
\definecolor{blue}{HTML}{0000FF}
\definecolor{green}{HTML}{009900}
\definecolor{orange}{HTML}{FFA500}
\definecolor{black}{HTML}{000000}

% section header
\titleformat{\section}[hang]{\bfseries}{
\huge\color{primaryIntense}\thesection}{10pt}{\huge\raggedleft\color{primaryIntense}}[{\titlerule[0.5pt]}]

% subsection header
\titleformat{\subsection}[hang]{\bfseries}{
\huge\color{primaryIntense}\thesubsection}{10pt}{\LARGE\raggedleft\color{primaryIntense}}
\setcounter{secnumdepth}{3}

% subsection header
\titleformat{\subsubsection}[hang]{\bfseries}{
\large\color{primaryIntense}\thesubsubsection}{10pt}{\large\raggedleft\color{primaryIntense}}


% table of contents
\pagestyle{fancy}
\hypersetup{linkcolor=primaryDark}
\maxtocdepth{subsubsection}

% colored description
%\setdescription{leftmargin=1cm,labelindent=0.4cm}
%\renewcommand{\descriptionlabel}[1]
%{\hspace{\labelsep}{\color{red}{\bfseries #1}}}
%
% color of texttt
\let\Oldtexttt\texttt
\renewcommand\texttt[1]{{\ttfamily\color{primaryDark} {\bfseries #1}}}

% chapter header
\makeatletter
\newlength\dlf@normtxtw
\setlength\dlf@normtxtw{\textwidth}
\def\myhelvetfont{\def\sfdefault{mdput}}
\newsavebox{\feline@chapter}
\newcommand\feline@chapter@marker[1][4cm]{%
\sbox\feline@chapter{%
\resizebox{!}{#1}{\fboxsep=1pt%
\colorbox{primaryBack}{\color{complementIntense}\bfseries\sffamily\thechapter}%
}}%
\rotatebox{90}{%
\resizebox{%
\heightof{\usebox{\feline@chapter}}+\depthof{\usebox{\feline@chapter}}}%
{!}{\scshape\so\@chapapp}}\quad%
\raisebox{\depthof{\usebox{\feline@chapter}}}{\usebox{\feline@chapter}}%
}
\newcommand\feline@chm[1][4cm]{%
\sbox\feline@chapter{\feline@chapter@marker[#1]}%
\makebox[0pt][l]{% aka \rlap
\makebox[1cm][r]{\usebox\feline@chapter}%
}}
\makechapterstyle{daleif1}{
\renewcommand\chapnamefont{\normalfont\Large\scshape\raggedleft\so}
\renewcommand\chaptitlefont{\normalfont\huge\bfseries\scshape\color{primaryIntense}}
\renewcommand\chapternamenum{}
\renewcommand\printchaptername{}
\renewcommand\printchapternum{\null\hfill\feline@chm[2.5cm]\par}
\renewcommand\afterchapternum{\par\vskip\midchapskip}
\renewcommand\printchaptertitle[1]{\chaptitlefont\raggedleft ##1\par}
}
\makeatother
\chapterstyle{daleif1}

%title page
\makeatletter
\newlength\drop
\newcommand*{\titleGM}{%
	\thispagestyle{empty}
	\begingroup% Gentle Madness
	\drop = 0.1\textheight
	\vspace*{\baselineskip}
	\hbox{%
		\hspace*{0.2\textwidth}%
		\hspace*{0.05\textwidth}% 
		\parbox[b]{0.75\textwidth}{%
		   \vbox{%
				 \vspace{\drop}
				 \rule{0.75\textwidth}{1pt}
				 {\Huge\bfseries\sffamily\raggedleft\color{primaryIntense}\@title\par}
				 \vskip2.37\baselineskip
				 {\Large\bfseries\sffamily\raggedleft\color{primaryDark}\@author\par}
				 \rule{0.75\textwidth}{1pt}
				 \vskip4.37\baselineskip
				 {\Large\bfseries\sffamily\raggedleft\color{primaryDark}Revision: \@date\par}
				 \vskip4.37\baselineskip
				 \includegraphics[scale=0.2]{mute-logo}
			 }
		}
	}
	\pagebreak
	\vfill
	\null
	\endgroup
}
\makeatother

%part page
\makeatletter

\renewcommand\printpartname{}
\renewcommand\printpartnum{}
\renewcommand{\parttitlefont}{\normalfont\raggedleft\Huge\scshape\color{primaryIntense}}
\newcommand{\partAbstract}[1]{\gdef\@partAbstract{\leftskip5em\normalsize\color{black}\textit{#1}}}
\renewcommand{\printparttitle}[1]{%
\raggedleft
\rule{0.75\textwidth}{1pt}
\vskip1.37\baselineskip
\parttitlefont Part \thepart: #1
\color{black}
\rule{0.75\textwidth}{1pt}
\vskip1.37\baselineskip
\@partAbstract\vfil
}
\makeatother

%paragraph
\setlength{\parindent}{0pt}
\setlength{\parskip}{0.5cm plus3mm minus2mm}

%table

\newcommand{\tempcaption}{}
\newenvironment{myTable}[2]{%
  \gdef\tempcaption{#1}%
  \begin{center}%

    \begin{tcolorbox}[
      colback=primaryLight,
      colframe=primaryIntense,
      left=1mm,right=1mm,top=1mm,bottom=1mm,boxsep=0mm,
      toptitle=0.5mm,
      bottomtitle=0.5mm,
      center title,
      title=\tempcaption,
    ]%
    \centering
    \begin{tabular}{#2}%
}{%
    \end{tabular}%
    \end{tcolorbox}%
  \end{center}%
}

%listing
\lstset{
  frame=tb,
  language=c++,
  aboveskip=3mm,
  belowskip=3mm,
  showstringspaces=false,
  columns=flexible,
  basicstyle={\small\ttfamily},
  %backgroundcolor=\color{primaryLight}, % Set the background color for the snippet - useful for highlighting
  firstnumber=1, % Line numbers begin at line 1
  stepnumber=5,
  frame=lines, % Frame around the code box, value can be: none, leftline, topline, bottomline, lines, single, shadowbox
  %frameround=tttt, % Rounds the corners of the frame for the top left, top right, bottom left and bottom right
  numbers=left,
  numbersep=10pt, % Distance of line numbers from the code box
  numberstyle=\tiny\color{primaryDark}, % Style used for line numbers
  rulecolor=\color{primaryDark}, % Frame border color
  keywordstyle=\color{blue},
  commentstyle=\color{green},
  stringstyle=\color{red},
  breaklines=true,
  breakatwhitespace=true,
  tabsize=2
}

\lstset{
  emph={Vec, Vec2, Circle, Memc, Str, FREPA, SQLColumn, SQLValues, TextData, File, 
	      Memx, SQL, GuiObjs, Window, Button, TextLine, TextNode},
  emphstyle={\color{blue}\bfseries}
}

\renewcommand{\labelitemi}{\SquareShadowBottomRight}
\renewcommand{\labelitemii}{\tiny\Square}

% special blocks
\newtheorem{example}{Example}
\newtheorem{remember}{Remember This!}

\tcbset{
myNote/.style={
  enhanced,
  colback=complementLight,
  colframe=complementDark,
  fonttitle=\scshape,
  title={Note},
  title style={fill=complementDark},
  }
}

\newtcolorbox{note}{myNote}

\tcbset{
myInfo/.style={
  enhanced,
  colback=complementLight,
  colframe=complementDark,
  fonttitle=\scshape,
  title={Info},
  title style={fill=complementDark},
  }
}

\newtcolorbox{info}{myInfo}

%\tcbset{
%myWorld/.style={
  %enhanced,
  %colback=thirdLight,
  %colframe=thirdIntense,
  %fonttitle=\scshape,
  %title={De echte wereld},
  %title style={fill=thirdDark},
  %coltitle=thirdLight,
  %}
%}
%
%\newtcolorbox{praktijk}{myWorld}

\tcbset{
myWarning/.style={
  enhanced,
  colback=thirdLight,
  colframe=thirdIntense,
  fonttitle=\scshape,
  title={Warning!},
  title style={fill=thirdDark},
  coltitle=thirdLight,
  }
}

\newtcolorbox{warning}{myWarning}

\tcbset{
myExercise/.style={
  enhanced,
  colback=primaryLight,
  colframe=primaryDark,
  fonttitle=\scshape,
  title={Time for Action},
  title style={fill=primaryDark},
  }
}

\newtcolorbox{exercise}{myExercise}

\newcommand*\tick{\item[\Checkmark]}
\newcommand*\fail{\item[\XSolidBrush]}
\newcommand*\pencil{\item[\PencilRightDown]}

\tcbset{
myTest/.style={
  enhanced,
  colback=secondaryLight,
  colframe=secondaryDark,
  fonttitle=\scshape,
  fontlower=\itshape,
  title={Test Jezelf},
  title style={fill=secondaryDark},
  collower=hidden,
  }
}

\newtcolorbox{test}{myTest}

\lstnewenvironment{code}
	{
		\lstset{language={[Visual]C++}, columns=fixed}
	}
	{
	}
%
\newcommand{\eeClass}[1]{\texttt{#1}}
\newcommand{\eeFunc}[1]{\texttt{#1}}
\newcommand{\eeOpp}[1]{\texttt{#1}}

\newcommand{\video}[1]{}

\author{Mute (http://mutecode.com)}