%!TEX program = xelatex

\documentclass[11pt, oldfontcommands, oneside, a4paper]{memoir}
\usepackage{etoolbox}

\usepackage[T1]{fontenc}
\usepackage{import}
\usepackage{media9} 
\usepackage{amssymb} % for changing item bullets
\usepackage{bbding} % dingbats (for bullets)
\usepackage{textcomp}
\usepackage{color,calc,graphicx,soul}
\usepackage{geometry}
\usepackage{layout}
\RequirePackage[calcwidth]{titlesec}
\usepackage{bookman}
\usepackage{fix-cm}
\let\footruleskip\undefined %will be redefined in fancyhdr
\usepackage{fancyhdr}
\usepackage[bookmarks, colorlinks=true]{hyperref}
\usepackage{makeidx}
\usepackage{hyphenat}
\usepackage{parskip}
\usepackage[most]{tcolorbox}
\usetikzlibrary{shadows}
\usepackage{environ}
\usepackage{polyglossia}
\usepackage{listings,fontspec}


\geometry{
	includeheadfoot,
	margin=2.54cm
}

%% colors 
\definecolor{primaryBack}{HTML}{4782D3}
\definecolor{primaryDark}{HTML}{14478D}
\definecolor{primaryLight}{HTML}{98BEF4}
\definecolor{primaryIntense}{HTML}{2965B9}

\definecolor{secondaryBack}{HTML}{FF8145}
\definecolor{secondaryDark}{HTML}{D64E0D}
\definecolor{secondaryLight}{HTML}{FFB795}
\definecolor{secondaryIntense}{HTML}{FF6C25}

\definecolor{thirdBack}{HTML}{FFDC45}
\definecolor{thirdDark}{HTML}{D6B00D}
\definecolor{thirdLight}{HTML}{FFEB95}
\definecolor{thirdIntense}{HTML}{FFD625}

\definecolor{complementBack}{HTML}{FFB945}
\definecolor{complementDark}{HTML}{D68A0D}
\definecolor{complementLight}{HTML}{FFD795}
\definecolor{complementIntense}{HTML}{FFAD25}

\definecolor{hidden}{HTML}{FFD7B5}

\definecolor{red}{HTML}{FF0000}
\definecolor{blue}{HTML}{0000FF}
\definecolor{green}{HTML}{009900}
\definecolor{orange}{HTML}{FFA500}
\definecolor{black}{HTML}{000000}

% section header
\titleformat{\section}[hang]{\bfseries}{
\huge\color{primaryIntense}\thesection}{10pt}{\huge\raggedleft\color{primaryIntense}}[{\titlerule[0.5pt]}]

% subsection header
\titleformat{\subsection}[hang]{\bfseries}{
\huge\color{primaryIntense}\thesubsection}{10pt}{\LARGE\raggedleft\color{primaryIntense}}
\setcounter{secnumdepth}{3}

% subsection header
\titleformat{\subsubsection}[hang]{\bfseries}{
\large\color{primaryIntense}\thesubsubsection}{10pt}{\large\raggedleft\color{primaryIntense}}


% table of contents
\pagestyle{fancy}
\hypersetup{linkcolor=primaryDark}
\maxtocdepth{subsubsection}

% colored description
%\setdescription{leftmargin=1cm,labelindent=0.4cm}
%\renewcommand{\descriptionlabel}[1]
%{\hspace{\labelsep}{\color{red}{\bfseries #1}}}
%
% color of texttt
\let\Oldtexttt\texttt
\renewcommand\texttt[1]{{\ttfamily\color{primaryDark} {\bfseries #1}}}

% chapter header
\makeatletter
\newlength\dlf@normtxtw
\setlength\dlf@normtxtw{\textwidth}
\def\myhelvetfont{\def\sfdefault{mdput}}
\newsavebox{\feline@chapter}
\newcommand\feline@chapter@marker[1][4cm]{%
\sbox\feline@chapter{%
\resizebox{!}{#1}{\fboxsep=1pt%
\colorbox{primaryBack}{\color{complementIntense}\bfseries\sffamily\thechapter}%
}}%
\rotatebox{90}{%
\resizebox{%
\heightof{\usebox{\feline@chapter}}+\depthof{\usebox{\feline@chapter}}}%
{!}{\scshape\so\@chapapp}}\quad%
\raisebox{\depthof{\usebox{\feline@chapter}}}{\usebox{\feline@chapter}}%
}
\newcommand\feline@chm[1][4cm]{%
\sbox\feline@chapter{\feline@chapter@marker[#1]}%
\makebox[0pt][l]{% aka \rlap
\makebox[1cm][r]{\usebox\feline@chapter}%
}}
\makechapterstyle{daleif1}{
\renewcommand\chapnamefont{\normalfont\Large\scshape\raggedleft\so}
\renewcommand\chaptitlefont{\normalfont\huge\bfseries\scshape\color{primaryIntense}}
\renewcommand\chapternamenum{}
\renewcommand\printchaptername{}
\renewcommand\printchapternum{\null\hfill\feline@chm[2.5cm]\par}
\renewcommand\afterchapternum{\par\vskip\midchapskip}
\renewcommand\printchaptertitle[1]{\chaptitlefont\raggedleft ##1\par}
}
\makeatother
\chapterstyle{daleif1}

%title page
\makeatletter
\newlength\drop
\newcommand*{\titleGM}{%
	\thispagestyle{empty}
	\begingroup% Gentle Madness
	\drop = 0.1\textheight
	\vspace*{\baselineskip}
	\hbox{%
		\hspace*{0.2\textwidth}%
		\hspace*{0.05\textwidth}% 
		\parbox[b]{0.75\textwidth}{%
		   \vbox{%
				 \vspace{\drop}
				 \rule{0.75\textwidth}{1pt}
				 {\Huge\bfseries\sffamily\raggedleft\color{primaryIntense}\@title\par}
				 \vskip2.37\baselineskip
				 {\Large\bfseries\sffamily\raggedleft\color{primaryDark}\@author\par}
				 \rule{0.75\textwidth}{1pt}
				 \vskip4.37\baselineskip
				 {\Large\bfseries\sffamily\raggedleft\color{primaryDark}Revision: \@date\par}
				 \vskip4.37\baselineskip
				 \includegraphics[scale=0.2]{mute-logo}
			 }
		}
	}
	\pagebreak
	\vfill
	\null
	\endgroup
}
\makeatother

%part page
\makeatletter

\renewcommand\printpartname{}
\renewcommand\printpartnum{}
\renewcommand{\parttitlefont}{\normalfont\raggedleft\Huge\scshape\color{primaryIntense}}
\newcommand{\partAbstract}[1]{\gdef\@partAbstract{\leftskip5em\normalsize\color{black}\textit{#1}}}
\renewcommand{\printparttitle}[1]{%
\raggedleft
\rule{0.75\textwidth}{1pt}
\vskip1.37\baselineskip
\parttitlefont Part \thepart: #1
\color{black}
\rule{0.75\textwidth}{1pt}
\vskip1.37\baselineskip
\@partAbstract\vfil
}
\makeatother

%paragraph
\setlength{\parindent}{0pt}
\setlength{\parskip}{0.5cm plus3mm minus2mm}

%table

\newcommand{\tempcaption}{}
\newenvironment{myTable}[2]{%
  \gdef\tempcaption{#1}%
  \begin{center}%

    \begin{tcolorbox}[
      colback=primaryLight,
      colframe=primaryIntense,
      left=1mm,right=1mm,top=1mm,bottom=1mm,boxsep=0mm,
      toptitle=0.5mm,
      bottomtitle=0.5mm,
      center title,
      title=\tempcaption,
    ]%
    \centering
    \begin{tabular}{#2}%
}{%
    \end{tabular}%
    \end{tcolorbox}%
  \end{center}%
}

%listing
\lstset{
  frame=tb,
  language=c++,
  aboveskip=3mm,
  belowskip=3mm,
  showstringspaces=false,
  columns=flexible,
  basicstyle={\small\ttfamily},
  %backgroundcolor=\color{primaryLight}, % Set the background color for the snippet - useful for highlighting
  firstnumber=1, % Line numbers begin at line 1
  stepnumber=5,
  frame=lines, % Frame around the code box, value can be: none, leftline, topline, bottomline, lines, single, shadowbox
  %frameround=tttt, % Rounds the corners of the frame for the top left, top right, bottom left and bottom right
  numbers=left,
  numbersep=10pt, % Distance of line numbers from the code box
  numberstyle=\tiny\color{primaryDark}, % Style used for line numbers
  rulecolor=\color{primaryDark}, % Frame border color
  keywordstyle=\color{blue},
  commentstyle=\color{green},
  stringstyle=\color{red},
  breaklines=true,
  breakatwhitespace=true,
  tabsize=2
}

\lstset{
  emph={Vec, Vec2, Circle, Memc, Str, FREPA, SQLColumn, SQLValues, TextData, File, 
	      Memx, SQL, GuiObjs, Window, Button, TextLine, TextNode},
  emphstyle={\color{blue}\bfseries}
}

\renewcommand{\labelitemi}{\SquareShadowBottomRight}
\renewcommand{\labelitemii}{\tiny\Square}

% special blocks
\newtheorem{example}{Example}
\newtheorem{remember}{Remember This!}

\tcbset{
myNote/.style={
  enhanced,
  colback=complementLight,
  colframe=complementDark,
  fonttitle=\scshape,
  title={Note},
  title style={fill=complementDark},
  }
}

\newtcolorbox{note}{myNote}

\tcbset{
myInfo/.style={
  enhanced,
  colback=complementLight,
  colframe=complementDark,
  fonttitle=\scshape,
  title={Info},
  title style={fill=complementDark},
  }
}

\newtcolorbox{info}{myInfo}

%\tcbset{
%myWorld/.style={
  %enhanced,
  %colback=thirdLight,
  %colframe=thirdIntense,
  %fonttitle=\scshape,
  %title={De echte wereld},
  %title style={fill=thirdDark},
  %coltitle=thirdLight,
  %}
%}
%
%\newtcolorbox{praktijk}{myWorld}

\tcbset{
myWarning/.style={
  enhanced,
  colback=thirdLight,
  colframe=thirdIntense,
  fonttitle=\scshape,
  title={Warning!},
  title style={fill=thirdDark},
  coltitle=thirdLight,
  }
}

\newtcolorbox{warning}{myWarning}

\tcbset{
myExercise/.style={
  enhanced,
  colback=primaryLight,
  colframe=primaryDark,
  fonttitle=\scshape,
  title={Time for Action},
  title style={fill=primaryDark},
  }
}

\newtcolorbox{exercise}{myExercise}

\newcommand*\tick{\item[\Checkmark]}
\newcommand*\fail{\item[\XSolidBrush]}
\newcommand*\pencil{\item[\PencilRightDown]}

\tcbset{
myTest/.style={
  enhanced,
  colback=secondaryLight,
  colframe=secondaryDark,
  fonttitle=\scshape,
  fontlower=\itshape,
  title={Test Jezelf},
  title style={fill=secondaryDark},
  collower=hidden,
  }
}

\newtcolorbox{test}{myTest}

\lstnewenvironment{code}
	{
		\lstset{language={[Visual]C++}, columns=fixed}
	}
	{
	}
%
\newcommand{\eeClass}[1]{\texttt{#1}}
\newcommand{\eeFunc}[1]{\texttt{#1}}
\newcommand{\eeOpp}[1]{\texttt{#1}}

\newcommand{\video}[1]{}

\author{Mute (http://mutecode.com)}

\setmainlanguage{dutch}
\PolyglossiaSetup{dutch}{indentfirst=false}

\graphicspath{ {images/} }



\title{Learning C++ with Esenthel}
\begin{document}
\titleGM
\tableofcontents

%\part{C++ en Esenthel}
%\chapter{Introduction}
In the previous chapters you have seen everything you need to create a simple 2D game. But how do you put a large project together in an orderly way? There is really no simple answer to that. You learn by practice, and not everyone agrees on the best method. Still, there are some rules that may make it easier for sure. And when you work in a group, the lead programmer will usually impose some rules that everyone has to follow. These are not necessarily good or bad rules, but they work as long as everyone follows them.

In this part of the course you will create a clone of the famous Tetris game. You will learn to develop a project step by step  without losing sight of the whole.

\begin{note}
Tetris consists of blocks which in turn consist of squares. When we talk about blocks in this course, we mean the entire block, not the squares it is made of. When a square is mentioned, we're discussing the squares that make up a block.
\end{note}

\section{Setup}
Open the `Tetris\_start' project. In it, you will find the graphics, sounds and fonts that we will use. A blank app `Tetris' is also provided, but we are not going to use it just yet.

\begin{enumerate}
	\item Create a library at the highest level of the explorer. You do this by right clicking and choosing `new library'. Name this library `Tetris parts'. A library is a green folder. The code in a library can be used from any application within your project, just like the library `Esenthel Engine' which is always present.
	\item Create a new application (blue folder). Call it `square tester'.
	\item in the application `square tester', create a code file `main'. 
	\item In the library `Tetris parts', create a new folder (yellow) called `definitions'.
	\item Mark `square tester' as the active application.
\end{enumerate}
	
Copy the code in `Tetris/initState' to `square tester/main'. Remove this line: 
	
\begin{code}
D.full(true);
\end{code}

This makes it easier to terminate your application when something goes wrong.	

\section{Constants}
In the previous chapter you learned about constants. It is a good idea to create some important constants before you start on the actual code. You can use them anywhere in your code and easily adjust their values if necessary. Create a new code file `constants' in the folder `Tetris parts/definitions'.

To be able to change the name of the app later on, create a constant \eeClass{Str} with a provisional name.

\begin{code}
C Str APP_NAME = "Tetris";
\end{code}

The size of the standard application window is not ideal for this game. A size in pixels is required for this. Declare a constant \eeClass{int} for this purpose.

\begin{code}
// The window size on the screen, in pixels
C int WINDOW_WIDTH  = 900;
C int WINDOW_HEIGHT = 800;
\end{code}

Tetris consists of rows and columns. we also define these:

\begin{code}
// This impacts the playing field
C int SQUARES_PER_ROW = 10;
C int ROWS            = 15;
\end{code}

It is also possible to add a few constants for the scoring system. There is a fixed number of levels, and we know how much points will be rewarded for a line and a level.

\begin{code}
// The scoring system uses these
C int POINTS_PER_LINE  =  525;
C int POINTS_PER_LEVEL = 6300;
C int NUM_LEVELS       =    5;
\end{code}

The speed of the game goes up with each level. This change can also be defined as a constant.

\begin{code}
// The speed will increase every level
C float INITIAL_SPEED = 1.0;
C float SPEED_CHANGE  = 0.1;
\end{code}

When playing tetris, there is a short period after moving a block down in which you are able to move it sidewards. This period has to be defined.

\begin{code}
// The time a block can be slided to the side
// When it hits bottom
C float SLIDE_TIME = 0.25;
\end{code}

We also have to determine the size of the playing field. This is done by defining the position of the lower left corner, and defining the size of the rectangle containing the playing field.

\begin{code}
// The area reserved for the playing field 
C Vec2 GAMEAREA     (-0.8, -0.8);
C Vec2 GAMEAREA_SIZE( 1.0,  1.4);
\end{code}

A new block will always appear at the top of the screen. This position can be deducted from information we already have:  SQUARES\_PER\_ROW and ROWS. In addition, there is a waiting position, top right of the playing field. You might notice we are not using \eeClass{Vec2}, but \eeClass{VecI2}. This is a vector which fits only integers. We do not want Tetris blocks midway between two positions, so there is no need for floats here.

\begin{code}
// Position for the current and next block
C VecI2 STARTPOS(SQUARES_PER_ROW / 2, ROWS - 1);
C VecI2 WAITPOS (SQUARES_PER_ROW + 4, ROWS - 3);
\end{code}

Now it is possible to calculate the size of a square with the information we already have. This has the advantage that we can modify the foregoing constants later, the size of a square being automatically adjusted.

\begin{code}
// The size of a square
C float SQUARE_SIZE = GAMEAREA_SIZE.x / SQUARES_PER_ROW;
\end{code}

\begin{note}
Of course it will rarely happen that you precisely know which constants are needed when you're just starting out on a project. In practice you will usually add a lot of constants while you are working on your project.
\end{note}

\section{Enums}
Create the code file `enumerations' in  `Tetris parts/definitions'. Add two enumerations that will be useful in your project. Firstly, there is the block type. Tetris has square blocks, T-blocks and so on. A list might look like this:

\begin{code}
enum BLOCK_TYPE
{
   BT_SQUARE     ,
   BT_T          ,
   BT_L          ,
   BT_BACKWARDS_L,
   BT_STRAIGHT   ,
   BT_S          ,
   BT_BACKWARDS_S,
   BT_NUM        , // number of block types used in the game
   BT_BACKGROUND , // special case, only for background
   BT_WALL       ,      
}
\end{code}

The last three values ​​deserve extra attention. The value `BT\_NUM' is useful because the number which represents that value is equal to the highest value + 1. This makes it easy to use a random function. Since the result of a random function does not include the maximum value, we will be able to use this kind of code later on in our application:

\begin{code}
blockType type = Random(BT_NUM);
\end{code}

So why is BT\_NUM not the last value in the list? Well, the last two values ​​are special cases. There will be special squares to draw the background and the borders of the game. And the color of such a square is determined by the block type. Because we do not want to use those types in the actual game, BT \_NUM is but before these values.

A second enumeration is used to determine the possible directions in which a block can move. A block can not upwards, only to the left, the right, or down. Blocks which are already down, do not have a direction anymore.

\begin{code}
enum DIRECTION
{
   D_LEFT ,
   D_RIGHT,
   D_DOWN ,
   D_NONE ,
}
\end{code}


\partAbstract{
Leren programmeren is een stuk leuker wanneer je wat resultaat ziet op het scherm. Je kan Esenthel gebruiken voor 2D en 3D applicaties, maar 2D is wel eenvoudiger om mee te beginnen. Op het einde van dit deel zal je shapes, afbeeldingen en text kunnen manipuleren en op het scherm tonen. En om het nog leuker te maken leer je ook geluid afspelen.
}
\part{2D Concepts}
\chapter{Positions in 2D}
\label{chapter:positions}

Positions in 2D have an x- and a y- coordinate, just as a point in a graph. In Esenthel the zero point is the middle of the screen. Positive values on the X-axis are to the right, negative values to the left. Positive values on the Y-axis are in the upper half of the screen, negative values on the lower half.

\begin{figure}[h]
\centering
\includegraphics[width=0.7\linewidth]{images/2Dpositions.png}
\caption[]{2D coordinates.}
\label{fig:pos2D}
\end{figure}

The class \eeClass{Vec2} is used to represent a 2D coordinate. There are several ways to set the x and y values:

\begin{code}
Vec2 pos;
pos.x =  0.1;       // set the x value to  0.1
pos.y = -0.3;       // set the y value to -0.3
pos   =  0.5;       // both x and y are set to 0.5
pos.set(0.1, -0.3); // use the function set(float x, float y) 
                    // to assign both x and y
\end{code}

\begin{exercise}
Where would these positions be located on the screen?
\end{exercise}

\video{https://www.youtube.com/embed/fbmTzkUnv3E}

\section{Show a position on the screen}

Although the class \eeClass{Vec2} is mainly used to calculate positions, it is also possible to show it on the screen. This can be done with the function \eeFunc{draw(Color)}. The argument should contain the color in which the coordinate must be drawn.

\begin{code}
Vec2 p1(0.2, 0.4); // Create a point p1, assigning x and y 
                   // with the constructor.
Vec2 p2;           // Create a point p2.

void InitPre()
{
   EE_INIT();
}

bool Init()
{
   p2.set(-0.2, -0.5); // assign values to p2 with the set function
   return true;
}

void Shut() {}

bool Update()
{
   if(Kb.bp(KB_ESC)) return false;
   
   return true;
}

void Draw()
{
   D.clear(BLACK); // Clear the screen
   p1.draw(RED  ); // draw p1 in red  
   p2.draw(BLUE ); // draw p2 in blue
   Vec2(0 ,0).draw(GREEN); // create a temporary object and draw 
	                         // it in green
}

\end{code}

\begin{exercise}
Write this code in the editor. Don't copy/paste it: you won't learn anything from copying code. Make sure it runs without error. If not, compare your code with this example. Learning to understand errors is very important. Try to understand what went wrong and why.
\end{exercise}

\section{Math}
You can do math with \eeClass{Vec2}, just as you can with plain numbers. It is possible to assign x and y values directly, but you can also do math with the class itself. In this case the operator will be applied to x as well as y:

\begin{code}
Vec2 p1(0.1,  0.3);
Vec2 p2(0.3, -0.1);
Vec2 p3 = p1 + p2;  // x: 0.4 , y: 0.2
p3 -= 0.1;          // x: 0.3 , y: 0.1
p3 *= 2;            // x: 0.6 , y: 0.2
p3 = p1 / 2.f;      // x: 0.05, y: 0.15 
\end{code}

\section{Points to remember}
The position \eeClass{Vec2(0,0)} always stands for the middle of the screen. But the borders are not always clear. Not every computer screen has the same size, but you probably want to know the border values are. For example, you want to draw something in the right corner of the screen. For this reason, Esenthel provides a few functions in the object \texttt{D} (display). You can request the width of the screen with \eeFunc{D.w()} and the height of the screen with the function \eeFunc{D.h()}. This makes it very easy to calculate the next points:

\begin{code}
Vec2 middle(0,0);
Vec2 left(-D.w(), 0);
Vec2 right(D.w(), 0);
Vec2 rightUpperCorner(D.w(), D.h());
Vec2 leftUpperCorner(-D.w(), D.h());
\end{code}

If you'd like to draw a point at a distance of 0.1 from the left upper corner, you could do that like this:

\begin{code}
Vec2(-D.w() + 0.1, D.h() - 0.1).draw(PINK);
\end{code}

\section{Summary}
\begin{itemize}
\item Points in 2D have an x- and y-coordinate. In Esenthel you will use the class \eeClass{Vec2} to store a point.
\item The class \eeClass{Vec2} has a function \eeFunc{draw(Color)} to draw a point on the screen.
\item You can do math with \eeClass{Vec2}, just as with an ordinary number.
\item \eeClass{Vec2(0,0)} will always be the middle of the screen.
\item The borders of the screen can be calculated with \eeClass{D.w()} and \eeClass{D.h()}.
\end{itemize}

\begin{exercise}
Create an application which draws these points:
\begin{enumerate}
	\item A white point in the middle of the screen.
	\item A red point on 0.1 units away from the left border of the screen.
	\item A blue point on 0.2 units from the upper right corner.
	\item A yellow dot on 0.35 units from the lower border of the screen, on 2/3 of the total screen width.
\end{enumerate}
\end{exercise}

\video{https://www.youtube.com/embed/a2k420oTySU}
\chapter{Interactie}
De meeste interacties met een programma gebeuren via het toetsenbord en de muis. (Mobile devices gebruiken vooral touches, maar die komen later aan bod.) In dit hoofdstuk overlopen we de verschillende mogelijkheden.

\section{Muis Interacties}
De class \eeClass{Mouse} vind je in Esenthel Engine $\Rightarrow$ Input $\Rightarrow$ Mouse. Computers zijn niet voorzien om meer dan \'e\'en mouse pointer te tonen, dus Esenthel voorziet alvast een object van de class: \eeClass{Ms}. Het is dus onnodig om zelf een object van de class \eeClass{Mouse} te maken.

Als je even naar de beschikbare functies kijkt in deze class, dan zie je dat er heel wat mogelijkheden zijn. Meestal zijn we vooral ge\"interesseerd in `clicks' en positie.

\subsection{Positie}
De positie van de muis op het scherm is een \eeClass{Vec2}. Je kan de huidige positie opvragen via de functie \eeFunc{Ms.pos()}. De volgende code toont een \eeClass{Vec2} op het scherm. De positie is steeds gelijk aan de positie van de muis. 



\begin{code}
Vec2 mousePos;

void InitPre()
{
   EE_INIT();
}

bool Init()
{   
   return true;
}

void Shut() {}

bool Update()
{
   if(Kb.bp(KB_ESC)) return false;  
   mousePos = Ms.pos();  
   return true;
}

void Draw()
{
   D.clear(BLACK);
   mousePos.draw(RED);
}
\end{code}

\begin{exercise}
Pas het bovenstaande voorbeeld aan, zodat mousePos iets onder de positie van de muis wordt getoond.
\end{exercise} 

\video{https://www.youtube.com/embed/muw77vAjqgs}

\subsection{Clicks}
Je kan via het \eeClass{Ms} object vanalles te weten komen over de status van de muis. Aangezien we meestal meer dan \'e\'en knop op een muis hebben, zal je als argument steeds een nummer moeten opgeven. De linkerknop heeft index 0, de rechterknop index 1. Enkele voorbeelden: 

\begin{code}
Ms.bp(0); // true als de linkerknop ingedrukt werd in dit frame
Ms.br(1); // true als de rechterknop los gelaten werd in dit frame
Ms.b (0); // true zolang de linkerknop ingedrukt is
Ms.bd(0); // true wanneer in dit frame een dubbelklik plaatsvond
\end{code}

Het verschil tussen \eeFunc{Ms.bp()} en \eeFunc{Ms.b()} is subtiel:
\begin{itemize}
	\item \eeFunc{Ms.bp()} Betekent dat de knop net werd ingedrukt. In de volgende update kan de muis misschien ook nog ingedrukt zijn, maar dan zal het resultaat van deze functie false zijn.
	\item \eeFunc{Ms.b()} Deze functie heeft true als resultaat zolang de knop niet los gelaten wordt.
\end{itemize}

Het volgende programma maakt dit verschil duidelijk:

\begin{code}
Vec2 mouse1Pos;
Vec2 mouse2Pos;

void InitPre()
{
   EE_INIT();
}

bool Init()
{   
   return true;
}

void Shut() {}

bool Update()
{
   if(Kb.bp(KB_ESC)) return false;
   
   if(Ms.bp(0)) {
      mouse1Pos = Ms.pos();
   }
   
   if(Ms.b(1)) {
      mouse2Pos = Ms.pos();
   }
   
   return true;
}

void Draw()
{
   D.clear(BLACK);
   mouse1Pos.draw(RED  );
   mouse2Pos.draw(GREEN);
}
\end{code}

Het rode punt zal enkel een nieuwe positie krijgen op het moment dat je de linker muisknop indrukt. Het groene punt daarentegen blijft de muis volgen zolang de rechter muisknop ingedrukt blijft. Tijdens het ontwikkelen van een programma zal je steeds moeten kiezen welk van beide functies het meest geschikt is.

\begin{exercise}
Pas het bovenstaande voorbeeld aan, zodat de tweede muispositie enkel zichtbaar is wanneer de rechter muisknop ingedrukt is.
\end{exercise} 


\video{https://www.youtube.com/embed/wXoxQpOvZ5k}

\subsection{Muiswiel}
Tegenwoordig heeft een muis meestal ook een wieltje. daarvoor bestaat geen absolute positie, maar esenthel houdt wel bij hoeveel het muiswiel gedraaid werd tijdens de huidige update. Een afstand die binnen een frame werd afgelegd heet een `delta'. Je kan zo spreken over de delta van de tijd, de delta van een beweging enzovoort. De functie die je nodig hebt om de delta van het muiswiel te weten is \eeFunc{Ms.wheel()}. Het resultaat is een float.

Het volgende voorbeeld laat een punt vertikaal bewegen als via het muiswieltje. In de update functie wordt de positie van het punt aangepast. We stellen nu geen nieuwe positie in zoals bij de vorige oefening. We passen slechts de positie op de y as aan. De functie \eeFunc{Ms.wheel()} geeft enkel de verplaatsing weer sinds de vorige update. Als je naar boven beweegt dan is dat een (zeer klein) positief getal. Beweeg je naar beneden, dan is dit getal negatief.

\begin{code}
Vec2 mousePos;

void InitPre()
{
   EE_INIT();
}

bool Init()
{   
   return true;
}

void Shut() {}

bool Update()
{
   if(Kb.bp(KB_ESC)) return false;  
   mousePos.y += Ms.wheel() * 0.1;  
   return true;
}

void Draw()
{
   D.clear(BLACK);
   mousePos.draw(RED);
}
\end{code}

\begin{exercise}
Pas het bovenstaande voorbeeld aan zodat, wanneer de linker muisknop ingedrukt is, de horizontale positie wordt aangepast. Is de rechter muisknop ingedrukt, dan pas je de vertikale positie aan.
\end{exercise} 
\video{https://www.youtube.com/embed/QDSWHMA-WRE}

\subsection{Cursor}
De muiscursor is niets anders dan een afbeelding die op het scherm getoond wordt. Je kan aan het begin van je programma (bijvoorbeeld in de Init functie) die afbeelding wijzigen:

\begin{code}
  Ms.cursor(Images( --drop hier de afbeelding-- ));
\end{code}

\begin{exercise}
Pas de cursor aan in de vorige oefening. Er staan enkele geschikte afbeeldingen in de map gfx $\Rightarrow$ mouse. Toon de afbeelding `BGNormal' wanneer het programma start. In de update functie zorg je er voor dat, enkel wanneer een muisknop ingedrukt is, `BGMove' getoond wordt.
\end{exercise}

\subsection{Overige functies}
De class \eeClass{Mouse} bevat nog veel meer functies. We bespreken ze niet allemaal in dit hoofdstuk. Als oefening zoek je zelf uit waar de volgende functies voor dienen: 

\begin{itemize}
	\item \eeFunc{Ms.hide()}
	\item \eeFunc{Ms.show()}
	\item \eeFunc{Ms.eat()}
\end{itemize}

\section{Keyboard Interacties}
Ook het toetsenbord wordt dikwijls gebruikt om interactie te sturen. Later, in het hoofdstuk over de GUI, zal je zien hoe we een toetsenbord gebruiken om tekst in te typen. In dit hoofdstuk controleren we enkel de status van de toetsen.

Elke toets heeft een naam. Via die naam kan je een bepaalde toets aanspreken. Hierbij moet je wel voor ogen houden dat die naam overeenkomt met de positie van die toets op een qwerty toetsenbord. Werk je op azerty en je wil de status van de `Z' toets weten, dan zal je naar de status van de `W' toets moeten vragen. Dit lijkt vreemd, maar bij de aansturing van een spel is vooral de positie van de toets belangrijk. Stel je voor dat de typische WASD aansturing de posities op een azerty toetsenbord zou volgen!

Verder is het controleren van een toets op je keyboard net zoals een muis klik. Je zal zien dat de functies bijna gelijk zijn.

\subsection{Key Functions}

Bekijk de volgende functies even:

\begin{code}
Kb.bp(KB_N) // true als de N toets werd ingedrukt tijdens de huidige frame.
Kb.br(KB_E) // true als de E toets werd losgelaten tijdens de huidige frame.
Kb.b (KB_R) // true als de R toets momenteel ingedrukt is
Kb.bd(KB_D) // true bij een dubbelklik op toets D
\end{code}

We gebruiken nu het object \eeClass{Kb} in plaats van \eeClass{Ms}, maar de functies zijn precies zoals die voor een muisklik. Waar je bij het \eeClass{Ms} object het nummer van de toets moest ingeven, gebruik je nu een code. Deze code begint steeds met KB\_. Dan volgt meestal een letter of een cijfer. Andere mogelijkheden zijn:

\begin{center}
\begin{tabular}{|c|c|c|c|c|}
\hline
\multicolumn{1}{|l|}{{\bf Functions}} & \multicolumn{1}{l|}{{\bf Control}} & \multicolumn{1}{l|}{{\bf Modifiers}} & \multicolumn{1}{l|}{{\bf Arrows}} & \multicolumn{1}{l|}{{\bf Numpad}} \\ \hline
KB\_F1                                & KB\_ESC                            & KB\_LCTRL                            & KB\_LEFT                          & KB\_NPDIV                         \\ \hline
KB\_F2                                & KB\_ENTER                          & KB\_RCTRL                            & KB\_RIGHT                         & KB\_NPENTER                       \\ \hline
\ldots                                   & KB\_SPACE                          & KB\_LSHIFT                           & KB\_UP                            & KB\_NP1                           \\ \hline
KB\_F12                               & KB\_BACK                           & KB\_RSHIFT                           & KB\_DOWN                          & KB\_NP2                           \\ \hline
                                      & KB\_TAB                            & \ldots                                  &                                   & \ldots                               \\ \hline
\end{tabular}
\end{center}



Een volledig overzicht met alle toetsen vind je in de map Esenthel Engine $\Rightarrow$ Input $\Rightarrow$ Input Buttons.

\begin{exercise}
Maak een programma met een punt op het scherm. Via de pijltjestoetsen kan je dit punt verplaatsen. Elke keer een pijltjestoets wordt ingedrukt, verplaats je het punt 0.1 units in de gewenste richting.

Het punt zelf teken je in het groen op je scherm, tenzij de spatiebalk ingedrukt is. Dan verschijnt het punt in het rood.
\end{exercise}

\subsection{Graduele wijzigingen}
\label{chapter:keyboardInteractie}
De functie \eeFunc{Kb.bp()} gebruik je vooral voor plotse wijzigingen. Je wil bijvoorbeeld een window openen, een object toevoegen, een menu tonen of het programma verlaten. In het volgende voorbeeld zie je hoe je deze functie zou kunnen gebruiken om een help window te tonen:

\begin{code}
if(Kb.bp(KB_F1)) HelpWindow.show();
\end{code}

Stel je voor dat je hier de functie \eeFunc{Kb.b()} zou gebruiken. In elke frame dat de toets F1 ingedrukt is, zou het help window opnieuw getoond worden. Op een snelle computer is dat ongeveer 60 keer per seconde. 

Toch is de twee functie heel bruikbaar, maar dan voor waarden die geleidelijk moeten veranderen. Je blijft een waarde dan wijzigen zolang de toets ingedrukt is:

\begin{code}
if(Kb.b(KB_RIGHT)) point.x += 0.01;
\end{code}

De x waarde van het punt zal tijdens elke frame iets groter worden. het punt verplaatst zich dus geleidelijk aan naar rechts.

\begin{exercise}
Maak een programma aan de hand van het laatste voorbeeld. Zorg dat het punt, via de pijltjestoetsen, in vier richtingen kan bewegen.
\end{exercise}

\subsection{Delta time}
De oefening die je hierboven maakte heeft een groot probleem: elke frame wordt de positie aangepast. Stel nu dat je je programma op twee computers test: de eerste computer heeft een snelle grafische kaart en haalt 60FPS. De tweede computer is al wat ouder en haalt slechts 30FPS. De beweging zal op de eerste computer dubbel zo snel verlopen! Voor een kleine oefening is dat geen probleem, maar bij een echte game is dat niet wenselijk.

De oplossing is de \textbf{delta time}. De delta time is gelijk aan de tijd die verstreken is sinds de vorige frame. \textsl{(Denk even terug aan de mouse wheel delta: de afstand die het muiswieltje aflegde sinds de vorige frame.)} De delta time kan je opvragen via de volgende functie:

\begin{code}
Time.d();
\end{code}

Wil je een object verplaatsen aan een snelheid van 1 unit per seconde? Dan kan je de volgende code gebruiken:

\begin{code}
if(Kb.b(KB_RIGHT)) point.x += 1 * Time.d();
\end{code}

\begin{exercise}
Plaats de vorige oefening aan, zodat je rekening houdt met de time delta.

\textbf{Uitbreiding:} gebruik in plaats van het getal 1 een float variabele die gelijk is aan 1. Via twee zelf te kiezen toetsen kan je deze waarde verhogen en verlagen.
\end{exercise}




\chapter{Tekst}

Ongetwijfeld wil je in een programma ook tekst op het scherm tonen. Voor letters, woorden en zelfs hele zinnen bestaat er de class \eeClass{Str}. Om het duidelijk te houden spreek je dat uit als `string'.

Je kan op de volgende manieren tekst in een \eeClass{Str} plaatsen:

\begin{code}
Str tekst("hello world"); // via de constructor
tekst  = "hello "; // via toekenning
tekst += "world" ; // via de plus assignment operator
\end{code}

\textit{De tekst tussen de quotes noemen we in het vakjargon een `string literal': een letterlijke string. Als je niets moet aanpassen aan een tekst, dan kan je dikwijls rechtstreeks met string literals werken.}

\section{Tekst op het scherm tonen}
Nu wil je een tekst dikwijls op het scherm tonen. In tegenstelling tot de class \eeClass{Vec2}, heeft \eeClass{Str} geen `draw' functie. Tekst op het scherm plaatsen gaat daarom via het object \eeClass{D}:

\begin{code}
Str myString;
myString = "hello world";
D.text(Vec2(0, 0), myString); // plaatst een tekst in het midden van het scherm
D.text(Vec2(0, -0.1), "Een string literal"); // het kan dus ook zonder Str
\end{code}

Het tweede argument van de functie \eeFunc{text} is de tekst die je op het scherm wil zetten. Het eerste argument is de positie. Je weet ondertussen genoeg over co\"ordinaten om te begrijpen waarom dit een \eeClass{Vec2} is.

\begin{exercise}
Maak om dit in te oefenen een programma dat de woorden `links', `rechts', `boven' en `onder' op een logische plaats op het scherm plaatst.
\end{exercise}

\begin{exercise}
Maak een programma dat de muis verbergt en op de positie van de muis het woord `mouse' toont.
\end{exercise}

\section{Getallen en tekst combineren}
Je kan getallen en tekst niet zomaar combineren. Voor je computer zijn 42 en ``42'' iets helemaal anders. Het eerste is een getal, het tweede een tekst die toevallig de characters 4 en 2 bevat. 

Nu, de \eeClass{Str} class in Esenthel helpt je. Die laat toe dat je getallen toewijst of toevoegt aan een string. Zo kan je bijvoorbeeld toch het volgende schrijven:

\begin{code}
int i = 42;
Str myString =  42;
myString    +=   i; 
myString    +=  42; 
myString    += 4.2; 
\end{code}

Maar wanneer je een string literal wil combineren met een nummer, dan beginnen de problemen. De compiler zal ze eerst willen samenvoegen, voor dat ze in de string gezet worden. En dat kan niet. Dit werkt dus niet:

\begin{code}
Str myString;
myString = "score: " + 42; 
myString = 42 + "score: "; 
\end{code}

We gaan hier niet in op de details, maar er bestaat een eenvoudige oplossing: een \eeClass{Str} object \eeClass{S} die voorzien is in de engine. Wanneer je een foutmelding krijgt bij het combineren van tekst met integers of floats, begin dan met \eeClass{S +}. 

\begin{code}
Str myString;
myString = S + 42  + " is your score"; 

int myScore = 10;
D.text(Vec2(0, 0), S + "score: " + myScore);

// more complex
D.text(Vec2(0, 0), S + "score: " + myScore + " / 10");
\end{code}

\begin{exercise}
Maak een programma met een int `score'. Telkens als je op de spatiebalk drukt, dan verhoogt de score met \'e\'en punt. Je toont de score ook op het scherm.
\end{exercise}

\section{Een Vec2 als tekst weergeven}
Om de werking van je programma te controleren is het soms handig om de x- en y-co\"ordinaat van een Vec2 op het scherm te zetten. Aangezien dat getallen zijn, kan je dat op de volgende manier doen:

\begin{code}
Vec2 pos = Ms.pos();
D.text(Vec2(0, 0), S + "x: " + pos.x + " y: " + pos.y);
\end{code}

Maar omdat elke programmeur zoiets regelmatig nodig heeft, beschikt de class \eeClass{Vec2} ook over een functie die dat eenvoudiger maakt:

\begin{code}
Vec2 pos = Ms.pos();
D.text(Vec2(0, 0), S + pos.asText());
// of ook rechtstreeks:
D.text(Vec2(0, 0), Ms.pos().asText());
\end{code}

\begin{exercise}
Maak nog eens een programma met een punt dat je via de pijltjestoetsen kan aanpassen. Toon de positie van dat punt op het scherm.
\end{exercise}

\section{Tekstopmaak}
\label{chapter:tekstopmaak}
Waarschijnlijk wil je de standaardweergave van tekst wel aanpassen als je een eigen game maakt. Dat is op zich niet moeilijk. Je maakt daarvoor een nieuw font, door rechts te klikken in de filetree en `New Font' te kiezen. Als je dat font opent kan je zowat alles aanpassen. Het meest belangrijke is daar de naam in het vak `System Font'.

Vervolgens maak je een nieuwe textStyle. Daarmee kan je de kleur en aligment van je font wijzigen. \textsl{(In een textStyle zie je onderaan een dropdown menu `Font'. Daar kan je aangeven welk van je font dat deze stijl moet gebruiken.)} Je kan ook meer dan \'e\'en textStyle maken van hetzelfde font.

Als je nu je eigen textStyle wil gebruiken om tekst op het scherm te tonen, dan kan je een twee versie van \eeFunc{D.text()} gebruiken:

\begin{code}
D.text(*TextStyles( -- drop style here -- ), Vec2(0, 0), "tekst");
\end{code}

Je sleept je textStyle tussen de haakjes van TextStyles(). Onthoud ook dat er een asterisk voor TextStyles staat. Later leer je waarom dat zo moet, maar het werkt in ieder geval niet als je die vergeet.

\begin{exercise}
Experimenteer met de verschillende mogelijkheden om tekst vorm te geven. Zet minstens vier teksten op het scherm, en gebruik voor elke tekst een andere stijl.
\end{exercise}
\chapter{Shapes}

Once you know how to show a dot on the screen, other shapes are easy. (If you don't, review chapter \ref{chapter:positions}.) Most mathematical shapes can be drawn on the screen. The classes to do that are in the green folder Esenthel Engine $\Rightarrow$ Math $\Rightarrow$ Shapes.

Not all available shapes are intended for display in 2D. Some classes, like \eeClass{Ball} and \eeClass{Tube}, are intended for 3D development. For now, we will focus on 2D shapes like \eeClass{Circle}, \eeClass{Edge} (line), \eeClass{Quad}, \eeClass{Rectangle} and \eeClass{Triangle}.

\section{Circle}
To draw a circle you need a radius(r) and a position(pos). The radius is a \eeClass{float}, the position a \eeClass{Vec2}. There is more than one way to pass these to a circle:

\begin{code}
// using the constructor, with r(float), pos.x(float), pos.y(float)
Circle c(0.1, 0, 0);

// using the constructor, with r(float), pos(Vec2)
Circle c(0.1, Vec2(0, 0));

// during the coarse of the application
c.set(0.1, 0, 0);
c.set(0.1, Vec2(0, 0));

// directly changing the variables
c.r = 0.1;
c.pos = Vec2(0, 0);
\end{code}

\subsection{Methods}
The set function aside, there are also methods available to retrieve the current area or perimeter, and to draw the circle on the screen.

\begin{code}
// retrieve area and perimeter
float a = c.area();
float b = c.perimeter();

// draw a blue circle on the screen
c.draw(BLUE);

// draw the perimeter only
c.draw(BLUE, false);
\end{code}

There's something remarkable about the \eeFunc{draw} method! It can be used with one as well as with two methods. To see how this is possible, look at the declaration of this method, which can be found at Esenthel Engine $\Rightarrow$ Math $\Rightarrow$ Shapes $\Rightarrow$ Circle:

\begin{code}
void draw(C Color & color, Bool fill = true, Int resolution = -1) C;
\end{code}

The first argument is a \eeClass{Color}. The second argument is a \eeClass{bool} named `fill'. You might suspect this means whether or not you desire to draw the circle as a perimeter or an area. And of course you're right. Only, the argument doesn't stop there: it has a value assigned ( = true). This is called a default value. If you agree with the default, you don't gave to explicitly pass true as an argument. Only when you don't agree, you will have to pass false.

\begin{note}
The third argument (resolution) is also optional. Experiment with several values to find out what is does. (Try values like 2, 3, 8, 14, \ldots)
\end{note}

\subsection{Math}
You can do math with circles. There are operators like \eeOpp{+=}, \eeOpp{-=}, \eeOpp{/=} and \eeOpp{*=}, which can be used to alter the radius or the position. How do you know which value will be altered? Well, if you add a float to a circle, the radius change. When you add a \eeClass{Vec2} this will later the position.

\begin{code}
Vec2 pos(0.1, 0);
Circle c(0.1, pos);

// move the circle 0.1 units to the right
c += pos;
// move the circle 0.2 units down
c -= Vec2(0, 0.2);
// double the radius
c *= 2;
\end{code}

\subsection{Excercises}
Recreate the folowing image by drawing circles on the screen. If you like a challenge, try creating a more fitting mouth with the method \eeFunc{drawPie}.

\begin{figure}[h]
\centering
\includegraphics[width=0.4\linewidth]{../images/circle_exercise}
\caption[]{Pay attention to the eyes!}
\label{fig:pos2D}
\end{figure}

\section{Edge2}
An \eeClass{Edge2} is a line. Just as with \eeClass{Vec2}, the '2' is important. The code will still be valid when you use an \eeClass{Edge} instead of an \eeClass{Edge2}, but that class is intended for drawing in 3D instead. To define an \eeClas{Edge2}, you need two points (\eeClass{Vec2}), being the beginning and the end. Esenthel offers two methods to pass these to the object:

\begin{code}
Edge2 e1;
Edge2 e2;

// using newly created vectors ...
e1.set(Vec2(0, 0), Vec2(0.1, 0.1));

// ... or existing vectors
Vec2 pos1(0.3, 0.6);
Vec2 pos2(0.1, 0.2);
e1.set(pos1, pos2);

// set all x and y values as floats
e2.set(-0.4, 0.2, -0.7, 0.9;
\end{code}

\subsection{Methods}
An \eeClass{Edge2} provides several neat methods. The most obvious will be \eeFunc{draw()}:

\begin{code}
line1.draw(PURPLE    ); // draw a purple line
line2.draw(GREEN, 0.1); // draw a green line, with width 0.1
line3.draw(GREEN, RED); // draw a line that starts out green but fades to read
\end{code}

Other methods of interest are \eeFunc{center()}, \eeFunc{delta()}, \eeFunc{dir()} and \eeFunc{length()}. All of them can be found in $\Rightarrow$ Math $\Rightarrow$ Shapes $\Rightarrow$ Edge.

\subsection{Exercises}

The methods \eeFunc{Sin()} en \eeFunc{Cos()} allow you to retrieve the sine and cosine from any value. This is most fun when that value is the current time. The result over time is a value which moves smoothly between -1 and 1.

\begin{code}
float x = Sin(Time.curTime());
float y = Cos(Time.curTime());
\end{code}

\begin{enumerate}
\item The code above illustrates how to use the sine and cosine with the current time. Create an application which shows a line, starting from the middle of the screen towards the current value of sine and cosine for the x and y value of the ending.
\item Change the width of the line.
\item Make the line move at double speed.
\item Decrease the length of the line. 
\item \textit{(This will be a bit harder)} Try to create an analog watch.
\end{enumerate}

Take a look at Esenthel Engine $\Rightarrow$ Math $\Rightarrow$ Shapes $\Rightarrow$ Edge. Examine the method \eeFunc{lerp(float s)} vinden. This function needs an argument between zero and one, and return a position on the line.

\begin{enumerate}
\item Define an \eeClass{Edge2} and a \eeClass{float} with value zero.
\item In the init function, assign a start and end position to your line.
\item In the update function you increase the float with \eeClass{Time.d()}. When the float value is larger than 1, is should be assigned 0.
\item Draw the line in black with a width of 0.05. Construct a \eeClass{Vec2} with the result of \eeFunc{lerp()}. The argument of the method should be your float. Draw this point in red, also with a width of 0.05.
\end{enumerate}

Draw a triangle on the screen.

\section{Rect}
The last shape in the chapter is a rectangle: \eeClass{Rect}. You will end up using this shape quite a lot, because a rectangle happens to be the shape of most gui elements, like buttons, windows and images.

After the last exercise, you know that you need 3 positions to draw a triangle. Logic dictates a rectangle requires 4 positions, right? But there's one property of a rectangle that makes it a lot easier: all corners have an angle of 90\%. This means that when you pass the lower left corner and the upper right corner, there is enough information to find the two remaining corner positions.

\begin{figure}[h]
\centering
\includegraphics[width=0.8\linewidth]{../images/rectangle}
\caption[]{The corners of a rectangle}
\label{fig:rect}
\end{figure}

This means a rectangle can be created like this:

\begin{code}
Rect(Vec2(-0.2, -0.1), Vec2(0.2, 0.1)).draw(BLACK, false);
\end{code}

Zoals je ziet is er ook nu een draw functie, waarin het eerste argument de kleur is, en je daarna optioneel kan aangeven dat je de rechthoek niet wil vullen. Net zoals bij een edge heb je ook nu weer de mogelijkheid om de x- en y-co\"ordinaat afzonderlijk in te geven:

\begin{code}
Rect(-0.2, -0.1, 0.2, 0.1).draw(BLACK);
\end{code}

\subsection{A moving Rectangle}
If you want to draw an image on the screen, it will require a rectangle. As such, a rectangle will be the basis of almost every element on your screen. It will often come in handy to have a special class for movable rectangles. As an example, we'll create a rectangle which can be moved with the arrow keys.

Rectangles have one big disadvantage when you try to move them. Instead of a central position, both corners have to be moved. This is why we often use a \eeClass{Vec2} to remember the center position. As an extra, this class will remember its color.


\begin{code}
class movableRect {
  	Vec2 pos;
	Color color;
}
\end{code}


Deze class kunnen we uitbreiden met een create functie, een update functie en een draw functie. Via de create functie geef je de kleur en de startpositie door. De startpositie heeft een standaard waarde, dus wanneer je later de create functie gebruikt zonder een positie, dan staat de rechthoek in het midden van het scherm.

\begin{code}
void create(C Color & color, C Vec2 & pos = 0) {
  T.color = color;
	T.pos = pos;
}
\end{code}

\begin{note}
In deze functie staan nog enkele nieuwigheden. De ampersand geeft aan dat we color en pos als referentie doorgeven. De \eeFunc{C} betekent dat we de waarde niet zullen aanpassen. Je leert hierover meer in een van de volgende hoofdstukken. De letter \eeFunc{T} lost een praktisch probleem op. Zowel het functie-argument als de Color in de class zelf hebben als naam `color'. De compiler kan daarom niet weten wanneer je welke variabele bedoelt. Je zou ze beiden een verschillende naam kunnen geven, maar dat maakt de code moeilijker leesbaar. De elegante oplossing bestaat er in \eeFunc{T.} voor de variabele van de class toe te voegen. De T staat voor `this' en daarmee bedoelen we `deze class'. De versie zonder \eeFunc{T.} is bijgevolg de variabele die we als functieargument doorgeven.
\end{note}

In de update functie kunnen we de positie wijzigen via de pijltjestoetsen. Deze code heb je reeds gezien in hoofdstuk \ref{chapter:keyboardInteractie}.

\begin{code}
void update() {
	if(Kb.b(KB_LEFT )) pos.x -= Time.d();
	if(Kb.b(KB_RIGHT)) pos.x += Time.d();
	if(Kb.b(KB_UP   )) pos.y += Time.d();
	if(Kb.b(KB_DOWN )) pos.y -= Time.d();
}
\end{code}


Tot slot is er een draw functie nodig. Het is pas op deze plaats dat we, tijdelijk, een rechthoek maken. In dit geval is dat het meest praktisch, maar dat hoeft niet altijd zo te zijn. Het zou bijvoorbeeld kunnen dat je in de update functie wil controleren of de rechthoek iets raakt. In zo'n geval zou je waarschijnlijk ook een \eeClass{Rect} aan je class toevoegen.

\begin{code}
void draw() {
  Rect(pos - Vec2(0.2, 0.1), pos + Vec2(0.2, 0.1)).draw(color);
}
\end{code}

Zoals je ziet gebruiken we de positie `pos' bij het maken van de rechthoek. Die is namelijk het middelpunt. Aangezien we de hoeken linksonder en rechtsboven nodig hebben bij het maken van een rechthoek, kunnen we eenvoudig een waarde respectievelijk aftrekken en optellen. We zouden deze class nog wat meer flexibel kunnen maken door een `size' te onthouden.

Omdat dit de eerste maal is dat we een volledige class uitwerken in esenthel krijg je ze hieronder nog eens helemaal te zien. Let ook op de keywords public en private. Die zijn later belangrijk als je je code overzichtelijk wil houden.

\begin{code}
class movableRect {
private:
  Vec2 pos, size;
	Color color;
	
public:
  void create(C Color & color, C Vec2 & pos = 0, C Vec2 & size = Vec2(0.1, 0.1)) {
	  T.color = color;
		T.pos   = pos  ;
		T.size  = size ;
	}
	
	void update() {
		if(Kb.b(KB_LEFT )) pos.x -= Time.d();
		if(Kb.b(KB_RIGHT)) pos.x += Time.d();
		if(Kb.b(KB_UP   )) pos.y += Time.d();
		if(Kb.b(KB_DOWN )) pos.y -= Time.d();
	}
	
	void draw() {
		Rect(pos - size, pos + size).draw(color);
	}
}
\end{code}
	
\subsection{Oefeningen}

\begin{enumerate}
\item Maak in een programma de volgende afbeelding na:

\begin{figure}[h]
\centering
\includegraphics[width=0.4\linewidth]{../images/nested_rectangles}
\caption[]{Rechthoeken}
\label{fig:nested_rect}
\end{figure}

\item Voeg een roze vierkant toe van de class \eeClass{movableRect} die we hierboven voorstelden.
\item \textbf{(Uitbreiding)} Zorg dat het vierkant niet buiten het scherm kan bewegen.
\end{enumerate}

\section{Cuts}

Een functie die je vaak gebruikt in verband met shapes is \eeFunc{Cuts}. Er bestaan heel veel varianten op deze functie, maar de betekenis is steeds dezelfde: raken twee objecten mekaar of niet? Zo kan je controleren of een punt en een cirkel mekaar raken, of een cirkel en een rechthoek, twee rechthoeken, een driehoek en een lijn, enzovoort.

Om te controleren o een punt een cirkel raakt, gebruik je bijvoorbeeld de volgende code:

\begin{code}
Vec2 pos(0.1, 0.1);
Circle area(0.3, Vec2(0));

if(Cuts(pos, area)) area.draw(RED);
\end{code}

Natuurlijk is het in dit voorbeeld al duidelijk dat het punt de cirkel steeds zal raken. Maar ook de positie van je muiswijzer is een punt. Je zou dus het volgende kunnen schrijven:

\begin{code}
Circle area(0.1, Vec2(0));

if(Cuts(Ms.pos(), area)) area.draw(RED);
else area.draw(BLACK);
\end{code}

De code hierboven ligt aan de basis van vele interactiemogelijkheden. Dikwijls gebeurt dit in combinatie met andere controles. Probeer zelf eens te bedenken wat de volgende code doet:

\begin{code}
Rect button(Vec2(-0.2, -0.1), Vec2(0.2, 0.1));
bool hover = false;

// in een update functie:
if (Cuts(Ms.pos(), button)) {
  hover = true;
	if(Ms.bp(0)) exit();
} else hover = false;

// in een draw functie
if(hover) {
  button.draw(Color(0, 255, 0));
} else {
  button.draw(Color(0, 155, 0));
}
\end{code}

\subsection{Oefeningen}
\begin{enumerate}
\item Maak een programma met 3 cirkels (onder mekaar) waarvan je enkel de rand toont, tenzij de muiscursor zich in een van de cirkels bevindt. In dat geval teken je die cirkel gevuld.
\item Pas het vorige programma aan. Zorg er voor dat een cirkels langzaam naar rechts beweegt wanneer die zich onder de muiscursor bevindt.
\item Maak een integer `score'. Wanneer een cirkel de rechterkant van het scherm raakt, verhoog je de score met \'e\'en en plaats je de cirkel terug links.
\end{enumerate}

\textbf{(Uitbreiding) }Maak een eigen class die zich als een button gedraagt. Je kan vertrekken van het voorbeeld hierboven en een hover effect implementeren. (Extra uitdaging: de kleur kan ook geleidelijk veranderen.) Een button zal ook een tekst moeten tonen, dus je voorziet een create functie die de positie en die tekst instelt. Een meer algemene functie om te bepalen wat de button doet wanneer je er op klikt is nog niet voor nu, maar je kan natuurlijk altijd proberen.






\chapter{Images and Sound}
\section{Images}
A modern 2D game will almost always contain images. Whatever happens on the screen, it mostly comes down to showing and manipulating images. And because every image is a rectangle, you will use the \eeClass{Rect} class to show them on the screen.

\begin{code}
Images(=== drop image here ===).draw(Rect(-0.1, -0.1, 0.1, 0.1));
\end{code}

You can use \eeClass{Images()} to refer to any image in your project. The image in question can be dropped as the function argument, between the parentheses. Once that is done, you use the function \eeFunc{draw} with a \eeClass{Rect} argument to show the image on the screen. It is very easy to make your image move this way. The only thing your application has to remember is the current position. The actual rectangle can be derived from that point.

\begin{code}
Vec2 ship(0, -0.8);

// during update:
if(Kb.b(KB_LEFT )) ship.x -= Time.d();
if(Kb.b(KB_RIGHT)) ship.x += Time.d();

// during draw:
Images(=== spaceship ===).draw(Rect(ship - 0.1,  ship + 0.1));
\end{code}

\begin{note}
When you're looking for images like the one in this example, just use google images. Mostly you will want images with a transparent background. This will be an image in GIF or PNG format. But Esenthel doesn't support GIF, so PNG is your target of choice. The search tools on Google Images allow you to search specifically for transparent images. Can't find what you're looking for? Try adding the term `icon' or `sprites' to your query.

Just remember this is great while you're experimenting. But once you're working on a real game you should not use images which aren't yours, unless you really verified their license permits the use you intend.
\end{note}

To add realism to your game it is a good idea to use variations on an image. In the next example, two alternate versions of the same image are used during movement.

\begin{code}
if(Kb.b(KB_LEFT))
{
	Images(=== spaceship ===      ).draw(Rect(ship - 0.1,  ship + 0.1));
} else if(Kb.b(KB_RIGHT))
{
	Images(=== spaceship_left === ).draw(Rect(ship - 0.1,  ship + 0.1));
} else
{
	Images(=== spaceship_right ===).draw(Rect(ship - 0.1,  ship + 0.1));
}
\end{code}

\begin{exercise}
Find 3 very sad images and create an application with a moving image. Make sure the images are switched one way or another.
\end{exercise}

Another way to add some dimension is by varying the image over time. You actually create a little animation by rotating through a list of images all the time. The following example presents a player class with three variations for every direction.

\begin{code}
class player
{
private: 
   Vec2 pos;
   float timer = 0.4; 
   DIR_ENUM dir = DIRE_DOWN;
   
public:    
   void update()
   {
      // adjust direction
      if(Kb.bp(KB_UP   )) dir = DIRE_UP   ;
      if(Kb.bp(KB_DOWN )) dir = DIRE_DOWN ;
      if(Kb.bp(KB_LEFT )) dir = DIRE_LEFT ;
      if(Kb.bp(KB_RIGHT)) dir = DIRE_RIGHT;
      
      // update position
      switch(dir)
      {
         case DIRE_UP   : pos.y += Time.d() * 0.5; break;
         case DIRE_DOWN : pos.y -= Time.d() * 0.5; break;
         case DIRE_LEFT : pos.x -= Time.d() * 0.5; break;
         case DIRE_RIGHT: pos.x += Time.d() * 0.5; break;
      }
      
      // animation timer
      timer -= Time.d();
      if(timer < 0) timer = 0.4;
   }
   
   void draw()
   {
      // pointer to an image
      ImagePtr current;
      
      // evaluate direction
      switch(dir)
      {
         case DIRE_UP:
         {
            // pick an image according to time (changes between 1 - 2 - 3 - 2)
            if      (timer > 0.3) current = Images(=== back1 ===);
            else if (timer > 0.2) current = Images(=== back2 ===);
            else if (timer > 0.1) current = Images(=== back3 ===);
            else                  current = Images(=== back2 ===);
            break;
         }
         
         case DIRE_DOWN:
         {
            if      (timer > 0.3) current = Images(=== front1 ===);
            else if (timer > 0.2) current = Images(=== front2 ===);
            else if (timer > 0.1) current = Images(=== front3 ===);
            else                  current = Images(=== front2 ===);
            break;
         }
         
         case DIRE_LEFT:
         {
            if      (timer > 0.3) current = Images(=== left1 ===);
            else if (timer > 0.2) current = Images(=== left2 ===);
            else if (timer > 0.1) current = Images(=== left3 ===);
            else                  current = Images(=== left2 ===);
            break;
         }
         
         case DIRE_RIGHT:
         {
            if      (timer > 0.3) current = Images(=== right1 ===);
            else if (timer > 0.2) current = Images(=== right2 ===);
            else if (timer > 0.1) current = Images(=== right3 ===);
            else                  current = Images(=== right2 ===);
            break;
         }
      }
      
      // show the current image on the screen
      current->draw(Rect(pos - 0.05, pos + 0.05));
   }   
}
\end{code}

\begin{exercise}
The example above can be found in the course template. Create an object of the player class and use it in an application to see the result of this code.
\end{exercise}

\begin{note}
On line 39 there's an object `current' of the class \eeClass{ImagePtr}. this class is a `pointer' to an \eeClass{Image}. The code below that line will make the pointer `point' to the image we want to show next. At the bottom, \verb|current->draw()| is used to draw that image. Note the arrow instead of the dot. This is a sign that current is not a real image object, but just points to one. (Don't worry if you find this hard to grasp. You will learn more about pointers in another chapter.)
\end{note}

\begin{exercise}
The \eeClass{Image} class in Esenthel contains a whole bunch of functions. Most of them you will not need any time soon, but it is good to remember that whatever you want to do with your image, there's a good chance there is a function which has you covered.

For now, experiment a bit with the functions below to learn about their intent.

\begin{itemize}
\item \eeFunc{draw} has a version which allows you to pass some colors as an argument. Try this out. (Hint: the second color will very often be \eeFunc{TRANSPARENT})
\item Draw an image using \eeFunc{drawFit}. How does this differ from \eeFunc{draw}?
\item Draw an image using \eeFunc{drawRotate}. Try rotating the image with the arrow keys.
\item Draw an image using \eeFunc{drawFS}
\item (a bit harder) Load an image, apply a blur and export as PNG. Can you do it?
\end{itemize}
\end{exercise}

\section{Sound}
To make your game a bit more attractive you will want to add sound. Generally speaking, there are two groups: music and effects (FX). Music will mostly be played in the background while FX is linked to certain actions like pressing a button or dying horribly.

\subsection{Music}
A soundtrack can be played with the class \eeClass{Sound}:

\begin{code}
Sound soundtrack;

void InitPre()
{
   EE_INIT();
}

bool Init()
{
   soundtrack.create(=== drop your audio file here ===);
   return true;
}

void Shut() {}

bool Update()
{
   if(Kb.bp(KB_ESC)) return false;
   
   if(Kb.bp(KB_SPACE))
   {
      if(soundtrack.playing())
      {
         soundtrack.pause();
      } else
      {
         soundtrack.play();
      }
   }
   return true;
}
\end{code}

There's a few things to remember, though:

\begin{itemize}
\item Before you can use a sound, it must be loaded from disk. This is done with the \eeFunc{create()} function. It needs at least one argument: the audio file. Like with images, you can simply drag the file from your resources on to your function. You will want to do this inside of the \eeFunc{Init()} function, because you don't want to load your file from disk at every update.
\item \eeFunc{play()} will cause the sound to start playing.
\item \eeFunc{pause()} will pause the sound. Who would have guessed, right? When you use play after pause, the sound will continue right where it left off. 
\item Instead of \eeFunc{pause()} you can also use \eeFunc{stop()}. Now when you start playing again, the sound will start from the beginning.
\end{itemize}

The \eeFunc{create()} function also has a few optional arguments. Here's an example with all of them:

\begin{code}
soundtrack.create(=== audio file ===, true, 0.8, VOLUME_MUSIC);
\end{code}

But what do they mean, little grasshopper?

\begin{enumerate}
\item The first argument is known. That's the audio file.
\item the second argument is the loop value. It can be true or false and is used to indicate if you'd like the sound to `loop'. (Which mean it will start from the top when it is finished.) The default value is false.
\item Next comes the volume. Volume scales from zero to one, with a default of 1.
\item The last argument is a `channel'. Esenthel has several channels for playing audio. If you don't use this argument, the sound will use the channel `VOLUME\_FX'. It is generally a good idea to use several channels for different types of sounds, because the volume of a channel can be changed. It makes it easy to implement volume changes for music, fx or voices.
\end{enumerate}

\begin{exercise}
Create an application which loads a soundtrack. Draw a green, an orange and a red circle on the screen. The track should start playing when you click the green circle, pause when you click the orange circle and stop when you click the red one.
\end{exercise}

\begin{exercise}
\textbf{Extra:} Search the header file of the sound class for a method to retrieve the current playing position within a sound file. Draw this position on the screen. 
\end{exercise}

\begin{exercise}
\textbf{Extra:} This will be a bit harder. Use the functions \eeFunc{fadeInFromSilence()} and \eeFunc{fadeOut()} to apply a fade of 3 seconds instead of an immediate start and stop.
\end{exercise}

\subsection{Playlists (Extra)}
To play music, you can also use playlists. This will bring more variation to your soundtrack, and also allows you to switch between playlists when the mood of the game changes. Just examine the code below to see how it works:

\begin{code}
// defined play lists
Playlist Battle , // battle playlist 
         Explore, // exploring playlist
         Calm   ; // calm playlist

void InitPre()
{
   EE_INIT();
}

bool Init()
{
   if(!Battle.songs()) // create 'Battle' playlist if not yet created
   {
      Battle += (=== drop action music ===); // add "battle0" 
      Battle += (=== same here ===); // add "battle1" 
   }
   if(!Explore.songs()) // create 'Explore' playlist if not yet created
   {
      Explore+= (=== drop tranquil music ===); // add "explore" track 
   }
   if(!Calm.songs())  // create 'Calm' playlist if not yet created
   {
      Calm   += (=== a very relaxed soundtrack ===); // add "calm" 
   }
   return true;
}

void Shut()
{
}

bool Update()
{
   if(Kb.bp(KB_ESC))return false;
   if(Kb.c('1'))Music.play(Battle );
   if(Kb.c('2'))Music.play(Explore);
   if(Kb.c('3'))Music.play(Calm   );
   if(Kb.c('4'))Music.play(null   );
   return true;
}

void Draw()
{
   D.clear(TURQ);

   if(Music.playlist()) // if any playlist playing
   {
      D.text(0, 0, S+"time " +Music.time()+" / "+Music.length()+" length");
   }else
   {
      D.text(0, 0, "No playlist playing");
   }
   D.text(0, -0.2, "Press 1-battle, 2-explore, 3-calm, 4-none");
}
\end{code}

\subsection{FX}
Short sounds can also be played with the method \eeFunc{SoundPlay()}. This will play the sound directly, without requiring you to create an object of the class \eeClass{Sound}. Because there is no object you won't have any control over the sound after you started it. Therefore this technique will mostly be used for very short effects, such as footsteps. 

\begin{exercise}
In the template project you will find a few sounds in the folder `sound'. Create an application that plays back a `blip' every time you push the arrow-down key. Use the `rotate' sound for arrow-left and arrow-right. And last, play back the `down' sample when you press the space bar.

\textbf{Extra:} Add the soundtrack again, but this time control the volume of the track with the mouse wheel.
\end{exercise}
 

\partAbstract{
In dit deel leer je over meer algemene concepten zoals random waarden en containers. Je herhaalt ook concepten zoals references, enumeraties, functies en classes terwijl je leert hoe die in Esenthel dikwijls gebruikt worden. Om dit deel af te sluiten leer je over application states. Vooral omdat er geen ander deel is waarin dat hoofdstuk past.
}
\part{Basics}
\chapter{Random}

\section{Gehele getallen}
Een game blijft zelden boeiend als die compleet voorspelbaar is. Om die voorspelbaarheid tegen te gaan maken ontwikkelaars gebruik van willekeurige waarden via de \eeClass{Random} class. Die class laat je toe om variatie in te bouwen. De functie \eeFunc{Random()} geeft je een willekeurig getal tussen 0 en 4.294.967.295. Je kan dat testen via het volgende voorbeeld:

\begin{code}
uint number = 0;

bool Update() {
  if(Kb.bp(KB_SPACE)) number = Random();
	return true;
}

void Draw() {
	D.clear(BLACK);
	D.text(0, 0, S + number);
}
\end{code}

In praktijk zal je zelden zo'n grote getallen willen. Je kan de functie \eeClass{Random()} daarom ook gebruiken met een of meerdere argumenten. E\'en argument zorgt er voor dat je een getal krijgt in de range 0 tot het argument - 1. Dus \eeClass{Random(5)} geeft je de waarde 0, 1, 2, 3 of 4 als resultaat. Let op: dat zijn 5 verschillende waarden, maar het getal 5 zal niet voorkomen!

Je kan ook twee argumenten gebruiken. \eeClass{Random(-2, 4)} heeft mogelijk de volgende waarden: -2, -1, 0, 1, 2, 3 of 4. In deze versie zijn de waarden dus wel inclusief.

\begin{exercise}
Probeer de bovenstaande code uit. Test ook de versies van de Random functie met \'e\'en of twee argumenten.
\end{exercise}

\begin{note}
Als je een random kleur wil, dan kan je de RGB waarden van Color een willekeurige waarde tussen 0 en 255 geven. Dat kan zo:

\begin{code}
Color myColor;
myColor.set(Random(255), Random(255), Random(255));
\end{code}
\end{note}

\section{Random Float}
De bovenstaande code geeft je een willekeurig geheel getal. Dikwijls zal je echter een floating point waarde willen. Daarvoor kan je een andere functie gebruiken: \eeFunc{RandomF()}. Deze functie geeft je een waarde tussen 0 en 1. Ook hier is het mogelijk om argumenten te gebruiken. \eeFunc{RandomF(3)} geeft je een waarde tussen 0 en 3. \eeFunc{RandomF(-1.3, 2.5)} levert een waarde tussen -1.3 en 2.5.

Dikwijls zal je deze functies gebruiken om een object op een willekeurige positie te tonen. Dat kan zo:

\begin{code}
Circle c;

c.pos.x = RandomF(-D.w(), D.w());
c.pos.y = RandomF(-D.h(), D.h());
\end{code}

Zoals je ziet zetten we niet letterlijk getallen in de functie RandomF. Dat kan wel, maar in dit geval is dat niet aangewezen. De hoogte en de breedte van het scherm kunnen namelijk verschillen van toestel tot toestel. Aangezien het middelpunt van het scherm gelijk is aan 0, is de negatieve breedte dus de linkerkant (minimum waarde voor x). De negatieve hoogte is de onderkant (minimum waarde voor y).

\begin{exercise}
\begin{enumerate}
\item Maak een programma dat een cirkel op het scherm toont. Telkens je op de spatiebalk drukt, geef je de cirkel een nieuwe positie.
\item Maak een programma dat een klein vierkant op het scherm toont. Elke seconde geef je het vierkant een nieuwe positie.
\item Dit is een uitbreiding op het vorige programma. Voor zie een int `score' die gelijk is aan nul. Je toont de score ergens op het scherm. Wanneer je de linkermuisknop indrukt, controleer je of de muiswijzer binnen het vierkant zit. Als dat zo is, verhoog je de score met 1.
\item Uitbreiding: zorg dat het vierkant steeds sneller van positie wisselt.
\end{enumerate}  
\end{exercise}




\chapter{Containers}

So far, you needed to define all global objects at the head of your application file. This is no problem for a little exercise, but when your project grows in size, this becomes a problem. You also have to know upfront how many objects you need. Even for a little game like asteroids, it is impossible to know how many rocks there will be on the screen at all times.

When you need several objects of the same type, you can use a container. When you declare a container for a certain object type, you can add objects to this container during the course of the application. An easy to use container is \eeClass{Memx}. The declaration of a container requires that you provide the type of objects it will contain. When you need a container for floats, you would declare it as a \eeClass{Memx<float>}. A container for rectangles would be a \eeClass{Memx<Rect>}. Look at this code for an example of a container with circles:

\begin{code}
// Declare a container for circles
Memx<Circle> circles;

void InitPre()
{
   EE_INIT();
}

bool Init()
{
    // add 10 circles to this container  
	for(int i = 0; i < 10; i++)
    {
	    // The method New() adds a new circle to the container. 
	    // At the same time the Circle method set() is used to 
	    // assign a radius and a position.
        circles.New().set(0.1, RandomF(-D.w(), D.w()), RandomF(-D.h(), D.h()));
   }
   return true;
}

void Shut() {}

bool Update()
{
   if(Kb.bp(KB_ESC)) return false;  
   return true;
}

void Draw()
{
   D.clear(BLACK);
   
   // Go over all circles in the container and
   // draw them on the screen.
   for(int i = 0; i < circles.elms(); i++)
   {
      circles[i].draw(RED);
   }
}
\end{code}

\begin{exercise}
\begin{enumerate}
\item What would happen if, by mistake, you place the code to generate circles in Update instead of Init?
\item Put this code back in Init, but add code to the Update function: every time you press the space bar, an extra circle should be added to the container.
\item Show an image on the screen instead of a circle. \textit{(Too hard? Start with a rectangle!)}
\end{enumerate}
\end{exercise}

\section{New()}
The method \eeFunc{New()} creates a new element at the end of the container. At the same time, it returns a reference to this new element, which is why can use the \eeClass{set()} method of circle in the example above. 

But suppose you need to use two methods of the newly created object? You could try something like this:

\begin{code}
for(int i = 0; i < 10; i++)
{
	circles.New().set(0.1, RandomF(-D.w(), D.w()), RandomF(-D.h(), D.h()));
	circles.New().extend(-0.05);
}
\end{code}

\ldots but it won't work. Instead you are creating two new circles at every iteration. The method \eeFunc{set} is called on the first circle, the method \eeFunc{extend} at the second. The solution is simple: Pass the result of \eeFunc{New()} to a temporary variable. The type of this variable must be a reference to a circle. (If you don't know what a reference is, don't worry. We'll talk about it later. For now, remember that you need to put an ampersand (\&) between the type and the name.

\begin{code}
for(int i = 0; i < 10; i++)
{
	Circle & c = circles.New();
	c.set(0.1, RandomF(-D.w(), D.w()), RandomF(-D.h(), D.h()));
	c.extend(-0.05);
}
\end{code}

\section{Using objects}
Very often, you need to iterate over all elements in a container. For example when you draw them all on the screen. It would be very annoying if you had to remember somehow exactly how many elements a container contains. Fortunately, you do not have to. Containers provide a method \eeFunc{elms()} which returns the current number of elements. And to access individual elements you can use square brackets, just like with primitive C arrays.

\begin{code}
for(int i = 0; i < circles.elms(); i++)
{
  circles[i].draw(RED);
}
\end{code}

Because you will need an iteration like this very, very often, Esenthel provides a `shortcut'. A macro \eeFunc{REPA} exists to replace the whole for-loop declaration with one instruction:

\begin{code}
REPA(circles)
{
  circles[i].draw(RED);
}
\end{code}

Remember this as `repeat all'. \textit{(Or don't remember it at all. Plain for-loops will always work just as well.)} And you can do more with this than just draw every element on the screen. Take a look at the next example and try to figure out what it does.

\begin{code}
REPA(circles)
{
	circles[i].pos.y += Time.d();
	if(circles[i].pos.y > D.h()) {
	  circles[i].pos.y -= (2*D.h() + RandomF(1));
	}
}
\end{code}

\begin{exercise}
\begin{enumerate}
\item Test the code above in an application. What function would you place this code in?
\item Add a function to add an extra circle every time you hit the space bar.
\item Instead of a fixed radius, use a random value between 0.01 and 0.1.
\item Draw only the perimeter of the circle, in white, on a blue background.
\item If there are any people nearby, shout out loud what this looks like.
\end{enumerate}
\end{exercise}

\section{Adding Objects}
You will add objects to a container quite a lot. This might happen in the Init function as well as the Update function. Below are a few examples to get you started, but there are a lot of different ways to add objects. It is up to you to figure out what is the best approach in your application.

\subsection{During Init}

Ten circles on random positions:

\begin{code}
for(int i = 0; i < 10; i++)
{
	Circle & c = circles.New();
	c.set(0.1, RandomF(-D.w(), D.w()), RandomF(-D.h(), D.h()));
}
\end{code}

Circles from the left to the right side of the screen:

\begin{code}
for(float i = -D.w(); i < D.w(); i += 0.2) {
  circles.New().set(0.1, i, 0);
}
\end{code}

Squares placed evenly over the screen:
\begin{code}
for(float i = -D.w(); i < D.w(); i += 0.2)
{
	for(float j = -D.h();  j < D.h();  j += 0.2)
	{
		 rects.New().set(i - 0.05, j - 0.05, i + 0.05, j + 0.05);
	}     
}
\end{code}

\subsection{During Update}

Respond to keyboard input:
\begin{code}
if(Kb.bp(KB_SPACE)) {
  circles.New().set(RandomF(0.05, 0.2), RandomF(-D.w(), D.w()), RandomF(-D.h(), D.h()));
}
\end{code}

Use the mouse position:
\begin{code}
if(Ms.bp(0)) {
  circles.New().set(0.05, Ms.pos());
}
\end{code}

With a timer:
\begin{code}
Flt timer = 3; // put this line on to of the file. Next lines belong in Update()

if(timer > 0) timer -= Time.d();
else {
  timer = 3;
	circles.New().set(RandomF(0.05, 0.2), RandomF(-D.w(), D.w()), RandomF(-D.h(), D.h()));
}
\end{code}

\begin{exercise}
Test all of the examples above and make sure you understand every one of them. Always add code to display all elements on the screen.
\end{exercise}

\section{Removing Objects}
Of course you also want to remove objects from a container. Which is not that hard:

\begin{code}
Memc<Vec2> dots;

// ... add a lot of dots

dots.remove(0); // remove the first dot
\end{code}

With the method \eeFunc{remove} and the index of the element as an argument, you delete an object in a container. Be careful though. Very often you will want to remove an element while iterating over a container. It is a common beginner mistake to alter an object after you've deleted it:

\begin{code}
for(int i = 0; i < dots.elms(); i++) {
    if(dots[i].y < -D.h()) {
        dots.remove(i);
	}
	dots[i].y -= Time.d();
}
\end{code}

In the example above, all dots are moved down at every update. When a dot arrives at the bottom of the screen, it will be removed from the container. After removing a dot, it is not the current dot that is moved down, but the next one in the container. This is not a big problem, unless this was actually the last dot in the container. In which case you try to move down an object past the end of the container. The result will be a program crash, your computer might explode and probably a kitten will die somewhere.

To prevent this from happening it is a good rule to put the remove method as the last statement in the loop:

\begin{code}
for(int i = 0; i < dots.elms(); i++) {
	dots[i].y -= Time.d();
  if(dots[i].y < -D.h()) dots.remove(i);
}
\end{code}

Things start to be a bit more complicated when you combine more than one container. In the next example we have container for dots and a container for circles. The code tries to verify if a dot hits a circle. If this is the case, both the circle and the dot must be removed from their container. To do this, we have to check every dot against every circle.

\begin{code}
for(int i = 0; i < dots.elms(); i++) {
	for(int j = 0; j < circles.elms(); j++) {
	    if(Cuts(dots[i], circles[j])) {
		    dots.remove(i);
		    circles.remove(j);
		    // At this point there is one less dot in the container, but next 
		    // circles will still be compared against the current dot. 
		    // If we are at the last dot, i will no longer be valid. 
		    // To prevent a crash, we add a break statement to go back to 
		    //the outer for loop:
			break;
		}		
	}
}
\end{code}

And if you'd like to clear all container elements at once:

\begin{code}
dots.clear();
\end{code}

\section{A little Game}

\begin{enumerate}
\item Create a triangle at the bottom of the screen. This triangle can be moved back and forth with the arrow keys.
\item Add a container for the class \eeClass{Vec2}. Every time you press the space bar, you add an element on the location of the triangle.
\item Increase the y value of every container element in the Update function.(Use \eeClass{Time.d()}!) If an element reaches the top of the screen, remove it from the container.
\item Draw all elements on the screen in the \eeFunc{Draw()} function.
\item Create a second container for circles. Every second a circle must be added somewhere at the top of the screen.
\item Move all circles down in the \eeFunc{Update()} function.
\item Show all circles in the \eeFunc{Draw()} function.
\item When a circle hits a \eeClass{Vec2} from the other container, both must be removed.
\item When a circle hits the triangle, `Game Over' must be shown on the screen.
\end{enumerate}

You could go even further with this game. Don't create new circles after the game is finished, and disable movement and shooting. Circles might move faster the longer you play, a score can be shown or you might give the player more than one life.

And instead of triangles and circles, images might be used. Have fun!


\chapter{Classes: the basics}

Gegevens die samen horen, hoor je te groeperen. Zo heeft \eeClass{Vec2} (eigenlijk een class!) twee float variabelen x en y om een positie te onthouden. Zonder het bestaan van \eeClass{Vec2} zou je voor elke positie twee afzonderlijke float variabelen moeten maken. Nu maak je er slechts 1: een \eeClass{Vec2}.

De ontwerper van de Esenthel engine voorzag dat een class voor een positie, met een x en een y waarde, heel veel gebruikt zou worden. Dus maakte hij die alvast zelf. De meest eenvoudige versie zou er zo kunnen uitzien:

\begin{code}
class Vec2 {
	float x;
	float y;
}  
\end{code}

Indien deze code bestaat, kan je die overal in je programma gebruiken:

\begin{code}
Vec2 pos;
Vec2 pos2 = pos; // wijst de waarden van pos toe aan pos2.
pos.x = 3;       // past de x waarde van pos aan.
pos.y = pos2.x;  // geeft de x waarde van pos2 aan de y waarde van pos
\end{code}

Hetzelfde kan je bereiken met je eigen classes. Je hoort classes te maken voor zo ongeveer elk aspect van je code. 

\begin{note}
Wanneer twee of meer objecten of variabelen samen horen, dan zet je ze samen in één class.
\end{note}

Merk op dat er een verschil is tussen een class en een object. In het voorbeeld hierboven is \eeClass{Vec2} een class, maar pos en pos2 zijn objecten van de class Vec2. Je kan \eeClass{Vec2} dus niet gebruiken als een object, maar je kan er wel objecten mee maken:

\begin{code}
Vec2.x = 3; // dit is fout!
Vec2 pos;
pos.x = 3; // dit is correct.
\end{code}

\section{Een eigen class maken}
Stel je voor dat je een bewegende cirkel wil in je programma. Om de cirkel tegen een vaste snelheid laten te bewegen heb je een variabele speed nodig. Die hoort duidelijk bij je cirkel, maar een cirkel heeft zelf geen variabele speed omdat cirkel niet altijd bedoeld zijn om te laten bewegen. Je maakt dus bijvoorbeeld de volgende class:

\begin{code}
class movingCircle {
  Circle c;
  Vec2 speed;
}

movingCirle mc;

// in de functie init
mc.c.set(0.1, Vec2(-0.4, -0.3));
mc.speed = Vec2(0.3, 0.5);

// in de functie update
mc.c.pos += mc.speed * Time.d();
\end{code}

Eenmaal je de class \eeClass{movingCirle} hebt, kan je zoveel cirkels maken als je wil, die allemaal hun eigen snelheid onthouden. (Verder in de cursus zie je hoe dit nog eenvoudiger kan.)


\begin{exercise}
\begin{enumerate}
  \item Maak een container voor de class movingCircle. Telkens je op de spatiebalk drukt, voeg je hier een object aan toe, op een willekeurige positie op het scherm. Elke cirkel geef je ook een willekeurige snelheid.
	\item Zorg dat alle cirkels bewegen. Wanneer een cirkel buiten het scherm komt, zet je hem terug in het midden.
	\item Breidt de class \eeClass{movingCirle} uit zodat die ook een kleur kan onthouden. Elk object geef je een willekeurige kleur, die je gebruikt wanneer je het op het scherm zet.
	\end{enumerate}
\end{exercise}

\begin{note}
In het volgende hoofdstuk werk je verder aan deze oefening. Sla ze dus op!
\end{note}

\chapter{Functies}

Functies zijn instructies die je kan toevoegen aan een eigen class. \textit{(In principe kan je ook functies buiten een class gebruiken, maar in regel probeer je dat best te vermijden.)} Met functies kan je duidelijke opdrachten maken die betrekking hebben op je class. Het is eigenlijk de bedoeling dat je de variabelen in je class nooit rechtstreeks gebruikt buiten de class. Als je ze nodig hebt, of wanneer je ze wil aanpassen, dan schrijf je daar een functie voor.

\section{Functies zonder argumenten of resultaat}

De meest eenvoudige functie doet steeds hetzelfde. Als we even terugdenken aan de class movingCircle, dan zou die misschien een functie kunnen gebruiken die de cirkel terug in het midden van je scherm zet:

\begin{code}
class movingCircle {
  Circle c;
  Vec2 speed;
  
  void reset() {
    c.pos = Vec2(0);
  }
}

// ergens in je programma
movingCircle mc;
mc.reset();

\end{code}

De functie reset in het vorige voorbeeld zet de cirkel terug in het midden. De functie bestaat uit de volgende delen:

\begin{description}
\item[void]Een aanduiding dat de functie geen resultaat heeft. (Daarover zodadelijk meer.)
\item[reset]De naam van de functie. Deze mag je zelf kiezen.
\item[()]Tussen deze haakjes kunnen argumenten staan. Deze eenvoudige functie heeft geen argumenten, dus hier staat niets.
\end{description}

\begin{exercise}
Voeg aan de class die je voor de vorige oefening maakte een functie \eeFunc{reset()} toe, die de cirkel in het van het scherm plaatst. Vervang in de Update functie van je programma de code om een cirkel in het midden te plaatsen door deze functie.

Voeg in dezelfde class ook een functie \eeFunc{draw()} toe. Deze functie kan je cirkel dadelijk in de juiste kleur tonen. In je programma vervang je weer de code om de cirkel te tonen door deze functie.
\end{exercise}

\section{Functies met een argument}
Je kan een functie ook een argument geven. Dit is handig omdat je dat argument kan gebruiken om een waarde door te geven aan een functie. De functie reset in het vorige voorbeeld heeft geen waarde nodig. We weten dat we met reset de cirkel in het midden zetten. Wil je hem op een andere plaats zetten, dan zou je wel moeten zeggen welke plaats dat dan is. Dat zou zo kunnen:

\begin{code}
class movingCircle {
  Circle c;
  Vec2 speed;
  
  void setPos(Vec2 pos) {
    c.pos = pos;
  }
}

// ergens in je programma
movingCircle mc;
Vec2 p(0.1, 0.4);
mc.setPos(p);

\end{code}

Nu geef je tussen de haakjes mee dat de functie gebruikt moet worden met een \eeClass{Vec2} als argument. De \eeClass{Vec2} die je meegeeft bepaalt dan de nieuwe positie van de cirkel. Je ziet ook dat de \eeClass{Vec2} `p' die we in het programma gebruiken een andere naam heeft dan \eeClass{Vec2} pos die we in de functie gebruiken. Je geeft enkel de waarde van de functie door en niet de naam. 

\begin{exercise}
Voeg een functie \eeFunc{setPos} zoals hierboven toe aan je oefening. In het programma zorg je dat, wanneer je op F1 drukt, alle cirkels een nieuwe random positie krijgen.
\end{exercise}

\section{Functies met meer argumenten}
Het is niet nodig om je tot een enkel argument te beperken. Je kan er ook meer gebruiken, gescheiden door komma's:

\begin{code}
class movingCircle {
  Circle c    ;
  Vec2   speed;
  
  void init(float r, Vec2 pos, Vec2 speed) {
    c.r     = r    ;
    c.pos   = pos  ;
    T.speed = speed;
  }
}

// ergens in je programma
movingCircle mc;
Vec2 p(0.1, 0.4);
mc.init(0.1, p, Vec2(0.3, -0.5));
\end{code}

Ook hier enkele nieuwigheden:
\begin{description}
\item[T] Met \eeClass{T} duiden we aan dat we het over de variabele van de class hebben. Immers, zowel de class als de functie hebben een variabele speed, en we moeten duidelijk maken welke we bedoelen. Je zou ook beide variabelen een andere naam kunnen geven, maar dat is minder duidelijk. 
\item[argumenten doorgeven] Bij het gebruik van de init functie gebruiken we een vooraf gedefinieerde \eeClass{Vec2} om de positie door te geven. Maar voor de snelheid maken we een \eeClass{Vec2} op het moment zelf. Beide opties zijn ok.
\end{description}

\begin{exercise}
Voeg ook een functie init toe aan je class. 
\end{exercise}

\section{Functies met een resultaat}
We stelden in het begin van dit hoofdstuk dat je variabelen van een class eigenlijk niet rechtstreeks zou mogen gebruiken buiten die class. We hebben ondertussen gezien hoe je zo'n waarde aanpast, maar hoe kom je buiten de class te weten wat de huidige waarde van een variabele is? Wel, je maakt een functie met een resultaat:

\begin{code}
class movingCircle {
  Circle c;
  Vec2 speed;
  
  void init(float r, Vec2 pos, Vec2 speed) {
    c.r         = r    ;
    c.pos       = pos  ;
    this->speed = speed;
  }
  
  Vec2 getPos() {
    return c.pos;
  }
}

// ergens in je programma
movingCircle mc;
mc.init(0.1, p, Vec2(0.3, -0.5));
Vec2 pos = mc.getPos();

\end{code}

De \eeClass{Vec2} zal na de laatste regel de waarde 0.1 hebben. Een functie met resultaat maak je op de volgende manier:

\begin{itemize}
\item In plaats van void zet je voor de functienaam het soort resultaat dat de functie zal geven. De positie van een cirkel is een \eeClass{Vec2}, dus in dit geval staat daar \eeClass{Vec2}.
\item In je functie geef je de instructie `return' gevolgd door de waarde je je wil als resultaat. Die waarde zal dan doorgegeven worden op de plaats waar je de functie gebruikt.
\end{itemize}

\begin{exercise}
Voeg deze functie toe aan je class. Vervang in je programma de code om de functie van je cirkel op te vragen door de nieuwe functie.
\end{exercise}

\section{Functies met argumenten en een resultaat}
Ook de combinatie van argumenten en een resultaat is mogelijk. Let wel op: alhoewel een functie meer dan een argument kan hebben, kan er slechts \'e\'en resultaat zijn.

In het volgende voorbeeld maken we een functie 'move'. Deze functie verplaatst de cirkel door de positie in het argument op te tellen bij de huidige positie. Maar ze doet dat enkel wanneer de nieuwe positie binnen het scherm past. Met een boolean resultaat (true of false) laat de functie weten of dat gelukt is.

\begin{code}
class movingCircle {
  Circle c    ;
  Vec2   speed;
  
  void init(float r, Vec2 pos, Vec2 speed) {
    c.r         = r    ;
    c.pos       = pos  ;
    this->speed = speed;
  }
  
  float getRadius() {
    return c.r;
  }
  
  bool move(Vec2 pos) {
    if(Cuts(c.pos + pos, D.viewRect())) {
      c.pos += pos;
      return true;
    } else {
      return false;
    }
  }
}

// ergens in je programma
movingCircle mc;
Vec2 p(0, 0);
mc.init(0.1, p, Vec2(0.3, -0.5));
if( !mc.move(mc.speed * Time.d()) ) {
  // doe iets wanneer dit niet gelukt is
}
\end{code}

\begin{exercise}
Pas nogmaals je class aan en voeg de functie move toe. In je programma kan je nu deze functie gebruiken om de cirkel te verplaatsen. Omdat de functie zelf aangeeft of dat al dan niet gelukt is, kan je nu ook de controle op de grenzen van het scherm weglaten. Wanneer de functie false als resultaat heeft, zet jet de cirkel terug in het midden.

Je kan nu de functie `move' nog verder vereenvoudigen. We weten immers dat we altijd op dezelfde manier willen bewegen. Je kan met andere woorden een functie move ook zonder argument maken en de berekening verplaatsen naar de functie.
\end{exercise}

\section{De oplossing}
Hieronder zie je de volledige oefening zoals je die tot hier toe moest uitwerken. Vergelijk deze code met je eigen uitwerken en vraag uitleg wanneer iets je niet duidelijk is.

De class movingCircle:
\begin{code}
class movingCircle
{
   Circle c    ;
   Vec2   speed;
   Color  color;
   
   void create(float radius, Vec2 pos, Vec2 speed, Color color)
   {
      c.r     = radius;
      c.pos   = pos   ;
      T.speed = speed ;
      T.color = color ;
   }
   
   bool move()
   {
      Vec2 newPos = c.pos + speed * Time.d();
      if(Cuts(newPos, D.viewRect()))
      {
         c.pos = newPos;
         return true;
      } else return false;
   }
   
   Vec2 getPos()
   {
      return c.pos;
   }
   
   void setPos(Vec2 pos)
   {
      c.pos = pos;
   }
   
   void reset()
   {
      c.pos = 0;
   }
   
   void draw()
   {
      c.draw(color);
   }
}
\end{code}

Het programma:
\begin{code}
Memc<movingCircle> circles;

void InitPre()
{
   EE_INIT();
}

bool Init()
{ 
   return true;
}

void Shut() {}

bool Update()
{
   if(Kb.bp(KB_ESC)) return false;
  
   if(Kb.bp(KB_SPACE))
   {
      circles.New().create(RandomF(0.05, 0.1), 
                           Vec2(0), 
                           Vec2(RandomF(-1, 1), RandomF(-1, 1)), 
                           Color(Random(255), Random(255), Random(255))
                           );
   }
   
   if(Kb.bp(KB_F1))
   {
      REPA(circles)
      {
         circles[i].setPos(Vec2(RandomF(-D.w(), D.w()), RandomF(-D.h(), D.h())));
      }
   }
   
   REPA(circles)
   {
      if(!circles[i].move()) circles[i].reset();
   }
   
   return true;
}

void Draw()
{
   D.clear(BLACK);
  
   REPA(circles)
   {
      circles[i].draw();
   }
}
\end{code}

\chapter{Ways to use a Class}

So you know how to create a class, but what is a good way to use them? There is no single best answer to this question, but some viable options are covered in this chapter.

\section{What does belong together?}
The idea behind classes is to group variables and functions which belong together. The the scoring system in a game, for instance. The next items could belong to the scoring system:

\begin{itemize}
\item a variable to hold the current score
\item a variable to hold the high score
\item a method to change the score
\item a method to reset the score
\item a method to check if the current score is higher than the high score
\item a method to request the current score
\item \ldots
\end{itemize}

Depending on the complexity of your game, a lot can be added here. But what does \textsl{not} belong in here is a method to display the score on the screen. You should always separate gui and logic. The score system in this case is part of the game logic. Displaying the score on the screen would be part of the gui. Mixing these will often result in code that is hard to maintain or reuse.

Say you like to display the score on the screen. There's a good chance that you want to show this score during the game, but also when the game is over. Chances are it will not be drawn exactly the same in those cases.

That is why you work with separate gui classes. Both gui classes will be able to request the current score from the score system, but they should not be part of it:


\begin{code}
class score {
  int points = 0;
  
  void reset() { points = 0;    }
  void get  () { return points; }
  void inc  () { points++;      }
}
score Score; // object

// this class will be used during the game
class overlay {
  void draw() {
    D.text(Vec2(0, 0.9), S + "Score: " + Score.getPoints());
  }
}

// this class will be used after the game
class gameOver {
  void draw() {
    D.text(Vec2(0,0), S + "Score: " + Score.getPoints());
  } 
}
\end{code}

\section{Setters and Getters}

It is good practice to use methods to retrieve and change class variables. This prevents you from changing class variables by mistake. It also allow you to do some checks on the new value and maybe trigger some other code when a variable is changed.

These methods are called setters and getters. You can implement them in several ways. Here's one:

\begin{itemize}
\item The method name is equal to the variable name, prepended by set or get.
\item The method changes the value with the same name as the name of the method.
\item The set method will be a void method with the an argument equal to the type of the variable.
\item The get method will have a return type equal to the variable type.
\end{itemize}

Here's an example of the score class (With only a `points' variable) with a set and get method:

\begin{code}
class score {
  int points;
  
  int  getPoints(         ) { return points ; }
  void setPoints(int value) { points = value; }
}
\end{code}

Another way to implement setters and getters would be to use a special symbol in the variable name. Most programmers use an underscore. This approach has the added bonus that there is a clear distinction between class variables and local variables, because all class variables start with an underscore.

\begin{code}
class score {
  int _points;
  
  int  points(         ) { return _points ; }
  void points(int value) { _points = value; }
}
\end{code}

\section{public and private}
The previous examples do not prevent you from changing the class variable directly, without using the set method. This might lead to mistakes when the class is used, especially when the user of this class is another programmer in your team which does not know the ins and outs of your class. Suppose you keep a counter to see how many times the score has been altered. Should the user of your class change the variable directly, the counter will not be updates. Mistakes will follow and computers explode!

This is why you should make all class variables private, which means they can only be altered by methods of the class. Private variables (or even methods) can never be changed from the outside. The counterpart of this is `public'. That's the part of your class you intend to use from the outside.

\begin{code}
class score {
private:
  int points  = 0;
  int counter = 0;
  
public:  
  int getPoints() { 
    return points;
  }
  
  void setPoints(int points) {
    this->points = points;
    counter++;
  } 
}
score Score;

// somewhere in your application
Score.points = 3; // this no longer works
Score.setPoints(3); // this will
\end{code}

\section{Global objects}
Some classes are intended for reuse. During the course of your application, you will need lots of them. Take \eeClass{Vec2} for example. You will constantly create objects of the class \eeClass{Vec2} while doing calculations in your application. Other classes are only intended to create a single object that will be used in your application. In this case you can declare the object below the class definition. It will become a global object which can be used anywhere in your code. The Score object in the previous example is such an object.

\begin{note}
The use of global objects is very common in Esenthel, but should you work with other C++ libraries in the future you will learn that it is not always a good idea. They are part of the design philosophy behind Esenthel and work very well with this design. Other libraries might be designed in a way that makes it very hard or even a impossible to work with global objects. When you find yourself struggling to use global objects with another library, the designer of that library probably had something else in mind.
\end{note}

\section{Manager classes}
\label{section:managerClass}
When you create a class for a moving circle, maybe because you need a lot of moving circles in your application, it is often a good idea to provide a `manager' class. this class will be responsible for managing all your circles: creating, removing, updating and drawing them all at once.

Such a class could look like this:

\begin{code}
class myCircle {
private:
  Circle c;
  
public:
  void update() {
    // insert update code here
  }
  
  void draw() {
    // insert draw code here
  }  
}

class myCircleManager {
private:
  Memx<myCircle> list;
  
public:
  void createCircle() {
    myCircle & temp = list.New();
    // do something with temp?
  }
  
  void update() {
    for(int i = 0; i < list.elms(); i++) {
      list[i].update();
    }
  }
  
  void draw() {
    for(int i = 0; i < list.elms(); i++) {
        list[i].draw();
      }
    }
  }
}

myCircleManager MyCircleManager;

// in your main application
void Init() {
  MyCircleManager.createCircle();
}

void Update() {
  MyCircleManager.update();
}

void Draw() {
  MyCircleManager.draw();
}
\end{code}

\begin{exercise}
\begin{enumerate}
\item Create a class called `movingCircle'. This class contains a \eeClass{Circle}, a \eeClass{Vec2} `direction' and a \eeClass{float} `speed'. 
\item Add a `create' method which provides the circle with a radius and an initial position. Assign a random float between -1 and 1 to both x and y of the direction. Assign a random value between 0.5 and 2 to speed.
\item Add an update method which will add the direction, multiplied with \eeFunc{Time.d()} and speed, to the circle's position. Next, the update method should check the new position. In case the y-coordinate is higher than the height of the screen, its sign should be reversed. Do the same for the x-coordinate and the width of the screen. 
\item Add a draw method which draws the circle on the screen.
\item Add an object of this class to your application. Use the create, update and draw methods to assure they work as they should.

\item Add a class `circleManager' with a memory container for the class `movingCircle'.
\item To this class, add a method `add' which adds a circle to the container.
\item Add an `update' method which calls the update method of every circle in the container.
\item Add a `draw' method to draw all circles on the screen.
\item Since you only need one object of this class, add an object with the name `CM' below the class declaration.

\item Remove the existing circle from your application. \textsl{(Also remove all method calls to this circle.)}
\item Add code to the `Update' function of your application to add a new circle to CM every time the space bar is pressed.
\item Call the update method of CM in the application's Update function. Also, add the draw method of CM to the application's Draw function.

\item challenge: Add a `remove' method to the circle manager. This method has a position as an argument. The method should compare this position with all circles in the container. If the position is inside that circle, the circle should be removed from the container.
\end{enumerate}
\end{exercise}

\section{Werken aan bestaande code}
Wanneer je iets moet aanpassen aan code die al bestaat, dan is het begin dikwijls het moeilijkst. Hierbij enkele hulpmiddelen.

\begin{itemize}
\item Zorg eerst dat je de weet wat waar staat. Bekijk de verschillende classes in het programma en zorg dat je weet welke class wat doet.
\item Kijk van welke classes er globale objecten bestaan.
\item Zoek naar management classes en onthoudt bij welke class die horen.
\item Zijn er nog andere classes die duidelijk samen horen?
\item Het kan geen kwaad om op een blad papier notities te nemen van het voorgaande. Dat helpt je om alles te onthouden en te structureren.
\end{itemize}

Wanneer je dan aan de code werkt, houdt dan het volgende in de gaten:
\begin{itemize}
\item Welke variabelen heeft deze class? Welke informatie heb ik nodig om een functie te laten doen wat ze moet doen?
\item Welke andere functies heeft deze class? Kan ik een van deze functies gebruiken?
\item Indien de class een ander object moet wijzigen, welke functies heeft die class al? Moet je er misschien een toevoegen voor dat je kan doen wat je wil doen?
\item Wordt er een set of get functie gevraagd? In dat geval zijn er regels die het je gemakkelijk maken.
\end{itemize}








\chapter{referenties}
\label{chapter:references}
\section{inleiding}

Bekijk even de volgende code:

\begin{code}
Vec som(Vec pos1, Vec pos2) {
  return pos1 + pos2;
}

// someplace else
Vec p1(0.1, 0.3, 0.5);
Vec p2(1.9, 2.7, 0.5);

Vec p3 = som(p1, p2); 
\end{code}

Alhoewel de bovenstaande code werkt zoals je verwacht, is dit helemaal niet optimaal. Je programma moet heel wat werk verrichten om de juiste waarde voor \texttt{p3} te berekenen.

Uit de leerstof van het 5de jaar heb je (hopelijk) onthouden dat een functie niet weet wat wat er in de rest van het programma gebeurt. In dit geval betekent dat dat de functie som enkel twee waarden ontvangt, die bij mekaar optelt en het resultaat ''terugstuurt'' naar het programma. 

Maar hoe kan de functie \texttt{som() p1} en \texttt{p2} optellen als die onzichtbaar zijn voor deze functie? Het antwoord is eenvoudig: op het moment dat \texttt{som()} uitgevoerd moet worden, kopi\"{e}ert de computer de waarden van \texttt{p1} en \texttt{p2} naar \texttt{pos1} en \texttt{pos2} die bestaan in de functie.

Wat terug naar het programma gaat is de waarde na \texttt{return}. Maar ook nu weet de functie niet af van \texttt{p3}. Dus kopi\"{e}ert de computer de waarde na \texttt{return} naar \texttt{p3}.

Maar nu het slechte nieuws: al dat kopi\"{e}eren kost tijd. Hieronder zie je wat er werkelijk gebeurt tijdens het uitvoeren van de functie \texttt{som()}.

\begin{itemize}
\item kopieer \texttt{p1.x} naar \texttt{pos1.x}
\item kopieer \texttt{p1.y} naar \texttt{pos1.y}
\item kopieer \texttt{p1.z} naar \texttt{pos1.z}
\item kopieer \texttt{p2.x} naar \texttt{pos2.x}
\item kopieer \texttt{p2.y} naar \texttt{pos2.y}
\item kopieer \texttt{p2.z} naar \texttt{pos2.z}
\item reserveer geheugen voor het resultaat (we noemen dit \texttt{result})
\item tel \texttt{pos1.x} bij \texttt{pos2.x} en sla de som op in \texttt{result.x}
\item tel \texttt{pos1.y} bij \texttt{pos2.y} en sla de som op in \texttt{result.y}
\item tel \texttt{pos1.z} bij \texttt{pos2.z} en sla de som op in \texttt{result.z}
\item kopieer \texttt{result.x} naar \texttt{p3.x}
\item kopieer \texttt{result.y} naar \texttt{p3.y}
\item kopieer \texttt{result.z} naar \texttt{p3.z}
\end{itemize}

En dan gebruikt deze functie nog maar eenvoudige vectoren! Wat als de argumenten containers met 2.000 vectoren zijn? Of misschien gebruik je deze \texttt{som()} functie wel op zoveel plaatsen in je code dat ze uiteindelijk 20.000 keer per seconde gebruikt wordt!

Met andere woorden: \emph{wanneer je performante software wil maken, dan moet je in de eerste plaats vermijden dat je objecten kopi\"{e}ert wanneer dat niet nodig is.}

\section{Pass by reference}
De manier waarop we tot hiertoe waarden doorgeven naar een functie, noemen we \textbf{pass by value}. We geven letterlijk de waarde van een object door, we maken met andere woorden een \textbf{kopie} van het object (in dit geval een \texttt{Vec}).

Een andere manier waarop je waarden kan doorgeven, heet \textbf{pass by reference}. Daarmee geven we niet de waarden zelf door, maar een \textbf{referentie} (of verwijzing) naar het object dat die waarden bevat.

De code ziet er dan zo uit:
\begin{code}
Vec som(Vec & pos1, Vec & pos2) {
  return pos1 + pos2;
}

// someplace else
Vec p1(0.1, 0.3, 0.5);
Vec p2(1.9, 2.7, 0.5);

Vec p3 = som(p1, p2); 
\end{code}

Het is dus slechts de \& (ampersand) die het verschil maakt. Dat lijkt misschien niet de moeite, maar het aantal stappen dat nodig is om de functie uit te voeren, kan wel flink dalen:

\begin{itemize}
\item zet een verwijzing naar \texttt{p1} naar \texttt{pos1}
\item zet een verwijzing naar \texttt{p2} naar \texttt{pos2}
\item reserveer geheugen voor het resultaat (we noemen dit \texttt{result})
\item tel \texttt{pos1.x} bij \texttt{pos2.x} en sla de som op in \texttt{result.x}
\item tel \texttt{pos1.y} bij \texttt{pos2.y} en sla de som op in \texttt{result.y}
\item tel \texttt{pos1.z} bij \texttt{pos2.z} en sla de som op in \texttt{result.z}
\item kopieer \texttt{result.x} naar \texttt{p3.x}
\item kopieer \texttt{result.y} naar \texttt{p3.y}
\item kopieer \texttt{result.z} naar \texttt{p3.z}
\end{itemize}

Het is dus bijna altijd een goed idee om een object als referentie door te geven. Enkel bij eenvoudige variabelen, zoals \texttt{int}, \texttt{float} en \texttt{bool} heeft dit geen zin. Het de referentie zou in dat geval niet even veel of zelfs meer geheugen in beslag nemen dan de waarde.

\section{return by reference}
Ongetwijfeld word je na het lezen van de bovenstaande tekst helemaal warm vanbinnen en wil je het voorbeeld nog verder verbeteren. Immers, waarom zou je ook bij de return waarde geen ampersand gebruiken?

Dus zo:
\begin{code}
Vec & som(Vec & pos1, Vec & pos2) {
  return pos1 + pos2;
}

// someplace else
Vec p1(0.1, 0.3, 0.5);
Vec p2(1.9, 2.7, 0.5);

Vec p3 = som(p1, p2); 
\end{code}

Helaas. Dit is geen goed idee. Want \texttt{p3} is nog steeds een gewone \texttt{Vec}, geen referentie naar een \texttt{Vec}. Dus dat zou betekenen:

\begin{itemize}
\item zet een verwijzing naar \texttt{p1} naar \texttt{pos1}
\item zet een verwijzing naar \texttt{p2} naar \texttt{pos2}
\item reserveer geheugen voor het resultaat (we noemen dit \texttt{result})
\item tel \texttt{pos1.x} bij \texttt{pos2.x} en sla de som op in \texttt{result.x}
\item tel \texttt{pos1.y} bij \texttt{pos2.y} en sla de som op in \texttt{result.y}
\item tel \texttt{pos1.z} bij \texttt{pos2.z} en sla de som op in \texttt{result.z}
\item \textbf{geef een referentie naar \texttt{result} als functie resultaat}
\item kopieer \texttt{result.x} naar \texttt{p3.x}
\item kopieer \texttt{result.y} naar \texttt{p3.y}
\item kopieer \texttt{result.z} naar \texttt{p3.z}

\end{itemize}

\texttt{p3} is een \texttt{Vec}, geen referentie naar een \texttt{Vec}. Dus je moet uiteindelijk de referentie toch nog kopieeren naar \texttt{p3}. Je dacht je code sneller te maken, maar ze wordt zelfs trager. Balen dus.

Maar wacht! Waarom maken we van \texttt{p3} dan niet gewoon een referentie? Ook dat is een no-go. Het probleem is dit: result is een tijdelijk object dat enkel tijdens de uitvoering van de functie bestaat. De \texttt{Vec \& p3} zou dan na de uitvoering van de functie een verwijzing bevatten naar result, maar result bestaat niet meer op dat moment.

Dit kan dus nooit goed gaan. Gelukkig zal de compiler je waarschuwen, mocht je dit willen proberen.

Kan je dan nooit een return by reference gebruiken? Toch wel. Kijk maar eens naar het volgende voorbeeld:

\begin{code}
Memc<Vec> points;
Vec & p = points.New();
p.x = 0.1;
...
\end{code}

Waarom kan dit wel? \texttt{points} is een container voor objecten van het type \texttt{Vec}. De functie \texttt{New()} maakt een \texttt{Vec} in die container en geeft een referentie als resultaat. De \texttt{Vec} bestaat dus ook na het uitvoeren van de functie nog steeds, binnen \texttt{points}. In dit geval is \texttt{p} dus een tijdelijke naam voor die \texttt{Vec} in \texttt{points}. Zo'n referentie is heel handig, omdat je via \texttt{p} een object in de container kan wijzigen waar je anders niet bij kan.

Kan dit dan nooit fout gaan? Jawel, maar dan zoek je het zelf. Als je de referentie gebruikt nadat je het object verwijdert, dan gaat het mis:

\begin{code}
Memc<Vec> points;
Vec & p = points.New();
points.clear();
p.x = 0.1; // auch!
\end{code}

\begin{exercise}
Open opnieuw de oefening die je op het eind van hoofdstuk \ref{section:managerClass} maakte. Bekijk \'e\'en voor \'e\'en de functies de je maakte. Vervang waar mogelijk een pass by value door een pass by reference.
\end{exercise}


\chapter{Pointers (Uitbreiding)}

\section{Inleiding}
Alhoewel references dikwijls heel handig zijn, zijn ze niet in alle omstandigheden bruikbaar. Om het gebruik van een reference zou gemakkelijk mogelijk te maken, verplicht de compiler je om de verwijzing te linken aan
een echte variabele op het moment dat ze gemaakt wordt.

Meestal is dat geen probleem. Zo bijvoorbeeld in dit voorbeeld:

\begin{code}
Vec som(Vec & pos1, Vec & pos2) {
  return pos1 + pos2;
}

// someplace else
Vec p1(0.1, 0.3, 0.5);
Vec p2(1.9, 2.7, 0.5);

Vec p3 = som(p1, p2); 
\end{code}

Op het moment dat je de functie som gebruikt, ken je dadelijk p1 toe aan de reference pos1 en p2 aan de reference pos2. So far so good. Maar wat met dit voorbeeld?

\begin{code}
class players {
  Memc<Player> list;
	
  Circle &  add(Vec2 & pos, Str & name) {
    Player & p = list.add();
    p.set(pos, name);
    return p;
  }	  

  Circle & findByName(Str & name) {
    FREPA(list) {
      if(Equal(list[i].getName(), name) {
        return list[i];
      }
    }
  }	
}
// globaal object
players Players;

// ergens in je programma 
Players.add(Vec2(1,1), "niceGuy");
player & vriend = Players.findByName("coolGuy");
\end{code}

De functie \texttt{add()} kan je geen problemen geven. De referentie argumenten kunnen niet anders dan bestaan wanneer je de functie gebruikt. Die functie geeft ook de nieuwe player als reference. Die kan in dit geval ook niet anders dan bestaan, want we hebben die net gemaakt.

De functie \texttt{findByName()} geeft ook een referentie naar een player als resultaat. Dat gaat goed zolang die player ook bestaat. Maar wat als er geen player bestaat met de naam die we zoeken? Dan is er geen resultaat, maar een referentie die niet dadelijk een waarde krijgt, geeft een fout. En dat is nu het geval met \texttt{player \& vriend}.

We zouden dit kunnen oplossen door een andere player als resultaat te geven. Maar dan krijgen we de verkeerde informatie, het is immers niet de player die we zochten. In zo'n geval zou het dus beter zijn als er geen resultaat was, maar dat kan niet met een referentie.

\section{Pointers dus\ldots}
Ook een pointer is een verwijzing naar een variabele. Maar wel met een ander symbool: de asterisk (*). Maar waar de compiler een oogje in het zeil houdt wanneer je references gebruikt, ben je bij pointers volledig op jezelf aangewezen. Pointers zijn niets meer dan een variabele waarin je een geheugenadres kan opslaan.

Stel je voor dat je een kast hebt met lades die allemaal precies 1 item kunnen bevatten. Je steekt je GSM in lade 1. Later verplaats je je GSM naar lade 2 en legt in lade 1 een briefje dat je GSM in lade 2 ligt.

Lade 1 is nu een ``pointer'' naar lade 2. We kunnen nu ook in lade 3 een pointer naar lade 2 steken. Lade 1 en lade 3 verwijzen nu beiden naar lade 2. Maar waarom niet gewoon lade 2 onthouden? In dit geval is het gebruikte systeem met pointers niet zo zinvol, maar het kan wel zinvol zijn.

Stel je voor dat we in alle laden, te beginnen bij 2, een GSM leggen. In lade 1 leggen we een verwijzing naar de GSM in lade 2. Iedereen die moet bellen, doet lade 1 open, en ziet dat de GSM die hij hoort te gebruiken in lade 2 ligt.

Na de GSM in lade twee een tijd gebruikt te hebben, merken we dat hij geen belkrediet meer heeft. We vervangen nu de pointer in lade 1 door een pointer naar lade 3. Iedereen weet nu dat de te gebruiken GSM in lade 3 ligt. Totdat die natuurlijk ook geen belkrediet meer heeft en we een pointer naar lade 4 in lade 1 steken.

In deze situatie heeft de verwijzing, of pointer, al veel zin. Niemand moet nog zoeken naar een bruikbare GSM, tenzij wanneer je de eerste bent die ontdekt dat de huidige GSM geen belkrediet meer heeft.

\section{Basisbewerkingen}
\subsection{Een pointer declareren}
Een pointer is een variabele, net zoals een andere. Alleen, in een pointer variabele sla je geen data op, maar een geheugenadres van data. 

Een pointer declareer je zo:

\begin{code}
int * p1;
int* p2;
int *p3;
Str * textPtr;
Player * playerPtr;
\end{code}

Je ziet dat je nogal wat vrijheid hebt. De eerste drie pointer variabelen zijn identiek. Ze kunnen verwijzen naar een integer. Waar de asterisk staat maakt dus niet uit.

De vierde pointer is een pointer naar een \texttt{Str} (string), terwijl de vijfde pointer naar een object van de eigen klasse \texttt{Player} verwijst.

\subsection{Een adres toekennen aan een pointer.}
Wanneer we een referentie declareren, dan moeten we die dadelijk een waarde geven. Bij een pointer is dat niet nodig.

Maar je weet dat wanneer we een integer gebruiken die we nooit een waarde gegeven hebben, die eender welke waarde kan hebben. Daarom is het een goed idee om een integer dadelijk te initialiseren, bijvoorbeeld met:

\begin{code}
int i = 0;
\end{code}

Ook bij een pointer kan je dat doen. Want zonder initialisatie kan een pointer naar om het even welk geheugenadres verwijzen. En als je vervolgens via de pointer de waarde in dat geheugen wijzigt, dan kan je programma crashen.

Maar je kan niet zomaar het volgende schrijven:
\begin{code}
int i = 42;
int * p = i;
\end{code}

In de code hierboven kennen we de inhoud van i, dus de waarde 42, toe aan p. Maar p verwacht een adres. Je laat p dus het geheugenadres 42 onthouden, niet het adres van i. Om aan te geven dat we het adres --en niet de waarde zelf-- willen doorgeven, gebruik je een ampersand (\&). Dus\ldots

\begin{code}
int   i = 42;
int * p = &i; // declaratie en initialisatie
int * p;      // enkel declaratie
p       = &i; // enkel initialisatie
p       =  i; // Fout! Het adres is nu 42 
\end{code}    

\ldots betekent: sla het adres van i op in p. Je ziet dat je dit dadelijk bij het declareren kan doen, maar ook later in het programma. De laatste regel is trouwens een waar je voor moet uitkijken. De compiler zal deze toewijzing niet verbieden, maar je programma zal waarschijnlijk crashen.

\subsection{Een variabele wijzigen via een pointer}
Wat nu als we i willen wijzigen via de pointer p? \texttt{p = i;} is fout, dat zagen we hierboven. We moeten aanduiden dat we niet p willen wijzigen, maar het geheugen waar p naar verwijst. Om dat aan te duiden hebben we een extra symbool nodig: de asterisk (*).

\begin{code}
int   i = 42;
int * p = i ; // het adres in p wordt het adres van i
*p      = 43; // de waarde op het adres in p wordt 43
\end{code}

We zetten de twee mogelijkheden nog eens op een rijtje:

\begin{code}
p = &i; // slaat het adres van i op in p
*p = i; // slaat de waarde van i op in het adres waar p naar verwijst.
\end{code}

En het kan nog leuker! Indien p1 en p2 beiden pointers naar integers zijn, kan je de waarde waar p2 naar verwijst, opslaan in het adres waar p1 naar verwijst:

\begin{code}
*p1 = *p2;
\end{code}

\subsection{pointers naar null}
Maar wat met een pointer die nergens naar verwijst? Zonder initialisatie verwijst die naar een willekeurig adres in het geheugen. Dikwijls is het nodig dat je programma weet of een pointer toegewezen is of niet. Daarom bestaat er een speciale waarde die aangeeft dat de pointer nergens naar wijst. We noemen dat een ``null pointer''. Je kan dat zo schrijven:

\begin{code}
int * p1 = null;
\end{code}

We zouden nu de code aan het begin van dit hoofdstuk kunnen herschrijven:

\begin{code}
class players {
  Memc<Player> list;
  
  // referenties geven hier geen probleem, dus blijven we die gebruiken	
  Circle &  add(Vec2 & pos, Str & name) {
    Player & p = list.add();
    p.set(pos, name);
    return p;
  }	  

  // pointer result in plaats van een reference	
  Circle * findByName(Str & name) {
    FREPA(list) {
      if(Equal(list[i].getName(), name) {
        return &list[i]; // notice the ampersand!
      }
    }
    // geen player gevonden met deze naam
    return null;
  }	
}
// globaal object
players Players;

// ergens in je programma 
Players.add(Vec2(1,1), "niceGuy");
player * vriend = Players.findByName("coolGuy");

// misschien is coolGuy niet online?
if(vriend != null) {
  Greet(vriend); // veronderstel Greet(player * p);
}
\end{code} 

In dit geval kunnen we verder met als de functie \texttt{findByName()} geen speler vond. Met een referentie kon dat niet. Maar je moet dan wel controleren of je geen null als resultaat kreeg.

\begin{note}
Sommige geheugencontainers, zoals Memc, kunnen tijdens de uitvoering van het programma verplaatst worden. Bijvoorbeeld omdat er op die plaats in het geheugen niet genoeg ruimte was om nieuwe players toe te voegen. Het programma zal dan de hele container verplaatsen naar een andere plaats in het geheugen. De opgevraagde pointer is dan niet meer geldig. Vraag dus elke keer opnieuw het adres aan. Als dat niet kan, of als je zo voortdurend opnieuw hetzelfde adres moet opvragen, dan gebruik je een ander soort container, zoals Memx. Die geeft de garantie dat een gegeven adres nooit wijzigt.
\end{note} 

\section{Alles op een rijtje}
Het is je al opgevallen dat pointers en references beiden de symbolen \& en * gebruiken. En ze hebben een andere betekenis, afhankelijk van de situatie. hieronder zie je een overzicht van de mogelijkheden.

Aangenomen dat de volgende variabelen reeds bestaan:

\begin{code}
int   j   = 43;
int & ref = j ;
int * ptr = &j;
\end{code}

kunnen we de volgende code gebruiken:

\begin{myTable}{references en pointers}{l||l|l}
  & reference & pointer \\ 
\hline Declaratie     &      & \lstinline|int * i;| \\ 
\hline Declaratie en Initialisatie  & \lstinline|int & i = j;|  & \lstinline|int * i = &j;|\\ 
\hline Initialisatie &  & \lstinline| i = &j;| \\
\hline\hline Waarde aanpassen & \lstinline|i = 42;| & \lstinline|*i = 42;| \\
\hline Waarde aanpassen & \lstinline|i = j;| & \lstinline|*i = j;| \\
\hline Waarde aanpassen via ref & \lstinline|i = ref;| & \lstinline|*i = ref;| \\
\hline Waarde aanpassen via ptr & \lstinline|i = *ptr;| & \lstinline|*i = *ptr;| \\
\hline\hline Adres aanpassen &  & \lstinline| i = &j;| \\
\hline Adres aanpassen via ref &  & \lstinline| i = &ref;| \\
\hline Adres aanpassen via ptr &  & \lstinline| i = ptr;| \\ 
 
\end{myTable} 

Zoals je ziet ogen de references een stuk eenvoudiger. Maar het valt ook op dat die references veel minder mogelijkheden hebben. Daarom is het soms echt noodzakelijk om pointers te gebruiken.

\begin{exercise}
\begin{enumerate}
  \item Maak een programma met 5 cirkels, gespreid over de onderkant van je scherm. Je gebruikt best een memory container om de cirkels in op te slaan.
	\item Als je muis zich boven een cirkel bevindt, dan verplaats je die langzaam naar boven.
	\item Maak een functie die als resultaat een pointer geeft naar de hoogste cirkel. Je voert deze functie uit in update en houdt het resultaat bij een een pointer variabele.
	\item In Draw teken je alle cirkels op het scherm. Vergelijk voor het tekenen van de cirkel het adres met de pointer variabele die het adres van de hoogste cirkel bevat. Als die gelijk zijn, teken je de cirkel groen, anders rood..
\end{enumerate}
\end{exercise}

	
\chapter{Enumeratie}
\section{Zo moet het niet...}
Enumeraties of enum's maken het mogelijk om getallen als tekst weer te geven. Als voorbeeld gebruiken we een class enemy. Die enemy kan een warrior, een rogue of een priest zijn, en afhankelijk daarvan moet er een andere afbeelding getoond worden. Je zou boolean's kunnen gebruiken om te onthouden class de enemy heeft:

\begin{code}
class enemy {
  bool warrior = false;
	bool rogue   = false;
	bool priest  = false;
	Rect r;
	
	void setWarrior() {
		warrior = true ;
		rogue   = false;
		priest  = false;
	}
	
	// de functies setRogue en setPriest zijn gelijkaardig
	// ...
	
	void draw() {
	  if     (warrior) Images(=== warriorImage ===).draw(r);
		else if(rogue  ) Images(=== rogueImage   ===).draw(r);
		else if(priest ) Images(=== priestImage  ===).draw(r);
	}
}
\end{code}

Alhoewel bovenstaande code werkt, is ze niet erg efficient. Nu gaat het nog maar om 3 classes, maar hoe meer mogelijkheden je hebt, hoe meer variabelen je moet aanpassen bij het selecteren van een class. Veel beter zou zijn om slechts \'e\'en variabele te gebruiken, want een enemy kan tenslotte maar 1 class hebben.

\section{Dit is niet veel beter.}
Je zou kunnen beslissen dat een warrior het cijfer 0 krijgt, een rogue het cijfer 1 en een priest het cijfer 2. Dan wordt de code al veel eenvoudiger.

\begin{code}
class enemy {
  int type = -1;
	Rect r;
	
	void setType(int type) {
		T.type = type;
	}
	
	void draw() {
		switch(type) {
			case 0: Images(=== warriorImage ===).draw(r); break;
			case 1: Images(=== rogueImage   ===).draw(r); break;
			case 2: Images(=== priestImage  ===).draw(r); break;
	  }
	}
}
\end{code}

De bovenstaande code is beter dan de eerste versie, maar het is nogal onwaarschijnlijk dat je niet vergeet welk nummer voor welke class staat. Of misschien gebruik je ergens een getal waarvoor geen class is voorzien. 

\section{Enumeration time!}
De oplossing voor dit probleem zijn enum's. Enumeraties zijn lijsten van woorden. Intern wordt het eerste woord gelijk aan 0 en krijgt elk volgend woord een hoger nummer. Je kan echter steeds die woorden gebruiken in plaats van dat nummer. 

\begin{code}
enum ENEMY_TYPE {
	ET_NONE   ,
  ET_WARRIOR,
  ET_ROGUE  ,
	ET_PRIEST , 
}

class enemy {
  ENEMY_TYPE type = ET_NONE;
	Rect r;
	
	void setType(ENEMY_TYPE type) {
		T.type = type;
	}
	
	void draw() {
		switch(type) {
			case ET_WARRIOR: Images(=== warriorImage ===).draw(r); break;
			case ET_ROGUE  : Images(=== rogueImage   ===).draw(r); break;
			case ET_PRIEST : Images(=== priestImage  ===).draw(r); break;
		}
	}
}
\end{code}

Het voordeel van deze code is dat je overal in je programma de waarden \texttt{ET\_WARRIOR} of \texttt{ET\_ROGUE} kan gebruiken. Je kan de functie setType ook geen getal meegeven van een class die niet bestaat. En bovendien zie je op elk moment duidelijk over welke vrucht je het hebt. 

De waarden van een enum schrijven we in hoofdletters. Dit is niet verplicht, maar wel een conventie. Je kan immers nooit het getal waar een enumeratie waarde voor staat, aanpassen. Iets als \texttt{ET\_WARRIOR = 3} is dus onmogelijk. Aangezien \texttt{ET\_WARRIOR} het tweede woord in de rij is, is zijn waarde steeds 1.

Het is ook niet nodig om elke waarde met \texttt{ET\_} te beginnen, maar in een groter programma is het dikwijls handig om een afkorting van de enumeratie te gebruiken. ``ENEMY\_TYPE'' wordt zo ET. Op die manier kan de autocomplete je helpen met het kiezen van een naam zodra je \texttt{ET\_} getypt hebt.

\begin{note}
Esenthel bevat ook een \textsl{enumeration editor}. Hiermee kan je grafisch de waarden van een enumeratie type ingeven. Deze waarden zijn dan bruikbaar in zowel je code als de world editor.
\end{note}


\chapter{Constanten}
\section{Globale Constanten}
Dikwijls gebruik je bepaalde waarden doorheen je hele programma. Zo zou je, in een programma dat veel berekeningen met circels moet doen, vaak het getal pi nodig hebben. Dat getal kan je elke keer opnieuw berekenen, maar dat is niet zo'n goed idee omdat de uitkomst van je berekening steeds hetzelfde is. Je zou daarom een globale variabele pi kunnen declareren:

\begin{code}
int pi = 3.1415926;
\end{code}

Nu kan je overal de waarde van pi gebruiken in je berekeningen. Maar stel je voor dat je ergens vergist:

\begin{code}
int value = 1;
// ... more code ...
if(pi = value) {
  // do something
}
\end{code}

Je wil in de bovenstaande code controleren of `value' gelijk is aan pi. Maar je schrijft een enkele in plaats van een dubbele =. Zo'n fout is snel gemaakt en valt op het eerste zicht niet zo op. Het gevolg is dat na de uitvoering van die code het getal pi niet meer gelijk is aan zijn oorspronkelijke waarde. Al je berekeningen zullen dus fout zijn!

Om dit soort fouten te voorkomen voorzien de meeste programmeertalen in een mogelijkheid om een variabele `constant' te maken. Dat wil zeggen dat ze na hun declaratie niet meer aangepast mogen worden. Om dat duidelijk te maken bestaat de afspraak om die variabelen steeds met hoofdletters te schrijven.

Hoe schrijf je zo'n variabele? In C++ doe je dat door voor het type \verb|const| toe te voegen. Esenthel geeft je daarnaast de mogelijkheid om dat af te korten tot \verb|C|. (Net zoals je \verb|this| kan afkorten tot \verb|T|). Je schrijft dus:

\begin{code}
C PI = 3.1415926;
\end{code}

Dit heeft twee voordelen:

\begin{enumerate}
	\item Je kan de waarde van PI niet langer per vergissing aanpassen.
	\item Als je het getal PI in je code wil aanpassen, dan moet je dat maar op \'e\'en plaats doen. \textsl{(In het geval van PI is dat wel h\'e\'el onwaarschijnlijk, maar bij andere constanten kan dat dikwijls wel. Als je bijvoorbeeld een constante variabele ATTACK\_RANGE gebruikt, dan kan je misschien later beslissen dat die toch iets te groot is.)} 
\end{enumerate}

\begin{note}
Omdat PI een nummer is dat alle programmeurs vaak nodig hebben, bestaat er al een constante PI in Esenthel. Niet enkel dat, er zijn ook al varianten voorzien, zoals PI\_2 (de helft van PI) en PI2 (twee maal PI).
\end{note}

\begin{exercise}
Maak een programma met de volgende constanten: playerColor, playerSize, enemyColor en enemySize. De player is een rechthoek en de enemies zijn cirkels. \textit{(Het is een erg abstract spel.)} Toon een speler en verschillende enemies op het scherm.
\end{exercise}

\section{Const Argumenten}
Er bestaat nog een andere situatie waarin je constanten gebruikt. Bekijk even de volgende functie:

\begin{code}
float calculateDistance(Vec2 & pos1, Vec2 & pos2);
\end{code}

Je kan deze functie gebruiken om de afstand tussen twee posities te berekenen. Je leerde al in hoofdstuk \ref{chapter:references} dat we de argumenten van die functie by reference doorgeven om het programma sneller te maken. Dat heeft \'e\'en nadeel. Je zou in principe de waarden van pos1 en pos2 kunnen aanpassen in de functie. En dan zijn ook de originele waarden in je programma aangepast. De naam van de functie laat in dit geval vermoeden dat dat niet zal gebeuren. Maar je weet nooit zeker of de progammeur van die functie zich niet vergist heeft.

Als er dus ergens iets fout gaat met de variabele \verb|pos1| in je programma, dan kan je niet anders dan ook de code van de functie \eeFunc{calculateDistance} nakijken. En misschien gebruikt die functie intern nog een andere functie die eveneens pass by reference argumenten heeft. Dat zou betekenen dat je echt alle onderliggende functies moet nakijken om uit te sluiten dat de fout daar zit.

Zoiets is in grote projecten niet werkbaar. En daarom kunnen we ook een functie argument constant maken, net zoals een globale variabele. Je schrijft de functie dan zo:

\begin{code}
float calculateDistance(C Vec2 & pos1, C Vec2 & pos2);
\end{code}

De gevolgen zijn dat:

\begin{enumerate}
	\item je tijdens het maken van de functie een foutmelding krijgt wanneer je toch zou proberen pos1 of pos2 aan te passen;
	\item de gebruiker van je functie zeker weet dat de waarde nooit aangepast kan zijn in de functie;
	\item je bijna zeker weet dat een functie waar de argumenten \textbf{niet} constant zijn, die argumenten zal aanpassen.
\end{enumerate}

Vanaf nu volg je dus de regel dat je alle functie argumenten als een const reference doorgeeft, tenzij het de bedoeling is dat de aangepaste waarde in het oorspronkelijke programma terecht komt.

Wat is nu een goede reden om een argument aan te passen? Kijk even naar de Esenthel functie:
\begin{code}
void Clamp(Vec2 & value, C Vec2 & min, C Vec2 & max);
\end{code}

Het is de bedoeling dat deze functie de eerste waarde binnen het gevraagde minimum en maximum houdt. Je gebruikt de functie op deze manier:

\begin{code}
Vec2 pos = Ms.pos();
Clamp(pos, Vec2(-0.4,-0.4), Vec2(0.4,0.4));
pos.draw(RED);
\end{code}

Het tweede en derde argument zijn constant. De functie \eeFunc{Clamp} kan dus niet het minimum of maximum aanpassen. Maar \eeFunc{pos} willen we natuurlijk net wel aanpassen. Hier gebruik je dus geen const reference.

\begin{exercise}
\begin{itemize}
	\item Doorzoek de Engine code naar meer functies die geen const reference gebruiken. Probeer te verklaren waarom ze dat niet doen.
	\item Schrijf een functie `ClampToScreen' die een gegeven coordinaat aanpast wanneer het buiten het scherm zou vallen. Test de functie met een eenvoudig programma. Gebruik je een const reference of niet?
	\item Schrijf een functie met een string argument die die string op het scherm plaatst. Je maakt een versie met een const reference en een versie met een gewone reference. Test beide versies met bestaande strings en met string literals. Waarom werkt de niet-const versie enkel met strings en niet met literals?
\end{itemize}
\end{exercise}
\chapter{Application States}
\label{chapter:application_states}

Een application state is een ``status'' van het programma. Wanneer twee delen van een programma nooit gecombineerd worden, dan kan je er twee afzonderlijke application states van maken. 

Wat veel voorkomt is bijvoorbeeld een game lobby en het eigenlijke spel. Je zal nooit de elementen van een game lobby combineren met het spel, dus kan je die volledig scheiden. Ook een login module kan een afzonderlijke application state zijn.

Elk programma heeft al een default application state. Die bestaat uit de functies \texttt{Init()}, \texttt{Shut()}, \texttt{Update()} en \texttt{Draw()}. Wanneer een state actief wordt, dan wordt \texttt{Init()} uitgevoerd. Daarna worden \texttt{Update()} en \texttt{Draw()} afwisselend uitgevoerd, totdat je het programma sluit of overgaat naar een andere state. Op dat moment wordt \texttt{Shut()} uitgevoerd.

\section{Intro}
Voor elke state maak je een afzonderlijk bestand. Bijvoorbeeld voor een intro state:

\begin{code}
bool InitIntro() {return true;}

void ShutIntro() {}

bool UpdateIntro()
{
   if(Time.stateTime()>3 || Kb.bp(KB_ESC)) {
      StateMenu.set(1.0);                    
   }
   return true;
}

void DrawIntro()
{
   D.clear(BLACK);
   D.text (0, 0, "Intro");
}

State StateIntro(UpdateIntro, DrawIntro, InitIntro, ShutIntro);
\end{code}

Je ziet dat dit veel lijkt op de standaard states in je programma. We voegen gewoon het woord Intro toe aan Init, Shut, Update en Draw. Dit houdt het overzichtelijk.

De eigenlijke state zit in de laatste regel:

\begin{code}
State StateIntro(UpdateIntro, DrawIntro, InitIntro, ShutIntro);
\end{code}

Daar geef je aan dat er een nieuwe gamestate is (\eeFunc{StateIntro}) die de typische functies voor een programma bevat. Een \eeFunc{InitPre()} functie kan je niet toelaten, die dient enkel voor de echte start van het programma.

Kijk ook even naar de constructor van \eeFunc{State}:

\begin{code}
State(Bool (*update)(), void (*draw)(), Bool (*init)()=NULL, void (*shut)()=NULL); 
\end{code}

Komt de asterisk (*) je bekend voor? Inderdaad, we hebben met pointers te maken. Pointers naar functies in dit geval. De constructor verwacht dat we aangeven waar de functies voor deze state staan. We verwijzen dus naar de functies die we net gemaakt hebben: \eeFunc{UpdateIntro()} en \eeFunc{DrawIntro()}. Je zegt eigenlijk ``zolang deze state actief is, voer je \eeFunc{UpdateIntro()} uit in plaats van de gewone \eeFunc{Update()} functie.

De volgende twee argumenten, voor de functies \eeFunc{InitIntro()} en \eeFunc{ShutIntro()} zijn optioneel. Je mag ze weglaten als er niets bijzonders moet gebeuren op dat moment.

\begin{note}
Indien een functie argument eindigt met \eeFunc{=NULL}, dan mag je het weglaten.
\end{note} 

\section{Menu}
De code hierboven bevat ook een verwijzing naar StateMenu:

\begin{code}
   if(StateActive.time()>3 || Kb.bp(KB_ESC)) {
      StateMenu.set(1.0);                    
   }
\end{code}

Met andere woorden: we wachten tot de huidige state 3 seconden actief is, of totdat de gebruiker op escape drukt. Dan zetten we een nieuwe application state actief met een crossfade van 1 seconde.

Deze nieuwe state zou er zo kunnen uitzien:
\begin{code}
bool InitMenu() {return true;}
void ShutMenu() {}

bool UpdateMenu()
{
   if(Kb.bp(KB_ESC))return false;
   if(Kb.bp(KB_ENTER))StateGame.set(0.5);
   return true;
}

void DrawMenu()
{
   D.clear(GREY);
   D.text (0,  0  , "Menu");
   D.text (0, -0.3, "Press Enter to start the game");
   D.text (0, -0.5, "Press Escape to exit");
}

State StateMenu(UpdateMenu, DrawMenu, InitMenu, ShutMenu);
\end{code}

Deze state lijkt sterk op de vorige. Maar dit maal kunnen we met Enter naar de game zelf. En dat is dan ook weer een nieuwe application state: \eeFunc{StateGame}.

\section{Game}
Deze code kan je voor \eeFunc{StateGame} gebruiken. Maak ook nu weer een afzonderlijk bestand.
\begin{code}
bool InitGame() {return true;}
void ShutGame() {}

bool UpdateGame()
{
   if(Kb.bp(KB_ESC))StateMenu.set(1.0);
   return true;
}

void DrawGame()
{
   D.clear(TURQ);
   D.text (0, 0, "Game");
}

State StateGame(UpdateGame, DrawGame, InitGame, ShutGame);
\end{code}

Door tijdens de game op escape te drukken, schakelen we terug naar \eeFunc{StateMenu}. In deze state ga je bij een echte game natuurlijk nog heel veel code moeten toevoegen.

\section{Default State}
Dan rest ons nog het starten van het programma. We hebben nu alle nodige states, maar \eeFunc{StateIntro} moet nog actief worden. Dit gebeurt door in de \eeFunc{Init()} functie van het programma dadelijk door te schakelen naar \eeFunc{StateIntro}. De functies \eeFunc{Update()} en \eeFunc{Draw()} worden in dit programma dus niet gebruikt.

\begin{code}
void InitPre()
{
   EE_INIT();
}

bool Init()
{
   StateIntro.set();
   return true;
}

void Shut() {}
bool Update() {return false;} // unused
void Draw  () {             } // unused
\end{code}

\begin{exercise}
Gebruik de code van dit hoofdstuk om een programma te maken dat wisselt tussen de voorziene application states. Elke state plaats je in een afzonderlijk bestand.
\end{exercise}


\part{Tetris}
\chapter{Inleiding}
Je hebt in de vorige hoofdstukken alles gezien om een eenvoudige 2D game te maken. Maar hoe breng je dat nu op een overzichtelijke manier samen in een groot project? Er bestaat eigenlijk niet \'e\'en antwoord daarop. Je leert dat vooral door ervaring op te doen. Wat voor de ene persoon werkt, vind de andere misschien minder goed. Toch zijn er een aantal regels die het je zeker makkelijker kunnen maken. En wanneer je in groep werkt dan zal de lead programmer meestal ook een aantal regels opleggen die iedereen moet volgen. Dat zijn niet noodzakelijk goede of slechte regels, maar ze werken zolang iedereen hetzelfde doet.

In dit deel van de cursus bouw je een project op voor een tetris kloon. Je leert om stap voor stap een project uit te werken zonder dat je het overzicht kwijt raakt.

\begin{note}
Tetris bestaat uit blokken die dan weer bestaan uit vierkanten. In deze cursus bedoelen we met een blok steeds het hele tetris figuurtje. Staat er `square' of vierkant, dan gaat het om de vierkantjes die samen een blok vormen.
\end{note}

\section{Setup}
Open eerst het project `Tetris\_start'. Daarin vind je alvast de graphics, geluiden en fonts die we gaan gebruiken. Er is ook alvast een lege app `Tetris' voorzien, maar die gaan we nog niet dadelijk gebruiken.

\begin{enumerate}
	\item Maak eerst op het hoogste niveau een Library aan. Dat doe je door rechts te klikken en `new library' te kiezen. Je geeft deze library de naam `Tetris parts'. Je Een library is een groene map. De code in een library kan je gebruiken vanuit elke applicatie binnen je project, net zoals de library `Esenthel Engine' die altijd aanwezig is.
	\item Maak ook een nieuwe applicatie (blauwe map). Die noem je `squareTester'.
	\item In de applicatie `squareTester' maak je een code bestand `main'. 
	\item In de library `Tetris parts' maak je een nieuwe folder (geel) met de naam `definitions'.
	\item Maak van `squareTester' de actieve applicatie.
\end{enumerate}
	
Je kan in het bestand `squareTester/main' de code van `Tetris/initState' overnemen. Verwijder dan wel de regel 
	
\begin{code}
D.full(true);
\end{code}

Dat maakt het makkelijker om je programma te be\"indigen wanneer er iets fout gaat.	

\section{Constants}
In het vorige hoofdstuk leerde je over constanten. Het is een goed idee om, voordat je aan de echte code begint, enkele belangrijke constanten vast te leggen. Die kan je dan overal in je code gebruiken en later eenvoudig aanpassen als dat nodig blijkt. Je begint daarom in de map `Tetris parts/definitions' met een nieuw code bestand dat je `constants'noemt.

Om later de naam van het programma eenvoudig te wijzigen, maken we alvast een constante \eeClass{Str} met de voorlopige naam.

\begin{code}
C Str APP_NAME = "Tetris";
\end{code}

De grootte van het standaard applicatie window is niet ideaal voor dit spel. Die grootte moet in pixels worden ingegeven. We declareren daar een constante \eeClass{int} voor, zodat we die later eenvoudig kunnen aanpassen.

\begin{code}
// the window size on the screen, in pixels
C int WINDOW_WIDTH  = 900;
C int WINDOW_HEIGHT = 800;
\end{code}

Tetris bestaat uit rijen en kolommen. Ook die leggen we vast:

\begin{code}
// this impacts the playing field
C int SQUARES_PER_ROW = 10;
C int ROWS            = 15;
\end{code}

Over de score kunnen we ook al iets zeggen. Er moeten een aantal levels zijn, de punten die je per lijn en per level krijgt liggen ook vast.

\begin{code}
// the score system uses these
C int POINTS_PER_LINE  =  525;
C int POINTS_PER_LEVEL = 6300;
C int NUM_LEVELS       =    5;
\end{code}

De snelheid van het spel gaat elk level omhoog. We kunnen die begrippen ook al vastleggen.

\begin{code}
// the speed will increase every level
C float INITIAL_SPEED = 1.0;
C float SPEED_CHANGE  = 0.1;
\end{code}

Als je tetris speelt, dan is er een korte periode waarin je een blok dat de onderkant van het spel bereikt nog opzij kan schuiven. Die periode leggen we ook vast.

\begin{code}
// the time a block can be slided to the side
// when it hits bottom
C float SLIDE_TIME = 0.25;
\end{code}

Dan moeten we bepalen hoe groot het speelveld is. We leggen daarom de linkeronderhoek vast, en de grootte van de rechthoek die we daarop toepassen.

\begin{code}
// the area reserved for the playing field 
C Vec2 GAMEAREA     (-0.8, -0.8);
C Vec2 GAMEAREA_SIZE( 1.0,  1.4);
\end{code}

Een nieuw blok verschijnt altijd bovenaan het scherm. Waar dat precies is, dat kunnen we afleiden uit de waarden die we al hebben: SQUARES\_PER\_ROW en ROWS. Daarnaast is er ook nog een wachtpositie, die je meestal bovenaan rechts naast het speelveld toont. Het valt je misschien op dat we hier geen \eeClass{Vec2} gebruiken, maar een \eeClass{VecI2}. Dat is een vector waar enkel gehele getallen in passen. We willen geen tetris waar blokken halverwege tussen twee posities kunnen zitten.

\begin{code}
// position for the current and next block
C VecI2 STARTPOS(SQUARES_PER_ROW / 2, ROWS - 1);
C VecI2 WAITPOS (SQUARES_PER_ROW + 4, ROWS - 3);
\end{code}

Tot slot kunnen we met de voorgaande constante ook de grootte van een vierkant berekenen. Dat heeft het voordeel dat we later de voorgaande constanten kunnen aanpassen en dat de grootte van een vierkant dan vanzelf aangepast wordt.

\begin{code}
// the size of a square
C float SQUARE_SIZE = GAMEAREA_SIZE.x / SQUARES_PER_ROW;
\end{code}

\begin{note}
Het zal natuurlijk zelden gebeuren dat je bij de start aan een project al perfect weet welke constanten je nodig zal hebben. In praktijk zal je dus meestal enkele waarden vastleggen en daarna de lijst aanvullen wanneer je merkt dat je nog constanten nodig hebt.
\end{note}

\section{Enumeraties}
Maak in de folder `Tetris parts/definitions' een nieuw code bestand `enumerations'. hierin voorzie je alvast twee enum's die je in het project nodig zal hebben. Ten eerste is er het blok type. Tetris heeft vierkante blokken, T blokken enzovoort. Een lijst kan er zo uitzien:

\begin{code}
enum BLOCK_TYPE
{
   BT_SQUARE     ,
   BT_T          ,
   BT_L          ,
   BT_BACKWARDS_L,
   BT_STRAIGHT   ,
   BT_S          ,
   BT_BACKWARDS_S,
   BT_NUM        , // number of block types used in the game
   BT_BACKGROUND , // special case, only for background
   BT_WALL       ,      
}
\end{code}

De drie laatste waarden  verdienen wat extra aandacht. De waarde `BT\_NUM' is handig omdat het nummer waar die waarde voor staat, gelijk is aan de mogelijkheden + 1. Dat maakt het eenvoudig om een random functie te gebruiken. Die is immers exclusief de hoogste waarde. Zo kunnen we later in het programma eenvoudig de volgende functie gebruiken:

\begin{code}
blockType type = Random(BT_NUM);
\end{code}

Waarom staat BT\_NUM dan niet op het eind? Wel, de laatste twee waarden zijn speciale gevallen. Het is handig voor de afwerking van het spel om ook vierkanten te gebruiken om de achtergrond en de randen van het spel te tekenen. En de kleur van zo'n vierkant wordt bepaald door het blok type. Omdat we niet willen dat die types ook in het spel gebruiken, plaatsen we BT\_NUM daar voor.

Een tweede enumeratie hebben we nodig om de mogelijke richtingen van een blok te bepalen. Een blok kan nooit naar boven, maar wel naar links, naar rechts, of naar beneden. Blokken die al beneden zijn, hebben geen richting meer.

\begin{code}
enum DIRECTION
{
   D_LEFT ,
   D_RIGHT,
   D_DOWN ,
   D_NONE ,
}
\end{code}

\chapter{Objects}
In Tetris you think mainly of blocks as the main item you move around. The fact that blocks exist of squares is ancillary to the player, but it is important for the developer. To check on possible movements all squares have to be evaluated, not the blocks on their own. And when a block reaches the bottom it becomes a part of a lot of squares, which can be removed row-by-row, regardless of the shape of the original block.

\section{Squares}
Create a folder called `objects' in the library `Tetris parts'. In it, create a code file `square'. This file should contain a class \eeClass{square} in which we describe a square.

A square must have a position. Because the playing field in Tetris is a grid (we can't put a square anywhere we want!), we will use another \eeClass{VecI2} to store integer positions only. Furthermore, the square must have a \eeClass{block} type to determine in what color it should be drawn. We have already defined the types of blocks in the `enumerations' file.

The class \eeClass{square} will also need some methods. We provide a \eeFunc{create} method to set the position and type of the block. In addition, we need a \eeFunc{move} method to move the block in a certain direction. (That direction will be the second enum we declared.) We also need a method to retrieve the current position, and a method to change the block's position directly. And finally, we want a method to draw a square on the screen.

The class looks like this:

\begin{code}
class square
{
private:
   VecI2      pos ; 
   BLOCK_TYPE type;
   
public:
   void create (C VecI2 & pos, BLOCK_TYPE type) {}
   void move (DIRECTION dir) {}
   VecI2 getPos(             ) C { }
   void setPos (C VecI2 & pos)   { }
	 void draw   (             ) C { }
}
\end{code}

Add the above code to your project. 

\subsection{Square Tester}
You'd expect this is when we add the content of these methods. Instead, we will work on the application `square tester'. You know you should test your code very frequently during the development of a program. But in a major project that is difficult. We have to write lots of classes before you can actually run the application.

That's why we use test programs. These will be written to test a specific class. In this case, we make sure that all methods of the class \eeClass{square} can be tested. The code for `square tester' could look like this:

\begin{code}

Memc<square> squares;

void InitPre()
{
   EE_INIT();
}                
       
bool Init()
{   
	 // Here you create the test function, with different
	 // block types.
   squares.New().create(VecI2( 2,  2), BT_S          );
   squares.New().create(VecI2( 4,  2), BT_T          );
   squares.New().create(VecI2( 6,  7), BT_L          );
   squares.New().create(VecI2(10, 12), BT_BACKWARDS_S);
   squares.New().create(VecI2( 8,  8), BT_BACKWARDS_L);
   squares.New().create(VecI2( 5,  1), BT_SQUARE     );
	 return true;
}

void Shut() {}

bool Update()
{
   if(Kb.bp(KB_ESC)) return false;
   
	 // We declare a direction and control the arrow keys
   DIRECTION d = D_NONE;
   if(Kb.bp(KB_DOWN )) d = D_DOWN ;
   if(Kb.bp(KB_LEFT )) d = D_LEFT ;
   if(Kb.bp(KB_RIGHT)) d = D_RIGHT;
   
	 // Next we move all the blocks in this direction. When
	 // no key is pressed, the squares should not move.
   REPA(squares)
   {
      squares[i].move(d);
   }
   
	 // The methods getPos and setPos should also be tested. We are doing that
	 // when the spacebar is pressed. We retrieve the current position
	 // of each square and change the vertical value. The altered position
	 // will be put back into the square.
   if(Kb.bp(KB_SPACE))
   {
      REPA(squares)
      {
         VecI2 pos = squares[i].getPos();
         pos.y += 4;
         squares[i].setPos(pos);
      }
   }
   
   return true;
}

void Draw()
{
   D.clear(BLACK);
	
	 // Here we test the draw function of each square.
   REPA(squares)
   {
      squares[i].draw();
   }
}

\end{code}

\begin{note}
You have often many ways to write a test program. The important thing is that you test as many class methods in the easiest way possible. This minimizes the chance of errors later on.
\end{note}

\subsection{Create and Draw}
The Create method of square is rather straightforward. You have to make sure the arguments are stored in the class variables. Complete this method on your own.

Slightly more difficult is the draw function. Let's sum up what this method should do.

\begin{enumerate}
  \item Depending on the BLOCK\_TYPE, a color should be set. 
	\item The position of the square is the position in the grid. We need to calculate the actual position on the screen.
	\item We have to show an image on the screen, at the calculated position.
\end{enumerate}

The color will be determined with a switch statement. We declare a variable of the type \eeClass{Color} and assign the value BLACK. Next, this is changed to the desired color for each type:

\begin{code}
Color color(BLACK);
      
switch(type)
{
	 case BT_SQUARE     : color = RED   ; break;
	 case BT_T          : color = PURPLE; break;
	 case BT_L          : color = GREY  ; break;
	 case BT_BACKWARDS_L: color = BLUE  ; break;
	 case BT_STRAIGHT   : color = GREEN ; break;
	 case BT_S          : color = PINK  ; break;
	 case BT_BACKWARDS_S: color = YELLOW; break;
	 case BT_BACKGROUND : color = Color(50, 50, 50) ; break;
	 case BT_WALL       : color = WHITE ; break;
}
\end{code}

Next we have to determine the position. The constant GAMEAREA is the lower left corner of the playing field. All positions are calculated from that point. The bottom left corner of the square at the position (0,0) would therefore have to be equal to the lower left corner of the playing field. Therefore:

\begin{code}
Vec2 screenpos = GAMEAREA;
\end{code}

Suppose you want a square at position (1,0). The bottom left corner of the square is equal to that of the playing field, plus the width of one square. For all other positions, the same applies: you multiply the grid position with the size of the square:

\begin{code}
Vec2 screenpos = GAMEAREA + (pos * SQUARE_SIZE);
\end{code}

The upper right corner of the square is exactly one square further. We can create a rectangle to draw on the screen in the following way:

\begin{code}
Vec2 screenpos = GAMEAREA + (pos * SQUARE_SIZE);
Rect r(screenpos, screenpos + SQUARE_SIZE);
\end{code}

To draw the actual image on the screen, use the following code: 

\begin{code}
Images(=== tetris square ===).draw(color, TRANSPARENT, r);
\end{code}

\subsection{Test}
At this time the square tester should show the blocks on the screen, even though they can not move yet. Run the tester to confirm this.

The functions move, getPos and setPos are not done yet, but you can surely complete those on your own. Afterwards, verify whether all methods are correct by running the square tester.

\section{Blocks}
Now add a code file `block' in the folder `Tetris parts/objects'. Create the class \eeClass{block} in that file. Just like a square, a block has a position and a type. But in addition, a block is a bunch of squares. So you need a container for squares inside a block.

And then there are methods. \ eeClass{block} needs a create method, just like a square. And we also provide a second create method with other arguments to copy an existing block.

We also need methods to move, rotate and draw a block. And finally we need a method to retrieve the type of a block and a method which returns the list of squares inside the block. These methods only contain a return statement. Because your test application will not work as long as these methods don't contain a return statement, they are already present in the code below. The result is this class:

\begin{code}
class block
{
private:
   VecI2        pos    ;
   BLOCK_TYPE   type   ;  
   Mems<square> squares;
   
public:
   void create(C VecI2 & pos  , BLOCK_TYPE type) { }
   void create(C block & other                 ) { }
	
   void move  (DIRECTION dir)   { }
   void rotate(             )   { }
   void draw  (             ) C { }
	
   BLOCK_TYPE       getType   () C { return type   ; }
	 C Mems<square> & getSquares() C { return squares; }
}
\end{code}

Add this code in the file `block' that you created before. Now create a new application `block tester'. Add code in this application to test a block. This time you do not need a container. Just declare a block and use the create, move, rotate and draw methods. The extra create method and the methods \eeFunc{getSquares()} and \{eeFunc{getType()} should not be used yet.

\subsection{Add Squares}
Before you start with the create and draw methods, create some 'helper' functions to make it easier on yourself. These methods are in the class \eeClass{block}, but they are private. The first function is \eeFunc{makeSBlock}. Which looks like this:

\begin{code}
void makeSBlock()
{
	//    [0][1]
	// [3][2] 
	
	squares.New().create(VecI2(pos.x    , pos.y    ), BT_S);
	squares.New().create(VecI2(pos.x + 1, pos.y    ), BT_S);
	squares.New().create(VecI2(pos.x    , pos.y - 1), BT_S);
	squares.New().create(VecI2(pos.x - 1, pos.y - 1), BT_S);     
}
\end{code}

Each block consists of 4 squares. The first square will have the same position as the block. This is important to properly rotate the block later. The other squares should get a position that is derived from the first position. You can look at the diagram in the comments to get a better picture of the positions. The second argument (BT\_S) indicates the type of the block. It is important for drawing the square in the right color. You can now create your own functions for the other blocks. Below are the declarations with a diagram for each block.

\begin{code}
void makeSquareBlock()
{
	// [0][2]
	// [1][3]
}

void makeTBlock()
{
	//    [1]
	// [2][0][3]
}

void makeLBlock()
{
	// [2]
	// [1]
	// [0][3]  
}

void makeBackwardsLBlock()
{
	//    [2]
	//    [1]
	// [3][0]  
}

void makeStraightBlock()
{
	// [2]
	// [1]
	// [0]
	// [3] 
}

void makeBackwardsSBlock() 
{
	// [1][0]
	//    [2][3]
}
\end{code}

We now create an additional private function \eeFunc{setupSquares()}. In it, you first empty the container \eeClass{squares}. Next, depending on the \ eeFunc {BLOCK\_type}, you call the appropriate function. You can start from this example and complete it yourself.

\begin{code}
void setupSquares()
{
	 squares.clear();
	
	 switch(type)
 {
	 	  case BT_SQUARE: makeSquareBlock(); break;
			// create the remaining code your self
 }
}
\end{code}

\subsection{Create and Draw}
The \eeFunc{create} method is easy to complete. Assign values to  \verb|pos| and \verb|Type|. Then call the method \eeFunc{setupSquares}. The second create method is a concept we haven't used before. Here we get a reference to another block as an argument. It is our intention to make a copy of the other block. We do this by copying over each variable from the other block. Because this is the first time you have to create such a method, the complete code is written below. Add this method to your project.

\begin{code}
void create(C block & other)
{
   T.pos  = other.pos ;
   T.type = other.type;
   
   squares.clear();
   FREPA(other.squares)
   {
      squares.New().create(other.squares[i].getPos(), other.type);
   }
}
\end{code}

The \eeFunc{draw} method is easy enough. You should write it yourself. To draw a block on the screen you just need to call the draw method of all the squares in the container.

At this moment it is already possible to run a test. The \eeFunc{create} and \eeFunc{draw} methods already work, so you should check if they do. Run the program with all possible block types and correct your code if something seems off.

\subsection{Move and Rotate}
Now to get your block moving. The \eeFunc{Move} method takes a direction as an argument. You should use a switch to, depending on the direction, adjust the x or y position. (Note: This position is the grid position, not the position on the screen. Therefore it should be increased by one. Using a time delta is not needed here.)

Once you have adjusted the position of the block, you must also pass the direction to all squares, so that they will be moved also. Add the necessary code to the \eeFunc{Move} method.

The rotate method is slightly more difficult. We need to rotate all the squares around the position of the block. This works best if the position of the block is (0,0), because rotating around another point is much more complicated. However, the block is not likely to be at that position. Therefore, we will first subtract the position of the block from the position of the square. After that, we exchange the x and y position of the square, and we add the position of the block back to the position of the square. Finally, we put the new position back into the square. The code looks like this:

\begin{code}
void rotate()
{
 	 FREPA(squares)
  {
 		  VecI2 pos = squares[i].getPos();
 		  pos -= T.pos;
 		  VecI2 newPos(-pos.y, pos.x);
 		  newPos += T.pos;
 		  squares[i].setPos(newPos);
  }
}
\end{code}

Now run your test application again. You should now be able to move and rotate a block. Test with different starting positions to make sure it always works.


\section{Pile}
When a block in Tetris reaches the bottom, the block is no more: all the squares of the block will be added to another container: the `pile'. This `Pile' class also needs a memory container to hold squares. It also needs methods to simplify the interaction with this pile.

\begin{code}
class pile
{
private:
   Memx<square> list;
   
   bool canMove  (C square & s,  DIRECTION dir) C {}   
   void removeRow(int row                     ) C {}
   
public:
   void init() {}
   
   bool collides(C block & b, DIRECTION dir) C {}   
   void add     (C block & b               )   {}
   
   int checkLines()   {}   
   void draw     () C {}
}

pile Pile;
\end{code}

It is also possible to create a test application for this class. In this app, you can manually add a few blocks to the pile and show this pile on the screen. The \eeFunc{collides} method can be tested by creating another block in the application and moving it. This is an example of a possible test:

\begin{code}
if(Kb.bp(KB_DOWN) && !Pile.collides(myBlock, D_DOWN)) {
  myBlock.move(D_DOWN);
}
\end{code}

The \eeFunc{check} method can be tested by calling it on pressing the space bar. Of course you first need to add enough blocks to the pile to form a solid line.

\subsection{Init and Draw}
The methods \eeFunc{init} and \eeFunc{draw} are easy. Write them on your own. The init method must empty the container with the squares. The draw method draws all elements of the container on the screen. If you do not know how to do that, review the chapter on containers.

\subsection{Add}
The \eeFunc{add} method requires a reference to a block as an argument. Its function is to add all squares in that block to the container. Therefore, we provided a method in the class \ eeClass{block} to gain access to all the squares of that block. The class \eeClass{squares} has a method \eeFunc{create} with arguments for a position and a block type. With that information, you write this method like this:

\begin{code}
void add(C block & b)
{
	 C Mems<square> & squares = b.getSquares();
	 REPA(squares)
 {
	 	  list.New().create(squares[i].getPos(), b.getType());
 }
}
\end{code}

Now that you've finished this feature, you can already try out the init, add and draw function in your test program.

\subsection{canMove \& collides}
The method \eeFunc{canMove} is private. It will only be used inside this class, called from the \eeFunc{collides} method. The method should check if a particular square can move in a certain direction (the second argument). 

Note that you should not move the square: only check if the movement is possible. For that reason it is not possible to use the \eeFunc{Move} method of class \eeClass{square}. \textsl{(It is also impossible to do so: precisely to avoid this error, the square is passed as a const reference.)}

The method consists of the following steps:

\begin{itemize}
  \item Create a local \eeClass{VecI2} with the square's position.
	\item Move this position in the required direction. (Take another look at the move method in \eeClass{square} if you have a trouble doing that.)
	\item Iterate over all elements in the container with squares and check whether that square is at the same position. If it is, the method should return false.
	\item If you reach the end of the container without finding a match, the method should return true. Indeed, if there is already a square would be on the new position, the function was abandoned during the audit. Meaning that the square can move in the requested direction.
\end{itemize}

The public method \eeFunc{collides} also has a direction argument, but with a const reference to a block. It is this method that will be used in the Tetris application.

Because a \eeClass{block} has a method to retrieve a reference to the squares in the block, we use that first:

\begin{code}
C Mems<square> & squares = b.getSquares();
\end{code}

The rest of the method will call the method \eeFunc{canMove} for each square. Even when only one square cannot move in the desired direction, the block collides with the pile. In this case the result will be false. Otherwise true is returned. Here's the entire method:

\begin{code}
bool collides(C block & b, DIRECTION dir) C
{
	// Get squares from this block
	C Mems<square> & squares = b.getSquares();
	
	// check all of them
	REPA(squares)
{
		 if(!canMove(squares[i], dir))
 {
				return true;
 }
}
	return false;
}
\end{code}

Now that you have completed these methods, you can reuse your test application to verify this class.

\subsection{checkLines \& removeRow}
Finally, we need to be able to check the pile for full rows and remove them. The method \eeFunc{removeRow} is not that difficult. The argument is the row to remove. So you should once more iterate over all elements in the container. When the \verb|y| position of an element is equal to the argument `row ', the element should be removed. Otherwise, you have to verify that the \verb|y| is greater than the argument row. If it is, you move the square down with the method \eeFunc{move}.

\begin{note}
Because of the way in which the container MemX works, there is no method \eeFunc{remove}. You need to call \eeFunc{removeValid} instead. The differences between the various types of containers will be discussed later. 
\end{note}

The method \eeFunc{checkLines} should check for full rows in the pile. If so, those rows are to be removed. The method must return the number of rows deleted, because that is important for the score.

You can copy the code below. Just make sure you understand all the steps.

\begin{code}
int checkLines()
{
	// Create an array for all rows. Here we provide three additional 
	// rows. At the time that the game is done, there will be
	// blocks in the pile that are positioned higher than the actual 
	playing field
	int squaresInRow[ROWS + 3];
	
	// Set all values ​​to zero.
	REPA(squaresInRow) squaresInRow[i] = 0;
	
	// Iterate the list and raise a row depending on the \verb|y| position 
	// of the square.
	REPA(list)
{
		 int row = list[i].getPos().y;
		 squaresInRow[row]++;
}
	
	// Start with 0 full lines.
	int completedLines = 0;
	
	// Iterate over all rows.
	REPA(squaresInRow)
{
		 // If the number of squares in this row is equal to the 
		 // constant SQUARES_PER_ROW, the row is complete.
		 if(squaresInRow[i] == SQUARES_PER_ROW)
 {
				// We delete this row, but have to account
				// for the fact that multiple rows can be deleted
				// during this loop. If so, the remaining
				// rows are already moved one position down.
				removeRow(i - completedLines);
				
				// Finally, we adjust the counter of deleted rows.
				completedLines++;           
 }
}
	
	// return the number of deleted rows.
	return completedLines;
}
\end{code}

\section{Wall}
The last element we need is the 'Wall', the boundary of the playing field. This is not a real element because we can deduce the boundaries of the playing field from the constants we have declared earlier. Yet it is convenient to use the idea of ​​a wall, so that with each movement we can check for collisions with the wall, just as we did with the pile.

Therefore, this class consists of only two methods. A private method \eeFunc{canMove} checks if a square is allowed to move in a certain direction. A public method \eeFunc{collides} checks whether a block would hit a wall if it would be moved. The class looks like this:

\begin{code}
class wall
{
private:
   bool canMove (C square & s, DIRECTION dir) C {}
   
public:  
   bool collides(C block  & b, DIRECTION dir) C {}
}

wall Wall;
\end{code}

You can surely write this class on your own. The method \eeFunc{collides} is identical to the method in the class \eeClass{pile}. The method \eeFunc{canMove} is almost the same, but now you don't compare the new position with the content of a pile. Instead, the result is \verb|false| when the x or the y position are smaller than 0. Also if the x position is equal to or greater than the value \verb|SQUARES_PER_ROW|, the result is \verb|false|. In all other cases, the result will be \verb|true|.

You can write a new test program or add the new class to the test application for the pile.
\chapter{Interface}

In dit hoofdstuk werken we de achtergrond en het geluid van de game uit. Ook nu maak je weer een testprogramma om deze classes te testen.

\section{Background}
De class voor de achtergrond houden we eenvoudig. We wijzigen nooit iets aan de achtergrond, dus een \eeFunc{create} en \eeFunc{draw} functie volstaan in dit geval. Ook zullen we nooit meer dan \'e\'en object van deze class nodig hebben. Daarom maken we er dadelijk een object van.

Het uiteindelijke speelveld zal er uitzien zoals afbeelding \ref{fig:tetris_background}. Gebruik die als referentie tijdens het testen van deze class.

\begin{figure}[ht]
\centering
\includegraphics[width=0.6\linewidth]{../images/tetris_background.png}
\caption[]{De tetris achtergrond}
\label{fig:tetris_background}
\end{figure}

Om het speelveld te tekenen gebruiken we squares. Dat had ook een afbeelding kunnen zijn, maar zo houden we de interface conform met het spel. Daarnaast heb je een \eeClass{Rect} nodig om een afbeelding te tonen op de plaats waar het volgende blok verschijnt. Tot slot moeten we de positie weten waar we de score en het level tonen.

\begin{code}
class background
{
   Mems<square> squares;
   Rect blockRect;
   Vec2 scorePos;
   Vec2 levelPos;
   
   void create()
   {
   }
   
   void draw()
   {
   }
}

background Background;
\end{code}

In je testprogramma kan je nu al de twee functies van deze class tonen. Daarnaast voeg je de volgende regel toe aan de functie \eeFunc{InitPre}:

\begin{code}
D.mode(WINDOW_WIDTH, WINDOW_HEIGHT);
\end{code}

Deze regel zorgt ervoor dat het window van de applicatie de grootte krijgt die we bij de definitie van de constanten ingaven. Bij de vorige testprogramma's was dat nog niet nodig, maar bij de achtergrond willen we wel zien wat het uiteindelijke resultaat is.

\subsection{Speelveld}

Zoals je weet is het speelveld een grid. Omdat we ook een `muur' rond het speelveld tekenen, beginnen we niet op positie 0 maar op positie -1 met het toevoegen van squares. De squares in het veld zelf geven we het type \eeFunc{BT\_BACKGROUND}. De muren krijgen het type \eeFunc{BT\_WALL}. Daardoor zullen ze in een andere kleur op het scherm gezet worden.

De code om de nodige squares toe te voegen krijg je kado. Eigenlijk is ze niet zo moeilijk, maar wel als je er voor het eerst aan moet beginnen. Lees deze code dus goed en vraag uitleg als je niet begrijpt hoe ze werkt. Daarna voeg je ze toe aan de functie \eeFunc{create}.

\begin{code}
for(int x = -1; x <= SQUARES_PER_ROW; x++)
{
	 for(int y = -1; y <= ROWS; y++)
	 {
			if(x == -1 || y == -1 || x == SQUARES_PER_ROW || y == ROWS)
			{
				 squares.New().create(VecI2(x, y), BT_WALL);
			} else
			{
				 squares.New().create(VecI2(x, y), BT_BACKGROUND);
			}
	 }
}
\end{code}

In de draw functie begin je met het scherm zwart te maken. Daarna toon je alle elementen van de container squares. Test je programma en controleer het resultaat.

\subsection{Next Block}
Op de plaats waar het volgende blok verschijnt plaatsen we een afbeelding. We hebben daar al een constante \eeFunc{WAITPOS} voor gemaakt. Maar dat is een positie in de grid. We moeten die dus steeds vermenigvuldigen met de grootte van een square (de constante \eeFunc{SQUARE\_SIZE}). Daar komt ook nog bij dat we pas mogen tellen vanaf het speelveld, en dat is de constante \eeFunc{GAMEAREA}.

Om te beginnen kunnen we daarom de volgende positie berekenen:

\begin{code}
Vec2 blockPos = GAMEAREA + (WAITPOS * SQUARE_SIZE);
\end{code}

Maar zoals je weet heb je voor een rechthoek een minimum en een maximum nodig. Als minimum trekken we van de gevonden positie twee squares af. Het zou logisch zijn om dan voor de maximum positie twee squares extra te rekenen. Later zal je zien dat dat er visueel minder goed uit ziet, maar dat kan je dan zelf aanpassen. De laatste regel gebruikt de twee nieuwe waarden om de eigenlijke rechthoek in te stellen.

\begin{code}
Vec2 min = blockPos - (2 * SQUARE_SIZE);
Vec2 max = blockPos + (2 * SQUARE_SIZE);
blockRect.set(min, max);
\end{code}

Voeg nu aan de functie \eeFunc{draw} code toe om de afbeelding `tetris\_score' te tekenen via de rechthoek die je net berekende.

\subsection{Tekst}
Je hebt nog twee posities nodig: \eeFunc{scorePos} en \eeFunc{levelPos}. Voor de positie van de score neem je blockPos als uitgangspunt. Van de vertikale waarde trek je vier maal de \eeFunc{SQUARE\_SIZE} af. Voor de level positie vertrek je van de gevonden positie voor de score, maar trekt van de vertikale waarde nog eens \'e\'en square af.

Als je dat in orde hebt kan je te score op het scherm tekenen. Voorlopig zet je daar enkel een tekst. De echte score voegen we later toe. Er is trouwens ook een speciaal font voorzien om in dit spel te gebruiken. Dat is het bestand gui $\Rightarrow$ tetrisFont $\Rightarrow$ tetris Style. Als je niet meer weet hoe je het uitzicht van een tekst moet aanpassen, kan je dat nakijken in hoofstuk \ref{chapter:tekstopmaak}.

Test uiteindelijk weer je programma.

\section{Het geluid}
De volgende class die we maken is \eeClass{soundManager}. Die is behoorlijk eenvoudig en je kan die zelf uitwerken. Er zijn geen create functies nodig, enkel functies die een bepaald geluid afspelen. Je voorziet de volgende functies:

\begin{enumerate}
  \item startMusic: start de soundtrack in een loop.
	\item blip: speelt het `blip' geluid.
	\item score: speelt het `rowdone' geluid.
	\item win: speelt het `won' geluid.
	\item lost: speelt het `lost' geluid.
	\item rotate: speelt het `rotate' geluid.
	\item moveDown: speelt het `down' geluid.
\end{enumerate}

Ook van deze class heb je maar \'e\'en object nodig, dus je maakt onder je class het object \eeClass{SoundManager}. Daarna kan je een eenvoudig testprogramma maken waarin je door op toetsen te drukken deze geluiden afspeelt. Zorg ervoor dat alle geluiden ongeveer even luid klinken. Indien nodig pas je in je class het volume van bepaalde geluiden aan.

\chapter{Application States}
\label{chapter:application_states}

Een application state is een ``status'' van het programma. Wanneer twee delen van een programma nooit gecombineerd worden, dan kan je er twee afzonderlijke application states van maken. 

Wat veel voorkomt is bijvoorbeeld een game lobby en het eigenlijke spel. Je zal nooit de elementen van een game lobby combineren met het spel, dus kan je die volledig scheiden. Ook een login module kan een afzonderlijke application state zijn.

Elk programma heeft al een default application state. Die bestaat uit de functies \texttt{Init()}, \texttt{Shut()}, \texttt{Update()} en \texttt{Draw()}. Wanneer een state actief wordt, dan wordt \texttt{Init()} uitgevoerd. Daarna worden \texttt{Update()} en \texttt{Draw()} afwisselend uitgevoerd, totdat je het programma sluit of overgaat naar een andere state. Op dat moment wordt \texttt{Shut()} uitgevoerd.

\section{Intro}
Voor elke state maak je een afzonderlijk bestand. Bijvoorbeeld voor een intro state:

\begin{code}
bool InitIntro() {return true;}

void ShutIntro() {}

bool UpdateIntro()
{
   if(Time.stateTime()>3 || Kb.bp(KB_ESC)) {
      StateMenu.set(1.0);                    
   }
   return true;
}

void DrawIntro()
{
   D.clear(BLACK);
   D.text (0, 0, "Intro");
}

State StateIntro(UpdateIntro, DrawIntro, InitIntro, ShutIntro);
\end{code}

Je ziet dat dit veel lijkt op de standaard states in je programma. We voegen gewoon het woord Intro toe aan Init, Shut, Update en Draw. Dit houdt het overzichtelijk.

De eigenlijke state zit in de laatste regel:

\begin{code}
State StateIntro(UpdateIntro, DrawIntro, InitIntro, ShutIntro);
\end{code}

Daar geef je aan dat er een nieuwe gamestate is (\eeFunc{StateIntro}) die de typische functies voor een programma bevat. Een \eeFunc{InitPre()} functie kan je niet toelaten, die dient enkel voor de echte start van het programma.

Kijk ook even naar de constructor van \eeFunc{State}:

\begin{code}
State(Bool (*update)(), void (*draw)(), Bool (*init)()=NULL, void (*shut)()=NULL); 
\end{code}

Komt de asterisk (*) je bekend voor? Inderdaad, we hebben met pointers te maken. Pointers naar functies in dit geval. De constructor verwacht dat we aangeven waar de functies voor deze state staan. We verwijzen dus naar de functies die we net gemaakt hebben: \eeFunc{UpdateIntro()} en \eeFunc{DrawIntro()}. Je zegt eigenlijk ``zolang deze state actief is, voer je \eeFunc{UpdateIntro()} uit in plaats van de gewone \eeFunc{Update()} functie.

De volgende twee argumenten, voor de functies \eeFunc{InitIntro()} en \eeFunc{ShutIntro()} zijn optioneel. Je mag ze weglaten als er niets bijzonders moet gebeuren op dat moment.

\begin{note}
Indien een functie argument eindigt met \eeFunc{=NULL}, dan mag je het weglaten.
\end{note} 

\section{Menu}
De code hierboven bevat ook een verwijzing naar StateMenu:

\begin{code}
   if(StateActive.time()>3 || Kb.bp(KB_ESC)) {
      StateMenu.set(1.0);                    
   }
\end{code}

Met andere woorden: we wachten tot de huidige state 3 seconden actief is, of totdat de gebruiker op escape drukt. Dan zetten we een nieuwe application state actief met een crossfade van 1 seconde.

Deze nieuwe state zou er zo kunnen uitzien:
\begin{code}
bool InitMenu() {return true;}
void ShutMenu() {}

bool UpdateMenu()
{
   if(Kb.bp(KB_ESC))return false;
   if(Kb.bp(KB_ENTER))StateGame.set(0.5);
   return true;
}

void DrawMenu()
{
   D.clear(GREY);
   D.text (0,  0  , "Menu");
   D.text (0, -0.3, "Press Enter to start the game");
   D.text (0, -0.5, "Press Escape to exit");
}

State StateMenu(UpdateMenu, DrawMenu, InitMenu, ShutMenu);
\end{code}

Deze state lijkt sterk op de vorige. Maar dit maal kunnen we met Enter naar de game zelf. En dat is dan ook weer een nieuwe application state: \eeFunc{StateGame}.

\section{Game}
Deze code kan je voor \eeFunc{StateGame} gebruiken. Maak ook nu weer een afzonderlijk bestand.
\begin{code}
bool InitGame() {return true;}
void ShutGame() {}

bool UpdateGame()
{
   if(Kb.bp(KB_ESC))StateMenu.set(1.0);
   return true;
}

void DrawGame()
{
   D.clear(TURQ);
   D.text (0, 0, "Game");
}

State StateGame(UpdateGame, DrawGame, InitGame, ShutGame);
\end{code}

Door tijdens de game op escape te drukken, schakelen we terug naar \eeFunc{StateMenu}. In deze state ga je bij een echte game natuurlijk nog heel veel code moeten toevoegen.

\section{Default State}
Dan rest ons nog het starten van het programma. We hebben nu alle nodige states, maar \eeFunc{StateIntro} moet nog actief worden. Dit gebeurt door in de \eeFunc{Init()} functie van het programma dadelijk door te schakelen naar \eeFunc{StateIntro}. De functies \eeFunc{Update()} en \eeFunc{Draw()} worden in dit programma dus niet gebruikt.

\begin{code}
void InitPre()
{
   EE_INIT();
}

bool Init()
{
   StateIntro.set();
   return true;
}

void Shut() {}
bool Update() {return false;} // unused
void Draw  () {             } // unused
\end{code}

\begin{exercise}
Gebruik de code van dit hoofdstuk om een programma te maken dat wisselt tussen de voorziene application states. Elke state plaats je in een afzonderlijk bestand.
\end{exercise}

\chapter{GameLogic}
Er rest ons nog \'e\'en class, maar dan wel de belangrijkste. We hebben alle onderdelen voor het spel klaar, maar die moeten nu samengebracht worden zodat het spel zich gedraagt zoals we verwachten. De class \eeClass{gameLogic} dient precies daar voor. Het framework ziet er zo uit:

\begin{code}
class gameLogic
{
private:
   // blocks in the game
   block currentBlock;
   block nextBlock   ;

   float forceDownCounter = 0         ;
   float slideCounter     = SLIDE_TIME;
   
   // to move a block completely down
   bool  toBottom      = false;
   float toBottomTimer = 0.05 ; 
   
   
   bool canRotate            (C block & b               ) C {}
	 bool canMove              (C block & b, DIRECTION dir) C {}
   void handleBottomCollision()   {}   
   void changeFocusBlock     ()   {}    
   void checkLoss            () C {}
   void handleInput          ()   {}
    
public:
   void create()   {}
   void update()   {}
	 void draw  () C {}
} 
gameLogic GameLogic;
\end{code}

Laten we eerst even de variabelen bekijken:

\begin{description}
	\item[currentBlock] Dit is het blok dat je beweegt tijdens het spel.
	\item[nextBlock] Dit is het volgende blok, dat klaar staat aan de rechterzijde.
	\item[forceDownCounter] Wanneer we het blok niet zelf naar beneden bewegen, dan moet dat na een korte tijd vanzelf gebeuren. Met deze timer regelen we hoe lang dat duurt.
	\item[slideCounter] In tetris kan je wanneer een blok de pile raakt, nog heel even het blok opzij plaatsen. Daar hebben we dus ook een timer voor nodig.
	\item[toBottom] Wanneer we op de spatiebalk drukken moet het blok helemaal naar beneden bewegen. Maar je moet het wel zien bewegen, dus je mag het niet zomaar in \'e\'en keer beneden plaatsen. Met deze bool houden we bij of het huidige blok snel naar beneden moet.
	\item[toBottomTimer] En die beweging heeft ook weer een timer nodig voor elke stap.
\end{description}

\section{De eenvoudige functies}
\subsection{CheckLoss}
Een functie die je zonder problemen kan uitwerken is \eeFunc{checkLoss}. Deze functie moet controleren of de speler het spel verloren heeft. Wanneer gebeurt dat? Wanneer in tetris een nieuw blok bovenaan verschijnt en dat blok kan niet naar beneden verplaatst worden, dan heeft de speler verloren. En wanneer kan een blok niet naar beneden verplaatst worden? Wanneer het zou botsen met de \eeClass{Pile}. 

Pile heeft al een functie \eeFunc{collides}. Aan die functie kan je dus het object \eeClass{currentBlock} doorgeven en de gewenste richting. Geeft de functie false als resultaat, dan kan je de \eeFunc{Score.gameIsLost()} uitvoeren.

\subsection{Create}
De create functie dient om bij de start van een spel alle variabelen een beginwaarde te geven. Je voegt als eerste regel dit toe:

\begin{code}
Random.randomize();
\end{code}

De bedoeling van deze regel is het volgende. Computers hebben een groot probleem met willekeurige getallen. Dat concept past eigenlijk niet binnen een computerlogica. We lossen dat op met het \eeFunc{Random} object, maar dat object heeft eigenlijk intern een lijstje met getallen die het \'e\'en voor \'e\'en af gaat. Elke keer je om een random getal vraagt, krijg je gewoon het volgende getal uit de lijst. Zou je dus je programma elke keer laten beginnen aan het begin van dat lijstje, dan krijg je steeds dezelfde `willekeurige' getallen. Dat maakt je spel na verloop van tijd wel erg voorspelbaar. De functie \eeFunc{randomize} zorgt er voor dat je naar een willekeurige plaats in de lijst springt. Zo krijg je steeds een andere reeks getallen.

\begin{note}
Moest je ooit software ontwikkelen voor een casino, dan zal je nooit de standaard random functies van de programmeertaal mogen gebruiken. Het is namelijk niet zo moeilijk om een programma te schrijven dat, na ingave van de eerste drie resultaten, opzoekt waar de lijst startte. Op dat moment kan je al behoorlijk goed voorspellen wat het volgende getal in de lijst zal zijn. Aangezien een casino toch ook winst wil maken, gebruikt men voor dat soort software een meer complexe random library.
\end{note}

De volgende statements zorgen voor twee willekeurige blokken:

\begin{code}
currentBlock.create(STARTPOS, (BLOCK_TYPE)Random(BT_NUM));
nextBlock   .create(WAITPOS , (BLOCK_TYPE)Random(BT_NUM));
\end{code}

Je ziet dat we de constanten \verb|STARTPOS| en \verb|WAITPOS| gebruiken om de posities in te stellen. Het tweede argument is het blok type. We willen telkens een willekeurig blok, dus we gebruiken de random functie. Het argument van \eeFunc{Random} bepaalt het hoogste getal. Maar die waarde is niet inclusief: dat wil zeggen dat je, als je bijvoorbeeld \eeFunc{Random(3)} schrijft, het resultaat 0, 1 of 2 kan zijn. Niet 3 dus. Waarom schrijven we hier dan \verb|BT_NUM|? Daarvoor moet je even terug in het bestand `enumerations' kijken. \verb|BT_NUM| is het achtste element in de lijst. Het eerste element is gelijk aan 0, dus het achtste element is gelijk aan 7. De \eeFunc{Random} functie zal hier dus een getal van 0 tot en met 6 teruggeven. En omdat een enumeratie niets anders is dan een naam voor een getal, kunnen we dat getal eenvoudig terug omzetten naar een \verb|BLOCK_TYPE|. Want dat is het type dat de create functie verwacht.

Na deze statements moeten we \eeFunc{forceDownCounter} de waarde 0 geven, en slideCounter gelijk stellen aan \verb|SLIDE_TIME|. Die statements kan je zelf wel verzinnen. Tot slot voer je ook de \eeFunc{init} functies van \eeClass{Pile} en \eeClass{Score} uit.

\subsection{Draw}
De draw functie van deze class moet drie elementen op het scherm tonen: het huidige blok, het blok in de wachtpositie en de pile. Voeg de statements toe om dat te doen.

\section{Iets moeilijker}

\subsection{Can Rotate}
We hebben al functies om bij een verplaatsing collisions met de pile of het speelveld te controleren. Nu moeten we ook controleren of het mogelijk is om een blok te roteren. Daarvoor dient de functie \eeFunc{canRotate}.

In deze functie maken we eerst een nieuw blok. We willen namelijk het blok dat we als functie argument binnen krijgen niet roteren, maar enkel controleren of het mogelijk is. Bij dat nieuwe blok voeren we de create functie uit, met als argument het bestaande blok \verb|b|. Daarna roteren we het nieuwe blok.

Nu kunnen we controleren of dit nieuwe blok botst met \eeClass{Wall} of \eeClass{Pile}. Het tweede argument is dan de richting \verb|D_NONE|, want we willen geen verplaatsing controleren. Wanneer een van die functies aangeeft dat er een collision is, dan is het functieresultaat \verb|false|. In het andere geval wordt het \verb|true|.

\subsection{Can Move}
De functie \eeFunc{canMove} zorgt er voor dat we een verplaatsing in \'e\'en keer kunnen controleren. We moeten namelijk zowel collisions met de wall als met de pile in de gaten houden. Als een van deze functies \verb|false| als resultaat heeft, dan is het functieresultaat ook \verb|false|. Is dat niet zo, dan is het resultaat \verb|true|. De argumenten van de functies kan je gewoon doorgeven aan de collide functies van \eeClass{Wall} en \eeClass{Pile}.

\subsection{Change Focus Block}
Op het moment dat een blok beneden is, moet je het toevoegen aan de pile en bovenaan een nieuw blok tonen. Het type van de blok moet gelijk zijn aan het blok in de wachtpositie. Daarna moet je ook nog een beslissen wat nu het volgende blok zal worden.

Je kan deze functie in drie stappen uitwerken:

\begin{enumerate}
	\item Voeg het huidige blok toe aan de pile.
	\item Voer opnieuw de create functie van het huidige blok uit. Als eerste argument gebruik je de constante \verb|STARTPOS|. Het tweede argument is het type van het blok op de wachtpositie. (Zoek in de class \eeClass{block} naar een functie die je dat type geeft.)
	\item Voer nu ook opnieuw de create functie van `nextBlock' uit. Die is gelijk aan het statement dat je eerder in de create functie van deze class schreef.
\end{enumerate}

\subsection{Handle Bottom Collision}
Deze functie beschrijft wat er moet gebeuren als een blok de pile raakt. Ook dat zijn vier eenvoudige statements:

\begin{enumerate}
	\item Voer de functie \eeFunc{changeFocusBlock} uit.
	\item Controleer of er rijen verwijderd kunnen worden uit de Pile.
	\item Geef het aantal verwijderde lijnen (het resultaat van de vorige regel) door aan het \eeClass{Score} object.
	\item Controleer via de functie \eeFunc{checkLoss} of het spel gedaan is.
\end{enumerate}

\subsection{Handle Input}
We hebben ook een functie nodig die reageert wanneer we een toets indrukken. Dat is de functie \eeFunc{handleInput}. Elke toets die we kunnen indrukken tijdens het spel moet hier behandeld worden. Zo moet, wanneer je de pijltjestoets naar beneden indrukt, eerst gecontroleerd worden of een verplaatsing naar beneden wel mogelijk is. Als dat zo is, dan verplaats je het blok naar beneden en laat je een geluidje horen. Dat kan zo:

\begin{code}
if(Kb.bp(KB_DOWN))
{
	 if(canMove(currentBlock, D_DOWN))
	 {
			currentBlock.move(D_DOWN);
			SoundManager.blip();
	 }
}
\end{code}

De code om een blok naar links of rechts te verplaatsen is gelijkaardig. Die werk je dus weer zelf uit.

De `UP' toets gebruik je in tetris om een blok te roteren. Voeg dus ook code toe om te controleren of deze toets wordt ingedrukt. Dit maal moet je enkel de functie \eeFunc{canRotate} uitvoeren met het huidige blok. Als het resultaat van die functie \verb|true| is, dan roteer je het blok en laat je weer een geluidje horen.

En als laatste is er de spatiebalk. Bij het indrukken van de spatiebalk moet een blok helemaal tot beneden bewegen. Om dat te doen geven we de variabele `toBottom' de waarde \verb|true| en de variabele `toBottomTimer' de waarde 0. En ook hier speelt er weer een geluid.

\section{Update}
En dan komen we bij de laatste functie, die het centrum van het spel vormt: de functie \eeFunc{update}. Die functie voert achtereenvolgens verschillende controles uit. 

\subsection{Force Down}
Eerst kijken we of het tijd is om een blok naar beneden te verplaatsen. Daarvoor moeten we de `forceDownCounter' verhogen. Als die hoger is dan de huidige game speed, dan moet het blok een stap naar beneden:

\begin{code}
forceDownCounter += Time.d();
if(forceDownCounter > Score.getSpeed())
{
	 if(canMove(currentBlock, D_DOWN))
	 {
			currentBlock.move(D_DOWN);
			forceDownCounter = 0;
	 }
}
\end{code}

\subsection{Slide Counter}
De slide counter dient om het blok nog even opzij te kunnen bewegen wanneer het de pile raakt. Daarom moeten we eerst weten of het blok de pile raakt en dat is het geval als het niet meer naar beneden kan. In dat geval zullen we de waarde van `slideCounter' verlagen. Als de slideCounter nul wordt, dan voeren we de functie \eeFunc{handleBottomCollision} uit.

\begin{code}
if(!canMove(currentBlock, D_DOWN))
{
	 slideCounter -= Time.d();
} else
{
	 slideCounter = SLIDE_TIME;
}

if(slideCounter <= 0)
{
	 slideCounter = SLIDE_TIME;
	 handleBottomCollision();
}
\end{code}

\subsection{To Bottom}
De bool `toBottom' is \verb|true| wanneer de speler op de spatiebalk drukte. We verplaatsen het blok dan snel naar beneden, maar dat moet nog steeds stap voor stap gebeuren om zichtbaar te zijn. We gebruiken dus een counter met een kleine waarde waar we telkens weer de tijdsdelta van aftrekken. Elke keer dat de counter nul wordt verplaatsen we het blok een positie naar beneden. Als dat niet meer mogelijk is, dan schiet de bovenstaande code (Slide Counter) in actie.

Enkel wanneer het blok niet naar beneden glijdt, controleren we de input van de speler.

\begin{code}
if(toBottom)
{
	 toBottomTimer -= Time.d();
	 if(toBottomTimer < 0)
	 {
			toBottomTimer = 0.05;
			if(canMove(currentBlock, D_DOWN))
			{
				 currentBlock.move(D_DOWN);

			} else
			{
				 toBottom = false;
			}
	 }
} else
{
	 handleInput();
}
\end{code}

Daarmee is ook deze functie af. Het enige wat je nu nog moet doen is de \eeFunc{create}, \eeFunc{update} en \eeFunc{draw} toevoegen aan de Game State. En daarna natuurlijk alle fouten oplossen tot je programma werkt zoals het hoort.

\section{Nabespreking}
Je hebt nu een volledig project uitgewerkt. Hopelijk zal je onthouden hoe belangrijk het is om alles in classes onder te brengen. Ook het gebruik van testprogramma's is erg belangrijk om een groot project goed uit te werken.

Maar natuurlijk komt deze manier van werken niet helemaal overeen met de realiteit. De auteur van deze cursus wist precies wat er moest gebeuren en in welke volgorde je dat het best kon aanpakken, nog voor je aan deze oefening begon. Als je zelf aan een project begint dan is dat wel anders. Het is heel gewoon dat je de classes die je vooraf maakt meermaals moet aanpassen. Dikwijls blijken er toch nog functies te ontbreken, of schrijf je functies die je uiteindelijk niet nodig blijkt te hebben. Dat is, zeker voor een beginnende programmeur, heel normaal.

Enkel door ervaring leer je steeds beter inschatten welke functies een class nodig zal hebben. En zelfs dan kan je dat niet altijd exact voorspellen.

\part{Gui}
\chapter{Introduction}
In the previous chapters you have seen everything you need to create a simple 2D game. But how do you put a large project together in an orderly way? There is really no simple answer to that. You learn by practice, and not everyone agrees on the best method. Still, there are some rules that may make it easier for sure. And when you work in a group, the lead programmer will usually impose some rules that everyone has to follow. These are not necessarily good or bad rules, but they work as long as everyone follows them.

In this part of the course you will create a clone of the famous Tetris game. You will learn to develop a project step by step  without losing sight of the whole.

\begin{note}
Tetris consists of blocks which in turn consist of squares. When we talk about blocks in this course, we mean the entire block, not the squares it is made of. When a square is mentioned, we're discussing the squares that make up a block.
\end{note}

\section{Setup}
Open the `Tetris\_start' project. In it, you will find the graphics, sounds and fonts that we will use. A blank app `Tetris' is also provided, but we are not going to use it just yet.

\begin{enumerate}
	\item Create a library at the highest level of the explorer. You do this by right clicking and choosing `new library'. Name this library `Tetris parts'. A library is a green folder. The code in a library can be used from any application within your project, just like the library `Esenthel Engine' which is always present.
	\item Create a new application (blue folder). Call it `square tester'.
	\item in the application `square tester', create a code file `main'. 
	\item In the library `Tetris parts', create a new folder (yellow) called `definitions'.
	\item Mark `square tester' as the active application.
\end{enumerate}
	
Copy the code in `Tetris/initState' to `square tester/main'. Remove this line: 
	
\begin{code}
D.full(true);
\end{code}

This makes it easier to terminate your application when something goes wrong.	

\section{Constants}
In the previous chapter you learned about constants. It is a good idea to create some important constants before you start on the actual code. You can use them anywhere in your code and easily adjust their values if necessary. Create a new code file `constants' in the folder `Tetris parts/definitions'.

To be able to change the name of the app later on, create a constant \eeClass{Str} with a provisional name.

\begin{code}
C Str APP_NAME = "Tetris";
\end{code}

The size of the standard application window is not ideal for this game. A size in pixels is required for this. Declare a constant \eeClass{int} for this purpose.

\begin{code}
// The window size on the screen, in pixels
C int WINDOW_WIDTH  = 900;
C int WINDOW_HEIGHT = 800;
\end{code}

Tetris consists of rows and columns. we also define these:

\begin{code}
// This impacts the playing field
C int SQUARES_PER_ROW = 10;
C int ROWS            = 15;
\end{code}

It is also possible to add a few constants for the scoring system. There is a fixed number of levels, and we know how much points will be rewarded for a line and a level.

\begin{code}
// The scoring system uses these
C int POINTS_PER_LINE  =  525;
C int POINTS_PER_LEVEL = 6300;
C int NUM_LEVELS       =    5;
\end{code}

The speed of the game goes up with each level. This change can also be defined as a constant.

\begin{code}
// The speed will increase every level
C float INITIAL_SPEED = 1.0;
C float SPEED_CHANGE  = 0.1;
\end{code}

When playing tetris, there is a short period after moving a block down in which you are able to move it sidewards. This period has to be defined.

\begin{code}
// The time a block can be slided to the side
// When it hits bottom
C float SLIDE_TIME = 0.25;
\end{code}

We also have to determine the size of the playing field. This is done by defining the position of the lower left corner, and defining the size of the rectangle containing the playing field.

\begin{code}
// The area reserved for the playing field 
C Vec2 GAMEAREA     (-0.8, -0.8);
C Vec2 GAMEAREA_SIZE( 1.0,  1.4);
\end{code}

A new block will always appear at the top of the screen. This position can be deducted from information we already have:  SQUARES\_PER\_ROW and ROWS. In addition, there is a waiting position, top right of the playing field. You might notice we are not using \eeClass{Vec2}, but \eeClass{VecI2}. This is a vector which fits only integers. We do not want Tetris blocks midway between two positions, so there is no need for floats here.

\begin{code}
// Position for the current and next block
C VecI2 STARTPOS(SQUARES_PER_ROW / 2, ROWS - 1);
C VecI2 WAITPOS (SQUARES_PER_ROW + 4, ROWS - 3);
\end{code}

Now it is possible to calculate the size of a square with the information we already have. This has the advantage that we can modify the foregoing constants later, the size of a square being automatically adjusted.

\begin{code}
// The size of a square
C float SQUARE_SIZE = GAMEAREA_SIZE.x / SQUARES_PER_ROW;
\end{code}

\begin{note}
Of course it will rarely happen that you precisely know which constants are needed when you're just starting out on a project. In practice you will usually add a lot of constants while you are working on your project.
\end{note}

\section{Enums}
Create the code file `enumerations' in  `Tetris parts/definitions'. Add two enumerations that will be useful in your project. Firstly, there is the block type. Tetris has square blocks, T-blocks and so on. A list might look like this:

\begin{code}
enum BLOCK_TYPE
{
   BT_SQUARE     ,
   BT_T          ,
   BT_L          ,
   BT_BACKWARDS_L,
   BT_STRAIGHT   ,
   BT_S          ,
   BT_BACKWARDS_S,
   BT_NUM        , // number of block types used in the game
   BT_BACKGROUND , // special case, only for background
   BT_WALL       ,      
}
\end{code}

The last three values ​​deserve extra attention. The value `BT\_NUM' is useful because the number which represents that value is equal to the highest value + 1. This makes it easy to use a random function. Since the result of a random function does not include the maximum value, we will be able to use this kind of code later on in our application:

\begin{code}
blockType type = Random(BT_NUM);
\end{code}

So why is BT\_NUM not the last value in the list? Well, the last two values ​​are special cases. There will be special squares to draw the background and the borders of the game. And the color of such a square is determined by the block type. Because we do not want to use those types in the actual game, BT \_NUM is but before these values.

A second enumeration is used to determine the possible directions in which a block can move. A block can not upwards, only to the left, the right, or down. Blocks which are already down, do not have a direction anymore.

\begin{code}
enum DIRECTION
{
   D_LEFT ,
   D_RIGHT,
   D_DOWN ,
   D_NONE ,
}
\end{code}

\chapter{Gui Window}
Alhoewel je gui elementen ook direct op het scherm kan zetten, plaats je ze meestal in een window. Zelfs al wil je dat window niet tonen (je kan het later onzichtbaar maken) is dit de meest eenvoudige manier om elementen bij mekaar te houden. Wil je bijvoorbeeld verschillende elementen samen tonen of verbergen, dan plaats je ze best samen in een window.

\section{Definitie}
Kijk eens naar de definitie van een \eeClass{Window}:

\begin{code}
const_mem_addr struct Window : GuiObj // Gui Window !! must be stored in constant memory address !!
{
   Byte      flag       ,
   ...
\end{code}

De commentaar en ook het eerste woord \eeFunc{const\_mem\_addr} laten je weten dat een window een vast geheugen adres nodig heeft. Wat betekent dat? Wel, je mag een window nooit opslaan in een gewone container. De elementen van een \eeClass{Memc} kunnen wel eens naar een andere locatie in het geheugen verplaatst worden zonder dat je dat zelf merkt. Voor GUI elementen mag zoiets niet gebeuren.

Nu is dat meestal geen probleem. Wanneer je een GUI ontwerpt dan wil je van elk window en de elementen daarin meestal maar \'e\'en object maken en dat doe je dan onder de declaratie van de class die je daarvoor maakt. (In het vorige hoofdstuk was dat \verb|loginWindow LoginWindow|). Op die manier heb je automatisch een vast geheugen adres. Heb je toch een container met meerdere objecten van deze class nodig? Gebruik dan de container \eeClass{Memx} of \eeClass{Meml}.

Wat ook opvalt is het laatste woord: \eeClass{GuiObj}. Dat betekent dat een Window gebaseerd is op de class \eeClass{GuiObj}. Alle eigenschappen van deze class zijn dus ook beschikbaar in \eeClass{Window}. We zeggen daarom dat \eeClass{GuiObj} de base class van \eeClass{Window} is.

\section{Class Methods}
Als je verder in de definitie van de class \eeClass{Window} kijkt, dan zie je heel wat lidfuncties. En daar komen ook de lidfuncties van \eeClass{GuiObj} nog bij. Zolang je je gui ontwerpt met de gui editor, zal je de meeste van deze functies niet zo vaak nodig hebben. We bespreken hier enkele functies die toch handig kunnen zijn:

\begin{description}
	\item[\eeFunc{setTitle(C Str \&title)}] Via deze functie kan je de titel van het window wijzigen.
	\item[\eeFunc{fadeIn()}] Samen met de functie \eeFunc{fadeOut()} bepaal je de zichtbaarheid van een window.
	\item[\eeFunc{showing()}] Samen met de functie \eeFunc{hiding()} controleer je de zichtbaarheid van een window. Je kan bijvoorbeeld de pijltjestoetsen enkel gebruiken om je avatar te bewegen wanneer een window niet zichtbaar is.
	\item[\eeFunc{button[3]}] is geen functie. Een window bevat altijd die buttons met een specifiek doel: minimize, maximize en close. De typische buttons die je vaak in de rechterbovenhoek ziet. Als je deze buttons wil tonen, dan kan je na het laden van je window de \eeFunc{show()} functie van de gewenste buttons uitvoeren.
\end{description}

\begin{exercise}
Maak in programma waarin je een window toont zolang de spatiebalk ingedrukt is. Wanneer het window zichtbaar is, dan stel je de titel van het window gelijk aan de tijd dat je programma loopt. (Gebruik de functie \eeFunc{Time.appTime()}.)
\end{exercise}

Er is nog een functie die bijzondere aandacht verdient: de functie \eeFunc{pos(C Vec2 \&pos)} van de base class \eeClass{GuiObj}. Deze functie komt vaak van pas om de positie van je window te corrigeren. Je kan in de gui editor wel de positie van een window bepalen, maar dat is vaak niet voldoende. Je wil bijvoorbeeld een window aan de linkerkant van het scherm, maar niet ieder scherm heeft dezelfde aspect ratio. Daarom bereken je best de positie voordat je een window toont. Voor de linkerbovenhoek zou dat er zo uit zien:

\begin{code}
void show() {
   window.pos(Vec2(-D.w(), D.h());
	 window.fadeIn();
}
\end{code}

Andere posities vragen dikwijls iets meer werk. Dan moet je immers ook de afmetingen van het window in rekening brengen. Gelukkig kan dat eenvoudig via de lidfunctie \eeFunc{rect()}. Die geeft je informatie over de grootte van je window, dat natuurlijk een rechthoek is. In dit voorbeeld zie je hoe je een window aan de linkeronderhoek van het scherm plaatst. 

\begin{code}
void show() {
   Vec2 pos;
	 pos.x = -D.w();
	 pos.y = -D.h() + window.rect().h();
   window.pos(pos);
	 window.fadeIn();
}
\end{code}

\begin{exercise}
Plaats een window aan de rechterkant van het scherm, in het midden. Maak ook een tweede functie \eeFunc{showCentered()} die het window in het midden van het scherm toont.
\end{exercise}

\section{Dialogs}
Esenthel bevat ook enkele windows met en meer specifiek doel. Een daarvan is de class \eeClass{Dialog}. Als je naar de declaratie van die class kijkt, dan zie je dat een \eeClass{Dialog} de class \eeClass{Window} als basis heeft. Dat betekent dat ook de functies van \eeClass{GuiObj} beschikbaar zijn.

Een dialog zal je niet een de gui editor ontwerpen. De lidfunctie \eeFunc{autoSize()} wordt gebruikt om na het instellen van de tekst en de buttons alle elementen voldoende plaats te geven. Dat betekent dat je in dit geval geen \eeClass{GuiObjs} en ook geen pointers gebruikt. Hier een voorbeeld:

\begin{code}
class confirmExit {
private:
   Dialog dialog;
   
   static void cancelFunc(confirmExit & obj)
   {
      obj.dialog.fadeOut();
   }
   
   static void proceedFunc(confirmExit & obj)
   {
      Exit();
   }
      
public:
   void create() {
      Mems<Str> buttonTexts;
      buttonTexts.New() = "cancel";
      buttonTexts.New() = "proceed";
      dialog.create("Exit?", "Ben je zeker dat je dit programma wil sluiten?", buttonTexts);
      dialog.autoSize();
      dialog.hide();
      
      dialog.buttons[0].func(cancelFunc, T);
      dialog.buttons[1].func(proceedFunc, T);
      
      Gui += dialog;
   }
   
   void show() {
      dialog.fadeIn();
   }
}

confirmExit ConfirmExit;
\end{code}

\begin{exercise}
Voeg deze class toe aan een programma en toon de dialog wanneer de gebruikt op escape drukt, in plaats van daar dadelijk het programma af te sluiten.
\end{exercise}


\chapter{Gui Buttons}

De meest gebruikte functie van de class button is \eeFunc{func()}. Daarmee koppel je een callback functie aan de button. Die functie wordt dan uitgevoerd wanneer je op de button klikt. In de inleiding werd dit al besproken, maar denk er aan dat dit een statische functie moet zijn, die dus los staat van het eigenlijke object. Alhoewel het niet strikt noodzakelijk is, geef je dus best altijd een referentie naar het huidige object door aan de button. 

Enkele andere handige functies zijn:

\begin{description}
	\item[\eeFunc{enabled(bool enabled)}] laat je toe om op bepaalde momenten een button non-actief te maken.
	\item[\eeFunc{setText(C Str \&text)}] wijzigt de tekst van een button.
	\item[\eeFunc{bool sound}] door deze eigenschap op `true' te zetten, wordt er een geluid afgespeeld wanneer je op deze button drukt. Het geluid zelf kan je instellen via \eeClass{Gui.click\_sound\_id}. Standaard is dat geluid leeg, maar je kan er eender welk geluidsbestand aan toewijzen. (Let op, dit is geen functie maar een property!)
\end{description}

\begin{exercise}
Pas de vorige oefening aan, zodat de buttons in je dialog een geluid afspelen.
\end{exercise}

\section{Toggle Buttons}
Een button kan ook gebruikt worden als toggle button. In dat geval gedraagt die zich net iets anders. Een enkele toggle button kan je als alternatief voor een checkbox gebruiken. Bijvoorbeeld om bepaalde elementen op het scherm te tonen. Hieronder zie je een voorbeeld van een toggle button die een crosshair aan of uit zet.

\begin{code}
class crossHair
{
private:
   Button button;
   bool on = false;
   
   static void buttonFunc(crossHair & obj)
   {
      obj.on = obj.button();
   }
   
public:
   void create()
   {
      button.create(Rect(-D.w() + 0.1, D.h() - 0.2, -D.w() + 0.6, D.h() - 0.1), "cross on/off");
      button.mode = BUTTON_TOGGLE;
      button.set(false);
      button.func(buttonFunc, T);
      Gui += button;
   }
   
   void draw()
   {
      if(!on) return;
      Circle(0.1).draw(RED, false);
      Edge2(0, -0.12, 0, 0.12).draw(RED);
      Edge2(-0.12, 0, 0.12, 0).draw(RED);
   }
}

crossHair CrossHair;
\end{code}

Enkele zaken vallen wellicht op in deze class. Ten eerste wordt de gui editor niet gebruikt. Om slechts \'e\'en button op het scherm te tonen zou dat wat omslachtig zin. Daarom is de Button geen pointer, maar een echt object. We dienen dan wel zelf de functie \eeFunc{create} uit te voeren, met als argument een rechthoek en een tekst.

Daarna wordt de mode aangepast, zodat we een toggle button krijgen. En via de set functie zetten we de huidige stand op `uit'. De statische functie \eeFunc{buttonFunc} geeft de stand van de button door aan de bool `on'. Via haakjes na de naam van de button kom je dus zijn huidige stand te weten. Dit is enkel zinvol bij toggle buttons. Een gewone button heeft immers geen stand.

\begin{exercise}
Voeg deze code toe aan een programma. Kies zelf een geschikte positie voor de button. Wanneer je tijdens de create functie de button zou inschakelen met \eeFunc{button.set(true)} dan zal de crosshair niet toch niet getoond worden. Zoek uit hoe dat komt en hoe je dat kan oplossen.
\end{exercise}
\chapter{CheckBox}
Een checkbox verschilt niet zoveel van een button in toggle modus. Welk van de twee je gebruikt, maakt enkel visueel een verschil. Net zoals alle gui classes is ook \eeClass{CheckBox} afgeleid van \eeClass{GuiObj}, dus je kan ook alle functies van die base class gebruiken.

\begin{exercise}
Maak een window met enkele checkboxes. De eerste checkbox verbind je aan een functie die een bool `showFPS' controleert. Wanneer die \eeFunc{true} is, dan toon je via de huidige fps op het scherm met behulp van de functie \eeFunc{Time.fps()}. Gebruik de andere checkboxes om enkele windows uit de vorige oefeningen te tonen en verbergen.
\end{exercise}


\chapter{Slider}

Een slider heeft een waarde tussen 0 en 1, afhankelijk van zijn positie. Je kan die waarde opvragen via de operator \eeFunc{()}. Voor een gewone slider wordt dat:

\begin{code}
float value = mySlider();
\end{code}

Maar wanneer je slider een pointer is, dan moet je wel aangeven dat je niet de pointer maar het object zelf wil aanspreken:

\begin{code}
float value = (*mySlider)();
\end{code}

Doe je dat dan vanuit een static functie die je koppelt aan de slider, dan wordt die variabele automatisch aangepast wanneer je de slider beweegt. Zo kan je het volgende schrijven:

\begin{code}
class sliderWindow
{
private:
   GuiObjs objs;
	 Window * window;
	 Slider * speedSlider;   
	 float currentSpeed = 1;	
	
   static void speedSliderFunction(sliderWindow & obj) {
		  currentSpeed = (*obj.speedSlider)() * 5;
	 }
	
public:
   void create()
   {
      objs.load( --- Drop Gui Object here --- );
      window = objs.findWindow("window");
			speedSlider = objs.findSlider("speedSlider");

			speedSlider.func(speedSliderFunction, T);
      Gui += objs;
   }
}
sliderWindow SliderWindow;
\end{code}

\begin{note}
Sliders geven altijd een waarde tussen 0 en 1. Je kan die schaal niet aanpassen, maar je kan wel het resultaat vermenigvuldigen tot de range die je nodig hebt. Wil je bijvoorbeeld een waarde tussen 10 en 50, dan schrijf je:

\begin{code}
float value = 10 + slider() * 40;
\end{code}
\end{note} 

\begin{exercise}
Maak een window met drie sliders en een \eeClass{Color}. De color stel je in het begin gelijk aan BLACK, maar de sliders wijzigen respectievelijk de r, g en b waarden van de color. Teken dan ook de achtergrond in deze kleur.
\end{exercise}
\chapter{TextLine}
TextLine is de enige gui class die je kan gebruiken om tekstinvoer van de gebruiker te krijgen. De visuele functies (grootte, positie, zichtbaarheid enzovoort) zijn identiek aan de andere classes die je zag. Maar de functie \eeFunc{func} werkt anders. Je kan daar op dezelfde manier een functie aan koppelen, maar waar bij een button of een checkbox de functie getriggerd wordt door een mouse click, zal een tekstline deze functie uitvoeren bij elke muisklik. Dat valt eenvoudig te demonstreren met de volgende code:

\begin{code}
Str myText;

// Om dit voorbeeld kort te houden wordt de gui class verder niet uitgewerkt.
// Je ziet enkel de callback functie.

void MyTextLineFunc(guiClass & obj) {
  myText = (*obj.myTextLine)();
}

// .. en de Draw functie
void Draw() {
	D.clear(BLACK);
	D.text(0, 0, myText);
}
\end{code}

Als je de invoer enkel wil controleren op een bepaald moment, dan kan je een button gebruiken en pas dan de invoer van de TextLine controleren. Een voorbeeld daarvan vind je in hoofdstuk \ref{chapter:gui_content}.

\section{Tekst omzetten naar een getal}

De inhoud van een TextLine is steeds een tekst, ook al bestaat die tekst uit cijfers. Als je die inhoud als een integer of float wil gebruiken, dan moet je die eerst converteren. Daarvoor bestaan er functies als \eeFunc{TextInt} en \eeFunc{textFlt}. 

\begin{code}
int   i = TextInt( (*obj.myTextLine)() );
float f = TextFlt( (*obj.myTextLine)() );
\end{code}

Let wel op: als je textline niet omgezet kan worden naar een getal, dan krijg je geen foutmelding. Het resultaat is dan steeds 0.

\section{Andere handige functies}
Wanneer je een tekst in een TextLine wil plaatsen, dan kan dat met de functie \eeFunc{set}. Die aanvaardt een string als argument:

\begin{code}
Str purpose = "life";
int meaning = 42;

myTextLine .set(purpose);
myTextLine2.set(S +  42);
\end{code}

Met de functie \eeFunc{password} kan je sterretjes in plaats van letters tonen:

\begin{code}
myTextLine.password(true);
\end{code}

Via de functie \eeFunc{clear} maak je een textline leeg:

\begin{code}
myTextLine.clear();
\end{code}

\begin{exercise}
TextLine heeft ook een functie \eeFunc{cursor} om de positie van de cursor op te vragen en aan te passen. Maak een programma met een tekstline waarin je een tekst plaatst. Maak ook functies \eeFunc{moveLeft} en \eeFunc{moveRight} die de cursor een positie naar links of naar rechts kunnen verplaatsen. In de \eeFunc{Update} functie van je class zorg je dat de cursor via F1 en F2 naar links en rechts verplaatst kan worden. Via F3 zet je de password modus aan of uit, en via F4 maak je het hele veld leeg.

Tot slot zoek je in de header file op hoe je tekst selecteert. Zorg er voor dat je via F5 de hele tekst selecteert, en via F6 de selectie ongedaan maakt.
\end{exercise}











\chapter{Translations}
In een ideale wereld spreekt iedereen dezelfde taal, maar zover zijn we helaas nog niet. Daarom zal je als vaak meerdere talen moeten ondersteunen in je software. In dit hoofdstuk maak je een eenvoudige translation manager. 

\begin{note}
Voor een niet al te uitgebreid programma met enkel korte zinnen is deze aanpak zeker geschikt. Wanneer je een programma met duizenden zinnen of volledige alinea's tekst maakt, dan heb je meer geavanceerde code nodig.
\end{note}

Maak om te beginnen enkele gui objecten die je in deze oefening wil vertalen. Tot hier toe was het zelden nodig om ook gewone teksten in een applicatie van een naam te voorzien. Die tekst veranderde namelijk nooit. Nu we andere versies in verschillende talen willen tonen is dat natuurlijk wel nodig. Met behulp van de uitleg uit de vorige hoofstukken zal je er zeker in slagen om ook een class voor je gui objecten uit te werken.

Maak daarnaast ook een gui met twee buttons, zoals in afbeelding \ref{fig:trans1}.

\begin{figure}[h]
\centering
\includegraphics[width=0.4\linewidth]{../images/translation_manager_1.png}
\caption[]{Een Language Gui.}
\label{fig:trans1}
\end{figure}

Bij deze gui hoort de volgende code:

\begin{code}
class languageGui
{
private:
   GuiObjs obj;  
   Window * window;
   Button * buttonDutch;
   Button * buttonEnglish;
   
public:     
   void create()
   {
      obj.load(=== drop gui object here ===);
      window        = obj.findWindow("window"       );
      buttonDutch   = obj.findButton("buttonDutch"  );
      buttonEnglish = obj.findButton("buttonEnglish");
      
      Gui += obj;
   }
}
languageGui LanguageGui;
\end{code}

Vergeet niet de create functie uit de voeren in \eeFunc{Init()}.

Maak ook alvast een class \eeClass{translationManager}. Die ziet er voorlopig zo uit.

\begin{code}
class translationManager {
private:
	LANG_TYPE language = LANG_ENGLISH;

public:
	void setLanguage(LANG_TYPE language) {
		T.language = language;
	}	

	C Str & translate(C Str & text) {
		return text;
	}
}
translationManager TM;
\end{code}

In deze class kan je een ingestelde taal onthouden via de enum \eeClass{LANG\_TYPE}. Daarnaast zijn er functies om de taal in te stellen en een string te vertalen. Die laatste functie heeft voorlopig gewoon de input als resultaat. We werken deze class later verder uit, maar kunnen ze nu al gebruiken om de basis van het programma uit te werken.

Nu de translation class bestaat, kunnen we ook de Gui classes afwerken. In de class \eeClass{languageGui} zullen we callback functies voor de buttons schrijven:

\begin{code}
static void funcDutch(ptr)
{
  TM.setLanguage(LANG_DUTCH);
}

static void funcEnglish(ptr)
{
  TM.setLanguage(LANG_ENGLISH);
}
\end{code}

Vergeet niet deze functies aan de respectievelijke buttons te koppelen, en voeg daarna ook de volgende functie aan de class \eeClass{languageGui} toe:

\begin{code}
void translate()
{
  window       .setTitle(TM.translate("Choose Language"));
  buttonDutch  .setText (TM.translate("Dutch"          ));
  buttonEnglish.setText (TM.translate("English"        ));
}
\end{code}

De bovenstaande functie maakt al duidelijk hoe we tewerk zullen gaan. We gebruiken Engelse teksten om de content van de Gui elementen in te stellen. Maar die tekst wordt eerst door de translation manager gestuurd. Daar kunnen we dan als resultaat de vertaling geven.

\section{The translationManager Class}
Tijd om aan het echte werk te beginnen: de translation manager. Deze heeft een container nodig om vertaalde strings op te slaan. Maar een container is niet erg geschikt voor dit doel. 

Typisch voor deze class is dat we steeds een Engelse tekst hebben en die willen vervangen door een alternatief in een andere taal. In een taal zoals php zou je dat op de volgende manier doen:

\begin{code}
translations["Life"] = "Leven";
\end{code}

\ldots wat toelaat om achteraf de vertaling te bekomen via:
\begin{code}
void printText(string s) {
   echo translations[s];
}
\end{code}

Zoiets kan niet in C++. De index van een container moet steeds een integer zijn. Hoe pak je dat dan aan?

\section{Map}
De held van de dag is de class \eeClass{Map}. Een map is perfect voor waarden die via een \'e\'en op \'e\'en relatie met mekaar gelinkt zijn, zoals een vertaling. Waar we een \eeClass{Memc} defini\"eren met enkel zijn data type:

\begin{code}
Memc<Str> strings;
\end{code}

vermelden we in \eeClass{Map} ook het type van de index:

\begin{code}
Map<Str, Str> translations;
\end{code}

Het eerste argument noemen we KEY, het tweede DATA. Niet enkel strings kunnen een key zijn, maar om het even welke class. En dat schept een nieuw probleem: om effici\"ent te zoeken in een Map, moet die intern gesorteerd worden. En om dat te doen, moet \eeClass{Map} elementen kunnen vergelijken. \eeClass{Map} moet kunnen beslissen of een key groter, kleiner of gelijk is aan een andere key. 

Je zal zelf een functie moeten schrijven die elementen vergelijkt. De definitie van \eeClass{Map} toont je hoe. Als eerste argument van de constructor map staat er:

\begin{code}
Int compare(C KEY &a, C KEY &b)
\end{code}

Je moet dus een functie schrijven die twee argumenten van hetzelfde type als je key aanvaardt, en die een integer als resultaat heeft. Esenthel verwacht dat dat resultaat -1, 0 of 1 is. 

\begin{itemize}
	\item -1: het eerste argument is kleiner dan het tweede.
	\item  0: beide argumenten zijn gelijk.
	\item  1: het eerste argument is groter dan het tweede.
\end{itemize}

Aangezien we in deze map een key van het type \eeClass{Str} hebben, is de oplossing heel eenvoudig. Er bestaat namelijk al een functie om strings op deze manier met mekaar te vergelijken. We kunnen de gevraagde functie dus zo schrijven: 

\begin{code}
static int mapCompare(C Str &a, C Str &b)
{
  return CompareCI(a, b);
}
\end{code}

Indien je een verschil wil maken tussen hoofdletters en kleine letters, kan je ook \eeFunc{CompareCS} gebruiken.

\begin{note}
Merk op dat dit een static functie is, net zoals een callback die je aan een button doorgeeft.
\end{note}

Nu deze functie bestaat, kan je de definitie van je Map verbeteren:

\begin{code}
Map<Str, Str> textMap(mapCompare);
\end{code}

\section{Load a translation}
Nu de map bestaat, maken we een functie om een bestand te laden en de inhoud in de map op te slaan.

We kiezen er in dit voorbeeld voor om te werken met een tekstbestand. In dat bestand staat telkens een Engelse tekst, gevolgd door een vertaling. De tekst en de vertaling worden gescheiden door een dubbele punt:

\begin{code}
Choose Language: Kies je taal
Dutch: Nederlands
English: Engels
\end{code}

Je maakt nu in de class \eeClass{translationManager} de volgende functie:

\begin{code}
void loadLanguage() {
	textMap.clear();
	if(language == LANG_ENGLISH) return;
}
\end{code}

Daarmee wijs je bij het laden van een vertaling eerst de bestaande vertaling. Wanneer de nieuwe taal Engels is, dan stoppen we de functie omdat er geen vertaling nodig is.

In het andere geval moeten we een bestand laden. Dit bestand zou uit een config directory kunnen komen, maar in dit geval importeren de vertalingen in het project. Dat kan zo:

\begin{enumerate}
	\item Maak een bestand `dutch.txt' met de inhoud hierboven.
	\item Sleep het bestand in de editor.
	\item Sleep het bestand naar de read functie, zoals in de code hieronder.
\end{enumerate}

Voeg aan de bovenstaande functie de volgende code toe:
\begin{code}
FileText ft;
switch(language) {
	case LANG_DUTCH:
	{ 
	    ft.read(=== drop file here ===);
	    break; 
	}
}
\end{code}

Je kan voor elke vertaling een bestand en een case instructie toevoegen. De functie \eeFunc{read} zorgt er voor dat het object \eeClass{ft} de inhoud van het bestand bevat.

\begin{note}
Telkens je content toevoegt aan het oorspronkelijke bestand, kan je in de editor rechts klikken op het bestand en `Reload' kiezen.
\end{note}

Tot slot moet elke regel in dit bestand gelezen worden, waarbij we de inhoud toevoegen aan de \eeClass{Map}. We gebruiken daarbij een container `parts' en de functie \eeFunc{Split}. De Split functie verdeelt elke regel in twee delen en slaat die op in `parts'. Bijgevolg bevat \eeFunc{parts[0]} de key en \eeFunc{parts[1]} de data.

\begin{code}
while(!ft.end())
{
 	Str line = ft.getLine();
 	Mems<Str> parts;
 	Split(parts, line, ':');
 
	(*textMap.get(parts[0])) = parts[1];
}
\end{code}

In de laatste regel gebruiken we de \eeFunc{get} functie van \eeClass{Map}, met als eerste argument de key. Die key bestaat nog niet, dus \eeClass{Map} maakt een nieuw element aan. De data ken je toe aan dat element.

Nu is de nieuwe vertaling wel geladen, maar de Gui zal die niet vanzelf gebruiken. Wel hebben we in de gui class een functie translate voorzien. Het is dus nodig om na het laden van een vertaling alle Gui's te updaten. We maken daar een afzonderlijke functie voor:

\begin{code}
void updateGuis()
{
  StatsGui.translate();
  // add other Gui's that must be translated
}
\end{code}

En nu kan je de functie \eeFunc{setLanguage} vervolledigen:

\begin{code}
void setLanguage(LANG_TYPE language)
{
  T.language = language;
  loadLanguage();
  updateGuis();
}
\end{code}

Wat nu nog ontbreekt is de werkelijke vertaling. Je past de functie \eeFunc{translate} aan en voegt de volgende code toe:

\begin{code}
C Str & translate(C Str & text)
{
  Str * result = textMap.find(text));
  if(result != null) return *result;
  
  return text;
}
\end{code}







\part{Data}
\chapter{Save data to File}
There are a lot of different ways to save data. Textfiles are simple to read, but less efficient when it comes to storage. Storing data in a binary format takes less storage, but the data is not readable by humans. A binary file is also less flexible when your data format might change. It is up to you to choose the right format for the job at hand.

\section{File Locations}
Whatever your choice, you will have to decide where to store the data. You could type in the full path in your code editor, but that's not a great idea.
Suppose you decide to save a config file to \verb|C:\myGame\gameData\config.txt.| This will be very annoying towards the user when she decides to save your app on another disk. It's even possible her computer does not even have a C drive!

Esenthel allows you to figure out some paths automatically. For example, you can retrieve the `documents' folder of the computer your app is running on:

\begin{code}
Str docPath = SystemPath(SP_DOCUMENTS);
\end{code}

When you'd like to save a file in the `documents' folder, it will be easy to construct the full path as:

\begin{code}
Str documentPath = S + docPath + "/myfile.txt";
\end{code} 

\begin{note}
You will be using the forward slash (/) in pathnames. Windows uses a backslash, but Esenthel will convert your forward slash when needed. This is because about every other operating system uses a forward slash in path names. To keep your project platform independent, always use forward slashes.
\end{note}

\verb|SP_DOCUMENTS| aside, there are some other system paths of use:

\begin{itemize}
\item \textbf{SP\_DESKTOP} will point to \verb|C:/Users/*/Desktop|.
\item \textbf{SP\_APP\_DATA} points to \verb|C:/Users/*/AppData/Roaming|. On a mobile device, this will be the location where the app is allowed to save data.
\item \textbf{SP\_APP\_DATA\_EXTERNAL} also refers \verb|C:/Users/*/AppData/Roaming|, but will point to an external SD card when your mobile device has one.
\item \textbf{SP\_ALL\_APP\_DATA} often refers to \verb|C:/ProgramData|
\end{itemize}

A complete list of possibilities can be found at \verb|Misc $\Rightarrow$ Misc|, below the enum declaration `SYSTEM\_PATH'. Not every path is available on every system though. You will not be able to write to the desktop of on a mobile platform.

\begin{exercise}
Find out where the paths on your computer point to. Create a simple test app which shows the different paths on the screen. \textsl{(Project Textfiles, ex. 01)}
\end{exercise}

\section{TextData}

The \eeClass{TextData} class is an easy way to store data as text. It's probably the best choice if you're looking to store a config file.

\subsection{Saving content}
You will uses `nodes' to store data. All data you consider worth saving must be assigned to a node. After assigning all data, the file must be saved to disk.

\begin{code}
TextData data;
data.getNode("name").setValue("John Doe");
data.getNode("highscore").setValue("100");
data.save(S + SystemPath(SP_DOCUMENTS) + "/config.txt");
\end{code} 

The method \eeFunc{getNode} will search for a node with a certain name. If no such node exists, a new one will be created. The method \eeFunc{setValue} will give the new node its value.

The resulting file will look like this:
\begin{verbatim}
name=`John Doe`
highscore=100
\end{verbatim}

\begin{exercise}
Create an app wit the variables \texttt{name}, \texttt{address}, and \texttt{age}. Assign default values to these variables and store them into a file. Open the file in a text editor and check these values.
\end{exercise}

\subsection{Loading a file}
\ldots goes like this:

\begin{code}
TextData data;
Str docPath = SystemPath(SP_DOCUMENTS);
data.load(S + docPath + "\config.txt");
\end{code}

If your documents folder has a file named `config.txt', it will be read. \textsl{(The function also returns a bool to indicate wether or not it succeeded. In a real application you should check its value.)}

\subsection{Reading content}
After loading a file, you can read its contents. To do this, you'll have to find the node you're looking for and assign its contents to a variable. Because it is possible that a certain node does not exist in your file, you should take two precautions:

\begin{enumerate}
	\item Assign a default value to all variables. When a node is not found, you still have the default value to work with.
	\item Only access a node when it exists. Accessing a node that does not exist will crash your application.
\end{enumerate}

\begin{code}
// config values
Str name = "new player";
int highScore = 0;
int credits = 100;

TextData data;
Str docPath = SystemPath(SP_DOCUMENTS);
if(data.load(S + docPath + "/config.txt")) {
   TextNode * node;
	 
	 // try finding the name
	 node = data.findNode("name");
	 if(node) {
	    name = node.asText();
	 }
	 
	 // try finding the highscore
	 node = data.findNode("highscore");
	 if(node) {
	    highScore = node.asInt();
	 }
}
\end{code}

\begin{note}
While saving a node, you can assign it about every value you want with the function\eeFunc{setValue}. But if you try to read it, you will have to tell the compiler what type of variable you're trying to read: you can use functions like \eeFunc{asInt} to load an integer, or \eeFunc{asText} to load a string. More possibilities can be found in \verb|File $\Rightarrow$ Xml|. (\eeClass{TextNode} is based on \eeClass{TextParam}.)
\end{note}

\begin{exercise}
Adjust the generated file from the previous exercise in a text editor and load it back into your application. Show the values on the screen and double check to see if they are correct. \textsl{(Project Textfiles, ex. 02)}
\end{exercise}

\subsection{A class for a config file}
The following example is a complete class for your own config file. You should be able to adapt this to the needs of your own project. \textsl{(Project Textfiles, ex. 03)}

\begin{code}
class configFile {
private:
  Str name      = "new player";
	int highscore = 0  ;
	int credits   = 100;
	
public:
  Str getName     () { return name     ; }
	int getHighscore() { return highscore; }
	int getCredits  () { return credits  ; }
	
	void setName     (C Str & name   ) { T.name      = name   ; }
	void setHighscore(  int   score  ) { T.highscore = score  ; }
	void setCredits  (  int   credits) { T.credits   = credits; }
	
	void load() {
	  TextData data;
		if(data.load(S + SystemPath(SP_APP_DATA) + "/mygame/config.txt")) {
		  TextNode * node;
			
			if(node = data.findNode("name"     )) name      = node.asText();
			if(node = data.findNode("highscore")) highscore = node.asInt ();
			if(node = data.findNode("credits"  )) credits   = node.asInt ();
	  }
  }
	
	void save() {
	  TextData data;
		data.getNode("name"     ).setValue(name     );
		data.getNode("highscore").setValue(highscore);
		data.getNode("credits"  ).setValue(credits  );
		
		data.save(S + SystemPath(SP_APP_DATA) + "/mygame/config.txt");
  }
}
configFile ConfigFile;
\end{code}	

\section{Binary Files}
Files created with the class TextData are very useful for config files because you can change the contents with any text editor. But it's not the most efficient approach to store data. Data can be stored much more compact with the class \eeClass{File}. This class will store the data in a binary format, without an ID or any kind of internal structure to figure out what is what. The next example shows you how to open a file for writing and store several values in it. Note that this time, you have to specify the type you would like to store.

\begin{code}
File f;                  // create file object
f.write("file.dat"    ); // start writing to a file
f.putInt(128          ); // write an integer
f.putFlt(3.14         ); // write a float
f.putStr("Hello world"); // write a string
\end{code}

Reading data is similar. The function \eeFunc{readTry} will open the file for reading, if this is possible. If not, it will return false. \textsl{(A function \eeFunc{read} also exists, but when reading the data does not succeed, it will crash your application.)}

When you want to read data from a file, you will have to do this in the same order you stored it. Remember, the class \eeClass{file} has no clue what kind of data it is holding. 

\begin{code}
File f;
if (f.readTry("file.dat")) {
	int   i    = f.getInt();
	float pi   = f.getFlt();
	Str   text = f.getStr();
}
\end{code}

\begin{exercise}
Create a gui class with a \eeClass{TextLine}, a \eeClass{Slider}, a \eeClass{CheckBox} and two \eeClass{Button}s. The first button will save the current value of the gui elements on disk. The second button will load these values from disk and assigns them to the gui elements. Make sure to check if everything works like it should.
\end{exercise}

\subsection{Data Order}
But what happens if you read the contents of a file in another order you used while storing it? Look at this example:

\begin{code}
void save() {
  File f;
	f.write("trouble.dat");
	f.putBool(true);
	f.putInt (12  );
}

void load() {
  File f;
	f.read("trouble.dat");
	int i = f.getInt ();
	int b = f.getBool();
}
\end{code}

The save and load functions above do not use the same order for storing and reading. But the compiler will not alert you because there is no real error in the code as such. A bool variable is exactly 1 bit in size, and an integer takes 32 bits. In total 33 bits are stored: 1 + 32. But while reading the file, the code will ask for an integer first, and for a bool after that. So it will read 32 + 1 bits. That's still 33 bits! Only \ldots the values will be different from the ones you saved. Figure \ref{fig:filebits} illustrates this.

\begin{figure}[ht]
\centering
\includegraphics[width=0.8\linewidth]{../images/filebits}
\caption[]{bits in a file}
\label{fig:filebits}
\end{figure}

\begin{note}
The variable order while reading must match exactly with the order while saving. Unexpected behavior will follow if it doesn't.
\end{note}

And there's another, less known, way in which things can go south. C++, like most programming languages, will handle function arguments from right to left. Most of the time we're not bothered by this. Take this code for example:
\begin{code}
float r = 0.1;
float x = 0.3;
float y = 0.4;

Circle c;
c.set(r, x, y);
\end{code}

If arguments are read from right to left, it means the instruction \eeFunc{pos2.set} happens in this order:

\begin{enumerate}
	\item Assign the value of y to the third argument of set.
	\item Assign the value of x to the second argument of set.
	\item Assign the value of r to the first argument of set.
	\item Execute whatever code is written inside set.
\end{enumerate}

We are not worried because it really doesn't matter in which order the data is being read. It's just there. The variables r, x and y can be read at any moment and they will always contain the right data.

The trouble starts when you read from a \textit{sequential} format. And yes, a binary file is a sequential format. So when you store these values to a file and read them like this\ldots

\begin{code}
float r = 0.1;
float x = 0.3;
float y = 0.4;

// Put data in an imaginary file
f.putFlt(r);
f.putFlt(x);
f.putFlt(y);

// ... after loading this data
Circle c;
c.set(f.getFlt(), f.getFlt(), f.getFlt());
\end{code}

\ldots you will be in for a big surprise (or at least a very confused coding session). Remember, the file contains the floats r, x and y. In that order.

When loading those floats in the last line of code, this is what happens:

\begin{enumerate}
\item The compiler sees that the third argument for set has to be retrieved from a file. "hey, give me a float", says the compiler to the file.
\item The file happily obliges and comes up with the first float in the data.
\item The compiler then finds out it needs another float, for the second argument. Again, a float is requested from the file.
\item The file, not having any clue what these floats are for, delivers the second float in the file.
\item You know what happens next. The third float in the file will be dumped as the first argument of set.
\end{enumerate}

To solve this sad misunderstanding between the compiler, the file and you, is is best to keep one thing in mind: don't retrieve more than one value from a file within a single argument list. In this particular case this means you don't use the set function.

\begin{code}
Circle c;
c.r     = f.getFlt();
c.pos.x = f.getFlt();
c.pos.y = f.getFlt();
\end{code}

\begin{exercise}
Now is your chance to do some evil programming. Abuse your skills to create a little program that demonstrates this error! \textsl{(Project Textfiles, ex. 04)}
\end{exercise}

\subsection{Storing multiple objects}
Sooner or later, you will want to store multiple objects in one file. There are several ways to do this, but as a rule of thumb two things will make your day easier.

\begin{enumerate}
	\item Before storing the actual object, store the number of objects that will follow.
	\item Create save and load methods for your object, and pass the current file as a reference.
\end{enumerate}

Below, you see a resource class with save and load methods. Notice that these methods do not save or load any file directly. They merely add or read data to and from a file that will be created elsewhere.

\begin{code}
class resource {
  int type  ;
	Vec pos   ;
	int amount;
	
	void create(int type, int amount, C Vec & pos) {
	  T.type   = type  ;
		T.pos    = pos   ;
		T.amount = amount;
	}
	
	// ... other functions are omitted to keep this example short ...
	
	void save(File & f) {
	  f.putInt(type  );
		f.putFlt(pos.x ).putFlt(pos.y).putFlt(pos.z);
		f.putInt(amount);
	}
	
	void load(File & f) {
	  type   = f.getInt();
		pos.x  = f.getFlt();
		pos.y  = f.getFlt();
		pos.z  = f.getFlt();
		amount = f.getInt();
	}
}	
\end{code}

The next part is done by some sort of manager class. It has a container for resources (you've seen this concept before) and provides methods for saving and loading. 

This time, a file is opened. First the number of elements is written to the file. Afterwards, every resource is asked to write its own data to the same file.

\begin{code}
class resourceManager {
  Memx<resource> resources;
	
	void save() {
		File f;
		f.write("resources.dat");
		
		f.putInt(resources.elms());
		FREPA(resources) {
			resources[i].save(f);
		}
	}
	
	void load() {
	  File f;
		if(f.readTry("resources.dat")) {
			
			int elms = f.getInt();
			for(int i = 0; i < elms; i++) {
				resources.New().load(f);
			}
		}
	}
}

resourceManager RM;
\end{code}

When the file is loaded from disk, the number of stored resources is retrieved first. This is followed by a loop to create and load just as many resources as were stored in the file.

You can even go one step further and store every type of data in the same file, as long as you keep an eye of the order in which you write and read.

A program can call save functions for a resource manager, config file or anything you need, just by passing the file as a reference to every object with a save function. 

\begin{code}
void save() {
  File f;
	f.write("allData.data");
	Config.save(f);
	RM    .save(f);
}

void load() {
	File f;
	f.load("allData.data");
	Config.load(f);
	RM    .load(f);
}
\end{code}

\begin{note}
There is one drawback to this approach. When you update your software and add some new data, the new version of your application will not be able to read files written by the old version. Of course there are ways around this, but for now you'll just have to remember this limitation.
\end{note}

\begin{exercise}
Create a class for a circle that can remember its color. Create a manager class that can store these circles. Every mouse click adds a circle to this manager, on the current mouse position. (Reread chapter \ref{section:managerClass} if you need help.)

When you close the application all circles are to be saved on disk. When the application starts, you must load all circles from the previous sessions.
\textsl{(Project Textfiles, ex. 05)}
\end{exercise}



\chapter{Databases}

\section{Database Servers}
Data kan je niet enkel in gewone bestanden opslaan, je kan ook een database gebruiken. De meest voorkomende standaard voor het werken met databases is SQL (sequel). Esenthel Engine ondersteunt 3 verschillende databases: Microsoft SQL, MySQL en SQLite. Elk van deze databases heeft voor- en nadelen.

\begin{itemize}
\item \textbf{Microsoft SQL:} De database server is zeer performant, maar je hebt wel een Windows OS nodig. Tijdens de ontwikkeling zal dit waarschijnlijk geen probleem zijn, maar wanneer je een server voor je programma laat hosten dan is een Windows host duurder.
\item \textbf{MySQL:} Dit is waarschijnlijk de meest gebruikte SQL server. MySQL draait op zowat elk OS, is gratis en open source. De installatie en het onderhoud zijn niet erg moeilijk, en de performantie is OK.
\item \textbf{SQLite:} SQLite gebruikt een gewoon bestand als database. Er moet dus geen server ge\"installeerd worden. Vooral tijdens de ontwikkelfase kan dat een voordeel zijn, maar het zorgt er wel voor dat SQLite heel wat trager is dan een `echte' database server. SQLite kan ook een goede oplossing zijn als je aan de client side data wil opslaan in een database, zonder dat je daarvoor de gebruiker verplicht om een database server te installeren.
\end{itemize}

Een belangrijk voordeel van de database class in Esenthel is dat je makkelijk van database kan wisselen. Je kan dus tijdens het ontwikkelen voor SQLite kiezen en pas achteraf overschakelen op een MySQL of MS SQL database.

\section{Een Database Gebruiken}
\subsection{De Verbinding}
Alvorens je een database kan gebruiken, moet je een verbinding maken. De \texttt{SQL} class laat je toe een verbinding te maken met elk soort database. In het geval van een SQLite database is de naam van een bestand voldoende, maar bij een echte server zal je de host, de naam van de database, een gebruiker en een wachtwoord moeten voorzien. MS SQL en MySQL laten je ook nog toe een pointer naar een string mee te geven. Indien de verbinding niet lukt, zal die string de foutmelding van de database server bevatten.

\begin{code}
SQL sql;
Str messages;

// voor Microsoft SQL
if (!sql.connectMSSQL("LocalHost\\SQLExpress", "test_db", "", "", &messages)) {
  // verbinding mislukt
	Exit(S + "Database error: " + messages);
}

// voor MySQL
if (!sql.connectMySQL("localhost", "test_db", "user", "password", &messages)) {
  // verbinding mislukt
	Exit(S + "Database error: " + messages);
}

// voor SQLite
if (!sql.connectSQLite("test.db")) {
  Exit(S + "Database error");
}
\end{code}

\subsection{Een tabel maken}
Je programma kan bestaande tabellen gebruiken, of je kan met een andere tool je de tabellen in je database aanmaken. Een derde optie is dat je de database aanmaakt op het moment dat je je programma start. Ook dit kan weer handig zijn tijdens de ontwikkelfase: je controleert of een tabel al bestaat, en indien niet dan laat je je programma die tabel maken. Eventueel kan je er ook via je code wat data in zetten. Op die manier kan je je programma eenvoudig testen. En als je de structuur van je data wil wijzigen, dan verwijder je gewoon de hele database zodat die opnieuw aangemaakt wil worden.

Zoals je weet bestaat een tabel uit kolommen met een naam en een type. Bovendien moet in elke tabel \'e\'en kolom de primary key zijn, eventueel met auto-increment. Je kan ook voor elke kolom een default waarde instellen, maar dat is niet verplicht.

\begin{code}
if(!sql.existsTable("accounts")) {
  // maak een tabel voor accounts
	
	Memc<SQLColumn> columns;
	columns.New().set("ID", SDT_INT).mode = SQLColumn.PRIMARY_AUTO;
	columns.New().set("name", SDT_STR, 32); // een string met maximaal 32 tekens
	columns.New().set("password", SDT_STR, 32);
	columns.New().set("active", SDT_BOOL);
	columsn.New().set("score", SDT_INT).default_val="0";
	
	if(!sql.createTable("accounts", columns, &messages)) {
	  Exit(S + "Can't create table for accounts: \n" + messages);
	}
	
	// voeg test data toe
	SQLValues values;
	values.New("name", "freddy");
	values.New("password", "kamerplant");
	values.New("active", true);
	if(!sql.newRow("accounts", values, &messages)) {
	  Exit(S + "Can't add data to table accounts: \n" + messages);
	}
}
\end{code}

\subsection{Data Lezen}
Als je eenvoudig alle data uit een tabel wil lezen, dan kan dat met de functie \texttt{getAllRows()}. Je wil die data dan waarschijnlijk wel ergens in het geheugen houden, dus daar gebruik je best een class en een memory container voor. Met de table ``accounts'' als voorbeeld zou je een class \texttt{account} kunnen maken om een rij uit de database in het geheugen te zetten.

\begin{code}
class account {
  // Om het voorbeeld kort te houden zijn alle variabelen public.
	// In een echt programma kan je meestal beter set en get functies gebruiken.
	int ID;
	Str name;
	bool active;
	int score;
}
\end{code}

De functie \texttt{getAllRows()} gebruik je eenvoudigweg met de naam van een tabel. Na het uitvoeren van de functie bevat het \texttt{sql} object alle rijen van de gevraagde tabel. Met de functie \texttt{getNextRow()} kan je dan een rij met gegevens opvragen totdat er geen volgende rij meer is. Vervolgens kan je de functie \texttt{getCol()} gebruiken om de waarde van een kolom te verkrijgen. 

\begin{note}
Vraag de kolommen van een rij steeds op in de volgorde waarin ze in de tabel staan. Je kan wel kolommen overslaan, maar niet terug gaan naar een vorige kolom.
\end{note}

\begin{code}
Memc<account> accounts;
sql.getAllRows("accounts");

for( ;sql.getNextRow(); ) {
  account & a = accounts.New();
	sql.getCol(0, a.ID);
	sql.getCol(1, a.name);
	// skip loading the password in column 2
	sql.getCol(3, a.active);
	sql.getCol(4, a.score);
}
\end{code}

Je wil niet steeds alle data uit een tabel in het geheugen laden. Dikwijls ben je maar ge\"interesseerd in een of in enkele records. In dat geval gebruik je de functie \texttt{getRows()}. Als argument kan je bij deze functie een voorwaarde opgeven, net zoals je dat in een SQL statement zou doen:

\begin{code}
sql.GetRows("accounts", "name='freddy'");
if(sql.getNextRow()) {
	// load column data
}
\end{code}

Dikwijls zal je dergelijke code in een functie gebruiken, en een bepaalde waarde als resultaat geven. Stel je voor dat we van een account de score willen weten. De functie zou er dan zo kunnen uitzien:

\begin{code}
int getScore(C Str & name) {
  int result = 0;
	sql.GetRows("accounts", S+ "name='" + name + "'");
	if(sql.getNextRow()) {
	  sql.getCol(4, result);
	}
	return result;
}
\end{code}

\begin{note}
Bij het opstellen van een voorwaarde moet elke waarde tussen enkele quotes staan. Als je dan ook nog variabelen toevoegt aan de vergelijking, dan moet je goed uitkijken dat de quotes op de juiste plaats staan.
\end{note}

Maar je zou ook een hele account als referentie kunnen doorgeven. Dat is vooral bruikbaar in het geval je een volledige record wil laden. We zouden dan eerst de class \texttt{account} uitbreiden:

\begin{code}
class account {
	int ID;
	Str name;
	Str password;
	bool active;
	int score;
	
	bool load(C Str & name) {
		T.name = name;
	  return loadAccount(T);
	}
}
\end{code}
Via de functie load kunnen we een account uit de database halen door een naam mee te geven. De return waarde laat ons ook weten of dat gelukt is. De functie \texttt{loadAccount()} kunnen we dan zo uitwerken:

\begin{code}
bool loadAccount(account & a) {
  sql.GetRows("accounts", S+"name='"+a.name+"'");
	if(sql.getNextRow()) {
		sql.getCol(0, a.ID);
		sql.getCol(2, a.password);
		sql.getCol(3, a.active);
		sql.getCol(4, a.score);
		return true;
  }
	return false;
}
\end{code}

\subsection{Records tellen}
Het gebeurt dat je niet in de inhoud van een record ge\"interesseerd bent, maar enkel wil weten of een record bestaat. Wil je eenvoudigweg weten hoeveel records een tabel bevat, dan gebruik je de functie \texttt{getAllRowsNum()}.

\begin{code} {
int accountsInDB() {
  return sql.getAllRowsNum("accounts");
}
\end{code}

Vaker wil je het aantal accounts dat aan een voorwaarde voldoet. Bijvoorbeeld het aantal actieve accounts. Dan gebruik je de functie \texttt{getRowsNum}, die weer een conditie als argument heeft, net zoals \texttt{getRows()}.

\begin{code}
int activeAccountsInDB() {
  return sql.getRowsNum("accounts", "active='true'");
}
\end{code}

Met zo'n voorwaarde kan je ook te weten komen of een combinatie van gegevens bestaat. Zo kan je bijvoorbeeld controleren of een wachtwoord juist is.

\begin{code}
bool validate(C Str & name, C Str & password) {
  int count = sql.getRowsNum("accounts", S + "name='" + name + "' AND password='" + password + "'");
	return (count > 0);
}
\end{code}

\subsection{Data Opslaan}
Je zal vrijwel nooit alle gegevens in een database willen overschrijven. Wel wil je regelmatig een record die je aangepast hebt, terug opslaan. Dat kan zo:

\begin{code}
void save(account & a) {
  SQLValues values;
	values.New("name", a.name);
	values.New("password", a.password);
	values.New("active", a.active);
	values.New("score", a.score);
	
	Str messages;
	if(!sql.setRow("accounts", S + "ID='" + a.ID + "'", values, &messages)) {
	  Exit(S + "Error saving account: \n" + messages);
	}
}
\end{code}

Je hoeft ook niet steeds alle data op te slaan. Enkel de waarden die je in \texttt{SQLValues} opneemt, worden aangepast. Zo kan je bijvoorbeeld enkel het wachtwoord opslaan.

\begin{code}
void savePassword(account & a) {
  SQLValues values;
	values.New("password", a.password);
	sql.setRow("accounts", S + "ID='" + a.ID + "'", values);
}
\end{code}


\begin{note}
Pas nooit de primary key van een record aan!
\end{note}

\subsection{Data Verwijderen}
Tot slot zal je ook records willen verwijderen die je niet meer nodig hebt. Daar kan je de functie \texttt{delRow} voor gebruiken. 

\begin{code}
void removeAccount(int ID) {
  sql.delRow("accounts", S + "ID='" + ID + "'");
}
\end{code}
\chapter{Over netwerk applicaties}
Tegenwoordig heeft bijna elke applicatie wel ergens een netwerk nodig. Bij online games is dat evident, maar ook andere toepassingen houden steeds meer info bij in \textit{'the cloud'}. \textit{(Die cloud is niets meer dan een fancy woord om aan te duiden dat je informatie op een server opslaat.)} Ook wordt een online component dikwijls gebruikt om een licentie te controleren.

Toch is een netwerk applicatie een pak moeilijker te ontwikkelen dan een gewone, stand-alone applicatie. Enkele redenen daarvoor:

\begin{itemize}
\tick Je schrijf niet \'e\'en, maar twee programma's. Er moet namelijk ook een server geschreven worden.
\tick Data die je verstuurt over het netwerk bestaat enkel uit bits. Zowel de client als de server moeten die op dezelfde manier interpreteren.
\tick Een netwerk kan traag en onbetrouwbaar zijn. Je software moet dat zo goed mogelijk opvangen.
\tick Mobile apps hebben dikwijls weinig bandbreedte. Je mag niet meer data versturen dan strikt noodzakelijk is.
\end{itemize}

Het is dan ook erg belangrijk dat je het overzicht bewaart bij het onwikkelen van een netwerk applicatie. In de volgende hoofdstukken zullen we de delen van zo'n programma bestuderen. Hierin staan verschillende technieken beschreven die je helpen dat overzicht te bewaren. In ieder geval is het belangrijk om je project goed te structureren. In figuur \ref{fig:filetree} zie je een file tree van het voorbeeldproject.

\begin{figure}[ht]
\centering
\includegraphics[width=0.3\linewidth]{../images/filetree.png}
\caption[]{Structuur van een client-server project.}
\label{fig:filetree}
\end{figure}

Je ziet dat zelfs een eenvoudig project al snel uit heel wat files bestaat. Je zet die niet allemaal in dezelfde map. Evident is dat we twee applicaties hebben: client en server. Maar het is zeker niet overbodig om ook binnen die applicaties folders te maken voor code die samen hoort. Zo heeft de client een folder voor alle gui elementen \textit{(zowel layout als code)} en een folder voor code i.v.m. peers \textit{(dat zijn andere spelers op het netwerk)}. We plaatsen dus ook de editor componenten niet zomaar in de root van het project. Immers, de server heeft dat gui element `details' helemaal niet nodig. Bij een verdere uitwerking zal de server waarschijnlijk zijn eigen gui elementen nodig hebben, waar de client dan weer niets aan heeft.

Daarnaast zie je ook een folder `enums'. Deze folder is w\'el beschikbaar voor elke app. We gebruiken enumeraties om berichten door te geven tussen de client en de server. Door ze in de root van het project te plaatsen zijn we zeker dat beide applicaties dezelfde lijst gebruiken.

Ook is er een library (groene folder) met de naam `Shared'. In deze folder plaatsen we alle code die zowel bij de client als de server hoort.

\begin{exercise}
Maak een nieuw project aan. Daarin maak je alvast alle folders en bestanden zoals in de afbeelding hierboven.
\end{exercise}

\section{Messages}
Data verzenden over het netwerk doe je via \texttt{Files}. Denk hierbij niet aan grote bestanden zoals op een harde schijf: elk bericht dat je verstuurt over het netwerk is in feite in klein bestandje. 

Aangezien je bij een file dient te vermelden hoe je het wil noemen, voordat je er data in kan zetten, moet je bij een netwerkfile aangeven dat het enkel om het bestand in het geheugen gaat. En omdat de ontvanger van het bericht moet weten over wat voor bericht het gaat, stuur als eerste byte steeds een enum met het type bericht. Hieronder zie je een voorbeeld van een dergelijk bestand, ditmaal om de positie van een speler te verzenden.

\begin{code}
File f;
f.writeMem();
f.putByte(M_CLIENT_POS);
f.putFlt(pos.x);
f.putFlt(pos.y);
\end{code}

Alle netwerkfuncties in de client en de server zullen dus steeds naar de eerste byte van een bericht kijken om te bepalen welke functie het bericht verder zal afhandelen. Je maakt best een enumeratie die alle mogelijke berichten bevat. In een groot programma kan ook een tweede byte geschakeld worden om een verdere onderverdeling te maken.

\begin{exercise} 
Maak in je project een enum `MESSAGE' met de volgende waarden: 
\begin{itemize}
	\item M\_CLIENT\_FULL
	\item M\_CLIENT\_POS
	\item M\_HELLO
	\item M\_ADD\_CLIENT
	\item M\_REMOVE\_CLIENT
\end{itemize}
\end{exercise}



\section{Gedeelde classes}
Je app zal data structuren nodig hebben die zowel bij de client als bij de server bekend zijn. Dat kan bijvoorbeeld een class voor de player zijn, met minstens een naam en een positie. Het zou ook een item in de game wereld kunnen zijn, een chat bericht of een quest. 

Aan de andere kant zijn er ook steeds verschillen tussen de client en de server. Zo zal een player bij de client op het scherm moeten verschijnen, en zal de gebruiker hem kunnen verplaatsen via de muis of het toetsenbord. Bij de server is dat niet wenselijk, maar moet een player wel in een database opgeslagen kunnen worden.

We maken voor dit soort classes een zogenaamde `base class' in de library `Shared'. Het voorbeeld bevat een dergelijke class voor een speler, die we \texttt{netClient} noemen:

\begin{code}
class netClient
{
   int id;
   Str name = "Player";
   Color color = RED;
   Vec2 pos;

   void writePosToFile(File & f)
   {
      f.putFlt(pos.x);
      f.putFlt(pos.y);
   }
   
   void readPosFromFile(File & f)
   {
      pos.x = f.getFlt();
      pos.y = f.getFlt();
   }
      
   void writeDetailsToFile(File & f)
   {
      f.putStr(name);
      f.putByte(color.r).putByte(color.g).putByte(color.b);
      writePosToFile(f);
   }
   
   void readDetailsFromFile(File & f)
   {
      name = f.getStr();
      color.r = f.getByte(); color.g = f.getByte(); color.b = f.getByte();
      readPosFromFile(f);
   }
}
\end{code}

Naast data zoals id, naam, kleur en positie bevat de class ook functies. Als we een bericht over het netwerk versturen, zullen we steeds deze functies gebruiken om de gewenste data aan een \texttt{File} toe te voegen. Zo zijn we zeker dat de client en de server dezelfde data verwachten. Mocht je in een later stadium bijvoorbeeld de snelheid van een speler willen onthouden en die ook meesturen in een `Details' bericht, dan moet je dat enkel hier aanpassen. 

\begin{note}
Het is soms verleidelijk om deze data rechtstreeks in de client of server applicatie naar een bestand te schrijven. Vroeg of laat zal je je echter vergissen, door bijvoorbeeld de volgorde van de data door mekaar te halen. Dat soort fouten is erg moeilijk te debuggen. Maak er daarom een gewoonte van om een gedeelde base class te maken. 
\end{note}

\chapter{De Server}
De server applicatie is verantwoordelijk voor het volgende:

\begin{itemize}
\tick connecties van nieuwe clients aanvaarden;
\tick berichten van bestaande clients aanvaarden en indien nodig doorsturen naar andere clients;
\tick opmerken wanneer een clients offline gaat en dat indien nodig laten weten aan andere clients;
\end{itemize}

Daarnaast is het meestal zo dat data ook opgeslagen wordt in een database, dat er game-events gegenereerd worden, het gedrag van AI's wordt berekend, etc.

\section{Main Program Loop}
Het hoofdprogramma van de server is meestal behoorlijk eenvoudig: je start een server object, update dat regelmatig en tekent eventueel wat op het scherm. Dat laatste is zelfs dikwijls ongewenst, want een serverprogramma draait dikwijls op een server OS zonder gui. We overlopen even de voorbeeld code.

\begin{code}
void InitPre()
{
   EE_INIT();
   App.flag = APP_WORK_IN_BACKGROUND|APP_NO_PAUSE_ON_WINDOW_MOVE_SIZE;
}
\end{code}

De \texttt{InitPre()} functie is niet bijzonder. De Applicatie krijgt enkele flags mee zodat de update functie ook door gaat als het programma geen focus heeft of geminimaliseerd is.

\begin{code}
bool Init()
{
   if(!Server.create())
   {
      Exit("Can't create Server");
   }
   return true;
}
\end{code}

In \texttt{Init()} cre\"eren we het \texttt{Server} object. Die server class moeten we wel zelf schrijven, wat hieronder aan bod komt. Als het niet lukt om de server te starten, dan wordt het programma afgesloten.

\begin{code}
void Shut()
{
   Server.del();
}
\end{code}

Tot nu toe hebben we de \texttt{Shut()} functie vrijwel nooit nodig gehad. Bij een server applicatie moeten we zeker zijn dat de netwerk resources terug vrijgegeven worden als het programma klaar is. Daarom is deze code noodzakelijk.

\begin{code}
bool Update()
{
   if(Kb.bp(KB_ESC))return false;
   Server.update();
	 Time.wait(1);
   return true;
}
\end{code}

De \texttt{Update()} functie update de server. Later zullen waarschijnlijk ook andere updates toegevoegd worden, zoals bijvoorbeeld een AI manager update. Op het eind van de update functie laten we het programma een milliseconde wachten. Zo belasten we de CPU niet harder dan nodig.

\begin{code}
void Draw()
{
   D.clear(TURQ);
   D.text(0, 0.7, S + "Server");
   D.text(0, 0.5, S + "Connected clients: " + Server.clients.elms());
   D.text(0, 0.3, S + "Local Addres: " + Server.addressLocal().asText());
}
\end{code}

Tot slot is er de \texttt{Draw()} functie die ons wat informatie geeft over de server. Het aantal actieve clients en het IP adres van de server.

\begin{note}
In dit voorbeeld gebruiken we het lokale IP adres van de server. Daarmee kan je een client op hetzelfde netwerk verbinden met de server, maar externe verbindingen zijn zo niet mogelijk. Wanneer het tijd is om de toepassing te testen over het internet, dan moet je het globale IP adres gebruiken. Het is mogelijk dat je daarvoor ook de router correct moet configureren.
\end{note}

\begin{exercise}
Voeg de functies hierboven toe aan het bestand `main'.
\end{exercise}

\section{De server Class}
De server zelf moet je niet zelf ontwikkelen. Esenthel voorziet een base class voor je server, waar je je eigen functies aan toevoegt. Meestal heb je maar enkele functies nodig. We overlopen stap voor stap wat er in deze class dient te staan. Ten eerste is er de class zelf:

\begin{code}
class server : connectionServer
{
   // andere code
}
server Server;
\end{code}

De \texttt{server} class heeft als base class \texttt{connectionServer}. Die laatste wordt voorzien door de engine. Aangezien er maar \'e\'en server object actief zal zijn, kunnen we er dadelijk een object van maken. 

\subsection{De Constructor}

Een eerste functie binnen de server class is de constructor:
\begin{code}
server() { clients.replaceClass<client>(); }
\end{code}

Constructors komen vaak voor in C++, maar in Esenthel schrijf je ze meestal niet zelf. Het betreft een functie met dezelfde naam als de class, die automatisch wordt uitgevoerd bij het maken van een object. De class \texttt{connectionServer} bevat een container clients waarin elke actieve client onthouden wordt. De class van die clients willen we vervangen door onze eigen `client' class die we zodadelijk zullen ontwerpen. Met de functie \texttt{replaceClass} laten we dat weten aan de \texttt{connectionServer}.

\subsection{Data verzenden}

\begin{code}
void sendToClients(File & f, client & sender)
{
	clients.lock();
	FREPA(clients)
	{
		 client & c = (client&) clients.lockedData(i);
		 if(&c != &sender) // don't send back to sender
		 {
				f.pos(0);
				c.connection.send(f);
		 }
	}
	clients.unlock();     
}
\end{code}

Vaak zullen we ontvangen informatie willen versturen naar alle clients, behalve naar de afzender. Omdat eenvoudig te doen vanuit andere delen van het programma voorzien we een functie die de te verzenden informatie \textsl{(een File)} en de afzender \textsl{(een client)} als argument heeft.

Nieuw hierbij is het \textsl{lock} concept. Een server zal dikwijls verschillende processoren tegelijk gebruiken om zo snel mogelijk te gebruiken. Stel je voor dat er clients bijkomen of verdwijnen terwijl we met \texttt{FREPA} alle clients afgaan. De server zou dan waarschijnlijk crashen. Om dat te voorkomen wordt de clients container `gelocked'. Na de loop dien je een `unlock' te gebruiken, zodat er terug clients toegevoegd kunnen worden.

Elke client wordt vervolgens omgezet naar onze eigen \texttt{client} class. Dat gebeurt met het statement:
\begin{code}
client & c = (client&) clients.lockedData(i);
\end{code}
Dit is nodig omdat de connectionServer zijn eigen base class voor een client heeft. We hebben die echter vervangen door een eigen class. Hier geven we aan dat we uit de clients container een referentie naar een client(i) willen gebruiken, maar wel als onze eigen class.

Vervolgens willen we zeker zijn dat we de data niet terug naar de afzender sturen. Om dat te doen vergelijken we het geheugenadres van de huidige client met dat van de afzender. Enkel wanneer die verschillend zijn, versturen we de File.

Het verzenden van die File bestaat uit twee stappen. We beginnen met de positie in de File terug op het begin te zetten. Vervolgens gebruiken we de connectie van de huidige client om de File naar die client te sturen.

\begin{note}
Een functie die data uit een File leest, doet dat vanaf de huidige positie en verplaatst die positie stap voor stap \textsl{(of beter bit voor bit)} naar het eind van de file. Indien een bestand nogmaals gelezen wordt, moet je de positie dus terug op nul zetten.
\end{note}

\subsection{Info voor nieuwe clients}
Telkens een nieuwe speler zich aanmeld bij de server, heeft die informatie nodig over alle andere spelers. We voorzien daarom een functie \eeFunc{sendAllClientDetails}. De functie ziet er zo uit:

\begin{code}
void sendAllClientDetails(client & destination)
{
	clients.lock();
	FREPA(clients)
	{
		 client & c = (client&) clients.lockedData(i);
		 if(&c != &destination) // don't send the client itself
		 {
				File f;
				f.writeMem();
				f.putByte(M_ADD_CLIENT);
				f.putInt(c.id);
				c.writeDetailsToFile(f);
				f.pos(0);
				destination.connection.send(f);
		 }
	}
	clients.unlock(); 
}
\end{code}

Zoals je ziet wordt ook hier het `lock' mechanisme gebruikt. Ditmaal is er maar een bestemming: de nieuwe client. Voor alle andere clients maken we een file met de details van die clients. Die file sturen we naar de client `destination'. Aangezien de inhoud voor elke client anders is, kunnen we niet zoals bij de vorige functie telkens dezelfde file versturen. We maken dus een nieuwe file voor elke client. 

\begin{exercise}
Voeg alle code hierboven samen tot de class \eeClass{server} in het bestand met dezelfde naam.
\end{exercise}

\section{De Client Class}

Wanneer de server ontdekt dat een nieuwe client een verbinding aanvraagt, zal er automatisch een object van de class \texttt{client} gemaakt worden. Ook de update functie van die client zal door de server automatisch uitgevoerd worden. Tenzij je wil dat die client absoluut niets doet, zal je wel een eigen class voor die client moeten voorzien. Deze class heeft twee base classes nodig: enerzijds \textbf{moet} de class gebaseerd zijn op \texttt{ConnectionServer.Client} om dat ze anders niet de standaard client class kan vervangen, anderzijds willen we ook de shared base class \texttt{netClient} gebruiken.

\begin{code}
int NextClientID = 0;

class client : ConnectionServer.Client, netClient 
{
  bool sentHello = false; 
	// add code here
}
\end{code}

\begin{note}
Voor de eigenlijke class staat een \eeClass{int} `NextClientID'. Dat is een globale variabele die we gebruiken om elke speler een uniek ID te geven. Daarna begint de class. Alhoewel die leeg lijkt, bevat ze op dit moment reeds alle functies en variabelen van zowel \texttt{ConnectionServer.Client} als \texttt{netClient}. Een variabele die je alvast bovenaan in de class kan toevoegen is de bool `sentHello'. Die gebruiken we later in de update functie om te controleren of dit een nieuwe client is.
\end{note}

\subsection{De Create Functie}

Vervolgens moet er in deze class een create functie voorzien worden. De server class zal bij het maken van een nieuwe client automatisch proberen de create functie uit te voeren, maar die moet dan wel dezelfde argumenten hebben als de base class \texttt{ConnectionServer.Client}. Als je een create functie maakt zonder die argumenten, zal de server je functie niet gebruiken en gewoon de create functie van de base class uitvoeren.

\begin{code}
void create(ConnectionServer &server)
{     
	// each client needs his own unique ID
	id = NextClientID++;
	
	// send details for this client to other clients
	File f;
	f.writeMem().putByte(M_ADD_CLIENT).putInt(id);
	writeDetailsToFile(f);
	Server.sendToClients(f, T);
}
\end{code}

In deze functie kennen we de nieuwe client een ID toe. Dit is belangrijk omdat we later updates over deze client naar alle andere clients willen sturen. Die clients kunnen enkel weten over welke client het bericht gaat, wanneer elk van die clients een uniek nummer heeft. Daarom voorzien we een globale integer \texttt{NextClientID}. We kennen de huidige waarde toe aan de variabele \texttt{id} \textsl{(aanwezig in netClient)} en verhogen daarna de waarde van NextClientID.

Vervolgens willen we alle bestaande clients laten weten dat er een nieuwe client toegevoegd moet worden. We maken daarom een File waarin we de id van de client plaatsen, gevolgd door de `details' van deze client. De functie \texttt{sendToClients} die we aan de server class toevoegden kan nu gebruikt worden om de \texttt{File} te verzenden. We gebruiken \texttt{T} \textsl{(dit object)} als afzender, zodat de informatie niet naar deze client gestuurd zal worden.

\begin{note}
In praktijk zal je de gegevens van een client meestal pas naar andere clients sturen na de controle van een login en wachtwoord. Je zal deze code dan moeten verplaatsen.
\end{note}

\subsection{De Update Functie}

Bijna alles gebeurt verder in de update functie. Die is verantwoordelijk voor drie zaken:
\begin{itemize}
\tick indien de client nog geen connectie heeft, moet die opgezet worden;
\tick wanneer de client een bericht naar de server stuurt, moet dat bericht behandeld worden;
\tick wanneer de client de verbinding verbreekt, moet hij verwijderd worden.
\end{itemize}

Net zoals de create functie, is ook de update functie aanwezig in de base class. Om er voor te zorgen dat de server ze automatisch uitvoert, moet dit een bool functie zijn, zonder argumenten. Dit maal is het wel belangrijk dat ook de update functie van de base class uitgevoerd wordt. Daarvoor gebruiken we het keyword \texttt{super}:

\begin{code}
bool update() 
{
  if(super.update()) {
	  // still connected, do something
		return true;
	} 
	
	// connection is lost if this point is reached
	return false;	
}
\end{code}

Een \texttt{return true} laat de server weten dat deze client nog steeds actief is. Bij \texttt{return false} veronderstelt de server dat de client offline is, en wordt deze client verwijderd uit het geheugen.
	
Indien een client niet meer actief is, dan willen we dat aan de andere clients laten weten. Dat gebeurt net voor de lijn `return false'. Op die plaats moet dit bericht gemaakt en verstuurd worden:
\begin{code}
File f;
f.writeMem().putByte(M_REMOVE_CLIENT).putInt(id);
Server.sendToClients(f, T);
\end{code}

Zolang \texttt{super.update()} lukt, blijft de verbinding actief. Het eerste wat dan, bij een nieuwe verbinding, moet gebeuren is de client laten weten dat de verbinding geslaagd is. We hebben om deze reden de bool `sentHello' toegevoegd aan de class. Standaard is die false, wat wil zeggen dat dit een nieuwe client is. We controleren in dat geval of de status van de verbinding gelijk is aan \texttt{CONNECT\_GREETED}. Als dat zo is, dan sturen we een kort `Hello' bericht terug naar de client. Op dat moment informeren we de client ook over de andere clients die al actief zijn. \textsl{Hiervoor voegen we een functie \texttt{sendAllClientDetails()} toe aan de server class.}

\begin{code}
if(!sentHello)
{
	if(connection.state() == CONNECT_GREETED) // connection is ready for data
	{
		 File f;
		 f.writeMem().putByte(M_HELLO).pos(0);
		 connection.send(f);
		 sentHello = true;
		 
		 // send other clients' details to this client
		 Server.sendAllClientDetails(T);
	}
}
\end{code}

Vervolgens controleren we tijdens elke update of er nieuwe berichten binnenkomen. Als er data ontvangen is, dan zit die in de file `connection.data'. De eerste byte van elk bericht geeft aan over wat voor bericht het gaat, via een enumeratie `MESSAGE'. Voor elk van de berichten die een client kan versturen voorzien we dan een functie om het bericht te verwerken, waarbij we `connection.data' meesturen als bestand.

\begin{code}
REP(8) if(connection.receive(0)) // if data is recieved
{
	// first byte is type of message
	byte message = connection.data.getByte();
	
	// get the rest of the data
	switch(message)
	{
		 case M_CLIENT_FULL: handleFullUpdate(connection.data); break;
		 case M_CLIENT_POS : handlePosUpdate (connection.data); break;
	}
}
\end{code}

\subsection{HandlePosUpdate}
Nu dienen we nog de functies te maken om deze messages effectief te verwerken. We beginnen met de functie \eeFunc{handlePosUpdate}, een lidfunctie van de client class. 

Op het moment dat de functie uitgevoerd wordt, weten we dat de data die in het bestand zit, gelezen kan worden via \texttt{netClient.readPosFromFile()}. Dat moet dan ook eerst gebeuren. In het geval van deze update willen we dat de ontvangen positie verstuurd wordt naar alle andere clients. We maken dus nu op de server een nieuw bericht, bijna zoals het binnenkomende bericht. Met dit verschil dat ook de id van deze client opgenomen wordt. Zo weten de andere clients over welke client het gaat.

\begin{code}
void handlePosUpdate(File & data)
{
	readPosFromFile(data);
	
	// send to other clients
	File f;
	f.writeMem().putByte(M_CLIENT_POS).putInt(id);
	writePosToFile(f);
	Server.sendToClients(f, T);
}
\end{code}

Een bericht doorsturen naar andere clients is meestal niet moeilijker dan het voorbeeld hierboven. Soms zal er wel meer moeten gebeuren, zoals het aanpassen van gegevens in een database. In dat geval zal je de tijd moeten nemen om een nieuwe functie uit te schrijven.

\subsection{HandleFullUpdate}
Er is nog een andere message mogelijk: \eeFunc{handleFullUpdate}. De client zal dit bericht sturen als de speler zijn naam of kleur heeft aangepast. De behandeling van dit bericht is ongeveer gelijk aan de vorige functie: we lezen de binnenkomende data, en sturen ze door naar alle andere clients.

\begin{code}
void handlePosUpdate(File & data)
{
	readPosFromFile(data);
	
	// send to other clients
	File f;
	f.writeMem().putByte(M_CLIENT_POS).putInt(id);
	writePosToFile(f);
	Server.sendToClients(f, T);
}
\end{code}

\begin{exercise}
Voeg alle code hierboven samen tot de class \eeClass{client} in het bestand met dezelfde naam.
\end{exercise}

 
\chapter{De Client}

Wat de client betreft beperkt deze cursus zich tot de elementen die betrekking hebben op het netwerk-gedeelte. Andere classes zoals \texttt{detailsGui} zijn tijdens de cursus al meermaals aan bod gekomen. Je word verwacht deze classes zelf uit te werken.

\section{De `Peer' Class}
Met `Peers' bedoelen we andere clients die zich in je buurt bevinden. Gemakkelijkheidshalve gaan we er van uit dat alle actieve clients hier in mekaars buurt zijn. Bij grotere games zal naar de positie van een client gekeken worden. De server beslist dan welke speler dicht genoeg in mekaars buurt zijn om als peer beschouwd te worden.

De class \texttt{peer} heeft als base class \texttt{netClient}. Iedere \texttt{peer} heeft dus een positie, een kleur en een naam. Die worden enkel via het netwerk aangepast. 

Wat een \texttt{peer} class bijzonder maakt is het gebruik van interpolatie. Positie updates worden ongeveer 10 keer per seconde verstuurd. Maar om een vloeiende beweging op het scherm te tonen is een fijnere aanpassing van de posities nodig. Meer posities over het netwerk versturen is een optie, maar die belast het netwerk al snel te veel. Daarom gaan we de positiewijzigingen tussen de updates zelf invullen. Daar hebben we interpolators voor nodig. In dit geval is dat een \eeClass{Interpolator2} voor de interpolatie van de positie, een \eeClass{Vec2}. Bij een 3D project zou je een \eeClass{Interpolator3} gebruiken. Daarnaast heb je ook steeds een interpolator voor de tijd nodig: \eeClass{InterpolatorTime}. De lege class ziet er zo uit:

\begin{code}
class peer : netClient
{
   Interpolator2 iPos;
   InterpolatorTime iTime;
   
   // hier worden functies toegevoegd
}
\end{code} 

\subsection{Update}
De update functie zal eerst de \eeClass{InterpolatorTime} updaten. Daarna moet ook de positie geupdate worden, met de interpolatietijd als argument.

\begin{code}
void update()
{
	iTime.update();
	iPos.update(iTime);
}
\end{code}

\subsection{posities}
We voorzien nog twee extra functies om het werken met posities vlot te laten verlopen. De eerste is \eeFunc{recalculatePos}. Deze functie zullen we telkens uitvoeren wanneer we een nieuwe positie via het netwerk ontvangen. Ze zorgt ervoor dat de interpolators hun werk kunnen doen.

\begin{code}
void recalculatePos()
{
	iPos.step(pos, iTime);
	iTime.step();
}
\end{code}

De positie die we gebruiken om de peer op het scherm te tonen krijgen we via \eeFunc{iPos()}. Dit zou verwarrend kunnen zijn, want via \eeClass{netClient} bestaat er ook al een variabele `pos'. Om te voorkomen dat we ons vergissen, maken we een extra functie \eeFunc{getPos()}.

\begin{code}
Vec2 getPos()
{
	return iPos();
}
\end{code}

\subsection{Draw}
Tot slot heeft de peer class een \eeClass{Draw()} functie nodig. Hierin tekenen we een cirkel en een tekst op het scherm. We gebruiken de functie \eeFunc{getPos()} om de ge\"interpoleerde positie op te vragen.

\begin{code}
void draw()
{
	Circle(0.05, getPos()).draw(color);
	Vec2 textPos = getPos();
	textPos.y += 0.1;
	D.text(textPos, name);
}
\end{code}

\begin{exercise}
Werk de volledige \eeClass{peer} class uit aan de hand van de bovenstaande code.
\end{exercise}

\section{PeerManager}

Aangezien er meer dan \'e\'en peer actief kan zijn, maken we hiervoor een typische manager class. Die bevat een geheugencontainer om peers te onthouden, evenals functies om een peer toe te voegen, te verwijderen of te zoeken. Ook is er een functie voorzien om alle peers in een keer op het scherm te tekenen.

De lege class ziet er zo uit:

\begin{code}
class peerManager {
private:
  Memx<peer> peers;
	
public:
  // add other code
}

peerManager PeerManager;
\end{code}

\subsection{Peer toevoegen}
Je hebt een functie nodig om een nieuwe peer toe te voegen. Deze functie is ongeveer gelijk aan functies die je in het verleden al gebruikte om iets aan een manager class toe te voegen. Een klein verschil is dat je de nieuwe ID van de andere speler als functieargument gebruikt. We stellen in deze functie dadelijk het ID in van de nieuwe speler, maar geven ook een referentie naar dat nieuwe object terug als functieresultaat. Zo kan de code die deze functie gebruikt de peer verder aanpassen.

\begin{code}
peer & add(int ID)
{
	peer & p = peers.New();
	p.id = ID;
	return p;
}
\end{code}

\subsection{Peer vinden}
Er is ook een functie nodig om te zoeken naar een peer met een bepaalde ID. Deze keer geven we geen referentie maar een pointer als resultaat. Het is immers mogelijk dat de peer met een bepaald ID niet bestaat. Maar het is onmogelijk om een lege referentie als resultaat te geven. Een lege pointer kan wel, dat is de `null' pointer.

\begin{code}
peer * find(int ID) 
{
	FREPA(peers)
	{
		 if(peers[i].id == ID)
		 {
				return &peers[i];
		 }
	}
	
	return null;
} 
\end{code}

\subsection{Peer verwijderen}
Wanneer een speler offline gaat, dan moet die ook bij de andere clients verdwijnen. Daarom voorzien we een functie \eeFunc{remove}. In deze functie zullen we de speler met een gegeven ID uit de container verwijderen.

\begin{code}
void remove(int ID)
{
	FREPA(peers)
	{
		 if(peers[i].id == ID)
		 {
				peers.removeValid(i);
				return;
		 }
	}
}
\end{code}

\subsection{Peers tellen}
We willen ook weten hoeveel spelers er online zijn. Dat getal is gelijk aan het aantal elementen in de container met peers. Maar omdat die container private is, voorzien we ook een functie om deze informatie aan andere classes door te geven.

\begin{code}
int elms() 
{
	return peers.elms();
}
\end{code}

\subsection{Update en Draw}
Tot slot zijn er de update en draw functies. Die zullen de gelijknamige functies van elke peer uitvoeren:

\begin{code}
void update() { FREPA(peers) peers[i].update(); }
void draw  () { FREPA(peers) peers[i].draw  (); }
\end{code}

\begin{exercise}
Gebruik de bovenstaande code om de class peerManager uit te werken.
\end{exercise}

\section{Peer Messages}

Het bestand `peerMessages' bevat functies om data van het netwerk te verwerken die bedoeld is om peers aan te passen. Eventueel hadden deze functies ook in de class \texttt{peerManager} kunnen staan. Maar het is wel overzichtelijk om ze netjes samen in \'e\'en bestand te plaatsen.

\subsection{AddPeer}
\begin{code}
void AddPeer(File & f)
{
   int id = f.getInt();
   peer & p = PeerManager.add(id);
   p.readDetailsFromFile(f);
}
\end{code}

De functie \texttt{AddPeer()} haalt, zoals al deze functies, eerst het id uit het bestand. Vervolgens wordt een nieuwe peer gegenereerd die dan verder de details uit het bestand leest.

\subsection{GetPeerDetails}
\begin{code}
void GetPeerDetails(File & f)
{
   int id = f.getInt();
   // try to find a peer with this id
   peer * p = PeerManager.find(id);
   if(p != null)
   {
      p.readDetailsFromFile(f);
   }
}
\end{code}

In het geval van \texttt{GetPeerDetails()} is het mogelijk dat de functie \texttt{find()} null als resultaat heeft. Je moet dan ook controleren of dat zo is, voor je probeert de functie \texttt{readDetailsFromFile()} uit te voeren.

\subsection{GetPeerPos}
\begin{code}
void GetPeerPos(File & f)
{
   int id = f.getInt();
   peer * p = PeerManager.find(id);
   if(p != null)
   {
      p.readPosFromFile(f);
      p.recalculatePos();
   }
}
\end{code}

Deze functie lijkt sterk op de vorige. In feite zullen alle functies die gebruikt om informatie van het netwerk naar objecten over te brengen, in grote mate op mekaar lijken. In dit geval zullen we na het lezen van de positie ook de functie \eeFunc{recalculatePos()} uitvoeren. Die functie dient om de interpolators te updaten. (Zie uitleg in de vorige sectie.)

\subsection{RemovePeer}
Wanneer er een bericht binnenkomt om een peer te verwijderen, dan bevat dat bericht enkel de ID van die peer. In de class \eeClass{peerManager} hebben we een functie gemaakt die een peer verwijdert aan de hand van zijn ID. We kunnen die functie hier eenvoudig gebruiken.

\begin{code}
void RemovePeer(File & f)
{
   int id = f.getInt();
   PeerManager.remove(id);
}
\end{code}

\begin{exercise}
Voeg de 4 functies hierboven toe aan het bestand `peerMessages'.
\end{exercise}

\section{De Network Class}
Een tweede class controleert alle netwerk messages. De lege class ziet er zo uit:

\begin{code}
class network
{
private:
   Connection connection;
	 float startTime;
	 bool connected = false;
	
public:
   // ... more code ...
}
network Network;
\end{code}

Het belangrijkst hier is de class \texttt{Connection} die de engine voorziet. Die gebruik je om een verbinding te maken met een server. Je start deze verbinding in de functie \texttt{create()}.

\subsection{Create}
\begin{code}
void create()
{
	startTime = Time.curTime();
	SockAddr serverAddress;
	serverAddress.setIP("127.0.0.1", 65535);
	connection.clientConnectToServer(serverAddress);
}
\end{code}

De \texttt{startTime} stel je gelijk aan de huidige tijd. Dat is belangrijk om later te controleren hoe lang je al op een verbinding wacht. Vervolgens heb je een IP adres nodig. Dit voorbeeld gebruikt het locale adres van je computer, maar dat zal je uiteindelijk aanpassen naar een publiek IP adres. \textsl{(Bij een afgewerkt programma is het zelfs gebruikelijk om eerst ergens op een webserver een bestandje te downloaden dat het huidige IP adres bevat van de game server.)}

\subsection{Update}
De \texttt{update()} functie bestaat uit twee delen. Het eerste deel wordt uitgevoerd wanneer de verbinding nog niet in orde is. Het tweede deel kijkt, in het geval van een geldige verbinding, of er nieuwe messages zijn.

\begin{code}
void update()
{
   if(!connected)
	 {
	    // code
	 } else {
	    // more code
	 }
}
\end{code}

In het deel waarin er nog geen verbinding is, wordt eerst gekeken of er nieuwe messages zijn. Het eerste bericht dat je de server laat terugsturen is \texttt{M\_HELLO}. Als er een nieuw bericht is, dan controleer je of dat het juiste bericht is en wordt \texttt{connected} gelijk aan \texttt{true}.

Indien er geen verbinding is en er is ook geen nieuw bericht, dan is het tijd om enkele testen uit te voeren. Je controleert of de state van de connectie wel in orde is. In het geval die gelijk is aan \texttt{CONNECT\_INVALID} of \texttt{CONNECT\_VERSION\_CONFLICT} dan is het duidelijk dat de verbinding niet zal lukken. We kunnen dat laten weten aan de gebruiker. Ook als we langer dan 5 seconden wachten op een server kunnen we aannemen dat het niet meer in orde komt. 

\begin{code}
if(connection.receive(0))
{
	if(connection.data.getByte() == M_HELLO) connected = true;
} else
{
	// not connected yet, check for errors
	if(connection.state() == CONNECT_INVALID || connection.state() == CONNECT_VERSION_CONFLICT)
	{
		 Exit("Couldn't connect to server");
	}
	
	if(Time.curTime() - startTime > 5)
	{
		 Exit("Connection Timeout");
	}
	
	Time.wait(1); // wait a bit
}
\end{code}

Als er w\'el een verbinding is, dan controleren we ook op nieuwe messages. De eerste byte laat je weten om welk bericht het gaat. Daarmee beslis je welke functie je uitvoert en geef je de rest van het bestand \textsl{(connection.data)} door aan die functie:

\begin{code}
REP(8) if(connection.receive(0))
{
	byte message = connection.data.getByte();
	switch(message)
	{
		 case M_ADD_CLIENT   : AddPeer       (connection.data); break;            
		 case M_CLIENT_FULL  : GetPeerDetails(connection.data); break;               
		 case M_CLIENT_POS   : GetPeerPos    (connection.data); break;              
		 case M_REMOVE_CLIENT: RemovePeer    (connection.data); break;
	}
}
\end{code}



Tot slot is er ook nog de functie \texttt{send()}. Die kan je overal in je programma gebruiken om data naar de server te sturen. Veiligheidshalve wordt de leespositie van de file eerst terug op nul gezet.

\begin{code}
void send(File & f)
{
   f.pos(0); 
   connection.send(f);
}

bool isConnected() {
  return connected;
}
\end{code}

\begin{exercise}
Maak aan de hand van de bovenstaande code de class \eeClass{network} in het gelijknamige bestand.
\end{exercise}

\section{De Player Class}
Het programma bevat ook een class voor een player. Net zoals de \texttt{client} class bij de server, wordt ook hier \texttt{netClient} als basis gebruikt. Die bevat immers variabelen voor de naam, de kleur en de positie van de player.

Hier start je met de volgende class:

\begin{code}
class player : netClient {
private:
	float timeForUpdate = 0;
	
public:
  // add functions later
}
player Player;
\end{code}

\subsection{Create}
De create functie is in dit geval kort. We stellen enkel een standaard kleur in voor de speler.

\begin{code}
void create() {
  color = RED;
}
\end{code}

\subsection{Update}

In de \texttt{update()} functie staat eerst de gebruikelijke code om de speler te bewegen:

\begin{code}
if(Kb.b(KB_LEFT )) pos.x -= Time.ad();
if(Kb.b(KB_RIGHT)) pos.x += Time.ad();
if(Kb.b(KB_UP   )) pos.y += Time.ad();
if(Kb.b(KB_DOWN )) pos.y -= Time.ad();
Clamp(pos.x, D.viewRect().min.x, D.viewRect().max.x);
Clamp(pos.y, D.viewRect().min.y, D.viewRect().max.y);
\end{code}

Daarna volgt de code om de positie regelmatig naar de server te sturen. We doen dit elke 0.1 seconde. In dat geval wordt er een bericht van het type \texttt{M\_CLIENT\_POS} gegenereerd dat via de Network class naar de server verzonden wordt.

\begin{code}
if(timeForUpdate > 0)
{
	 timeForUpdate -= Time.ad();
} else {
	 // notify server of new position
	 File f;
	 f.writeMem().putByte(M_CLIENT_POS);
	 writePosToFile(f);
	 Network.send(f);
	 
	 timeForUpdate = 0.1;
}
\end{code}

\subsection{Draw}
De draw functie is eenvoudig. Net zoals in de \eeClass{peer} class tekenen we een cirkel en een tekst op het scherm:

\begin{code}
void draw()
{
	Circle(0.05, pos).draw(color);
	Vec2 textPos = pos;
	textPos.y += 0.1;
	D.text(textPos, name);
}
\end{code}

\subsection{setDetails}

Ook van belang is de functie \texttt{setDetails()}, die wordt uitgevoerd wanneer er op de `ok'-knop wordt gedrukt in de gui. Op het moment dat we de naam en de kleur van de speler wijzigen, maken we een bericht van het type \texttt{M\_CLIENT\_FULL} dat we naar de server sturen.

\begin{code}
void setDetails(C Str & name, C Color & color)
{
	T.name = name;
	T.color = color;
	
	// send details to server
	File f;
	f.writeMem().putByte(M_CLIENT_FULL);
	writeDetailsToFile(f);
	Network.send(f);
}
\end{code}

\begin{note}
In tegenstelling tot de server code voor het versturen van de berichten, voegen we nu niet het id van deze client toe. De server weet immers al van welke client dit komt.
\end{note}

\begin{exercise}
Voeg de bovenstaande code samen tot de class \eeClass{player}.
\end{exercise}

\section{Main Program Loop}

Een volwaardig programma zal waarschijnlijk uit meerdere application states bestaan. Om dit voorbeeld eenvoudig te houden is enkel de default state aanwezig. We overlopen even de verschillende functies.

\subsection{InitPre}

\begin{code}
void InitPre()
{
   EE_INIT();
   App.flag=APP_WORK_IN_BACKGROUND|APP_NO_PAUSE_ON_WINDOW_MOVE_SIZE;
}
\end{code}

De \eeFunc{InitPre} functie bevat een extra lijn code om enkele application flags in te stellen. Deze flags zijn opties die beinvloeden hoe het programma zich gedraagt. Om de werking van deze app te demonstreren zullen we verschillende instances van deze app gelijktijdig openen. We willen dat die allemaal het scherm updaten, niet enkel de applicatie met de window focus.

\subsection{Init}
\begin{code}
bool Init()
{
   Network   .create();
   Player    .create();
   DetailsGui.create();
   
   return true;
}
\end{code}

Er is niets bijzonders aan deze code. De verschillende objecten (netwerk, player en gui) worden ge\"initialiseerd.

\subsection{Shut}
\begin{code}
void Shut() {}
\end{code}

\subsection{Update}
\begin{code}
bool Update()
{
   if(Kb.bp(KB_ESC))return false;
   
   Network.update();
   
   // the program is not really active as long as the 
   // client is not connected to the server
   if(Network.isConnected())
   {
      if(Kb.bp(KB_F1)) DetailsGui.show();
      Gui.update();
      
      Player.update();
      PeerManager.update();
   }
   return true;
}
\end{code}

Hierin is enkel het volgende concept nieuw:

\begin{code}
if(Network.isConnected())
{
	// ...
}
\end{code}

Het \texttt{Network} object dat we hieronder uitwerken bevat een functie om te controleren of er een verbinding is. In dit voorbeeld gebruiken wordt het eigenlijke programma niet geupdated of getoond zolang dat niet het geval is. In een echt programma zal je waarschijnlijk wel een loginscherm willen tonen op dat moment.

\subsection{Draw}
\begin{code}
void Draw()
{
   D.clear(WHITE);
   
   // Don't do anything if not connected
   if(!Network.isConnected())
   {
      D.text(0, 0, "Waiting for Server...");
      return;
   }
   
   // draw peers and player on screen
   PeerManager.draw();
   Player.draw();
   
   // add texts and gui
   D.text(0, -0.9,  S + "Press F1 for options");
   D.text(0, -0.8,  S + "Connected players: " + PeerManager.elms());
   Gui.draw();
}
\end{code}

\begin{exercise}
Voeg alle code hierboven toe aan het bestand `main'.
\end{exercise}

\section{De Gui}
De gui voor dit project kan je zeker zelf uitwerken. Maak een gui Window met een \eeClass{TextLine} om je naam in te vullen, een \eeClass{Button} `color' die een \eeClass{ColorPicker} toont en een \eeClass{Button} `Ok' om de nieuwe waarden toe te kennen aan de Player.

Je zou met behulp van de code in de hoofdstukken over GUI zelf je gui class moeten kunnen schrijven. In de callback functie voor de button `Ok' voer je de functie \eeFunc{Player.setDetails()} uit. Als argument zet je daar natuurlijk de nieuwe naam en kleur.

\begin{exercise}
Werk de gui class uit.
\end{exercise}

\part{3D Worlds}
\chapter{Introduction}
In the previous chapters you have seen everything you need to create a simple 2D game. But how do you put a large project together in an orderly way? There is really no simple answer to that. You learn by practice, and not everyone agrees on the best method. Still, there are some rules that may make it easier for sure. And when you work in a group, the lead programmer will usually impose some rules that everyone has to follow. These are not necessarily good or bad rules, but they work as long as everyone follows them.

In this part of the course you will create a clone of the famous Tetris game. You will learn to develop a project step by step  without losing sight of the whole.

\begin{note}
Tetris consists of blocks which in turn consist of squares. When we talk about blocks in this course, we mean the entire block, not the squares it is made of. When a square is mentioned, we're discussing the squares that make up a block.
\end{note}

\section{Setup}
Open the `Tetris\_start' project. In it, you will find the graphics, sounds and fonts that we will use. A blank app `Tetris' is also provided, but we are not going to use it just yet.

\begin{enumerate}
	\item Create a library at the highest level of the explorer. You do this by right clicking and choosing `new library'. Name this library `Tetris parts'. A library is a green folder. The code in a library can be used from any application within your project, just like the library `Esenthel Engine' which is always present.
	\item Create a new application (blue folder). Call it `square tester'.
	\item in the application `square tester', create a code file `main'. 
	\item In the library `Tetris parts', create a new folder (yellow) called `definitions'.
	\item Mark `square tester' as the active application.
\end{enumerate}
	
Copy the code in `Tetris/initState' to `square tester/main'. Remove this line: 
	
\begin{code}
D.full(true);
\end{code}

This makes it easier to terminate your application when something goes wrong.	

\section{Constants}
In the previous chapter you learned about constants. It is a good idea to create some important constants before you start on the actual code. You can use them anywhere in your code and easily adjust their values if necessary. Create a new code file `constants' in the folder `Tetris parts/definitions'.

To be able to change the name of the app later on, create a constant \eeClass{Str} with a provisional name.

\begin{code}
C Str APP_NAME = "Tetris";
\end{code}

The size of the standard application window is not ideal for this game. A size in pixels is required for this. Declare a constant \eeClass{int} for this purpose.

\begin{code}
// The window size on the screen, in pixels
C int WINDOW_WIDTH  = 900;
C int WINDOW_HEIGHT = 800;
\end{code}

Tetris consists of rows and columns. we also define these:

\begin{code}
// This impacts the playing field
C int SQUARES_PER_ROW = 10;
C int ROWS            = 15;
\end{code}

It is also possible to add a few constants for the scoring system. There is a fixed number of levels, and we know how much points will be rewarded for a line and a level.

\begin{code}
// The scoring system uses these
C int POINTS_PER_LINE  =  525;
C int POINTS_PER_LEVEL = 6300;
C int NUM_LEVELS       =    5;
\end{code}

The speed of the game goes up with each level. This change can also be defined as a constant.

\begin{code}
// The speed will increase every level
C float INITIAL_SPEED = 1.0;
C float SPEED_CHANGE  = 0.1;
\end{code}

When playing tetris, there is a short period after moving a block down in which you are able to move it sidewards. This period has to be defined.

\begin{code}
// The time a block can be slided to the side
// When it hits bottom
C float SLIDE_TIME = 0.25;
\end{code}

We also have to determine the size of the playing field. This is done by defining the position of the lower left corner, and defining the size of the rectangle containing the playing field.

\begin{code}
// The area reserved for the playing field 
C Vec2 GAMEAREA     (-0.8, -0.8);
C Vec2 GAMEAREA_SIZE( 1.0,  1.4);
\end{code}

A new block will always appear at the top of the screen. This position can be deducted from information we already have:  SQUARES\_PER\_ROW and ROWS. In addition, there is a waiting position, top right of the playing field. You might notice we are not using \eeClass{Vec2}, but \eeClass{VecI2}. This is a vector which fits only integers. We do not want Tetris blocks midway between two positions, so there is no need for floats here.

\begin{code}
// Position for the current and next block
C VecI2 STARTPOS(SQUARES_PER_ROW / 2, ROWS - 1);
C VecI2 WAITPOS (SQUARES_PER_ROW + 4, ROWS - 3);
\end{code}

Now it is possible to calculate the size of a square with the information we already have. This has the advantage that we can modify the foregoing constants later, the size of a square being automatically adjusted.

\begin{code}
// The size of a square
C float SQUARE_SIZE = GAMEAREA_SIZE.x / SQUARES_PER_ROW;
\end{code}

\begin{note}
Of course it will rarely happen that you precisely know which constants are needed when you're just starting out on a project. In practice you will usually add a lot of constants while you are working on your project.
\end{note}

\section{Enums}
Create the code file `enumerations' in  `Tetris parts/definitions'. Add two enumerations that will be useful in your project. Firstly, there is the block type. Tetris has square blocks, T-blocks and so on. A list might look like this:

\begin{code}
enum BLOCK_TYPE
{
   BT_SQUARE     ,
   BT_T          ,
   BT_L          ,
   BT_BACKWARDS_L,
   BT_STRAIGHT   ,
   BT_S          ,
   BT_BACKWARDS_S,
   BT_NUM        , // number of block types used in the game
   BT_BACKGROUND , // special case, only for background
   BT_WALL       ,      
}
\end{code}

The last three values ​​deserve extra attention. The value `BT\_NUM' is useful because the number which represents that value is equal to the highest value + 1. This makes it easy to use a random function. Since the result of a random function does not include the maximum value, we will be able to use this kind of code later on in our application:

\begin{code}
blockType type = Random(BT_NUM);
\end{code}

So why is BT\_NUM not the last value in the list? Well, the last two values ​​are special cases. There will be special squares to draw the background and the borders of the game. And the color of such a square is determined by the block type. Because we do not want to use those types in the actual game, BT \_NUM is but before these values.

A second enumeration is used to determine the possible directions in which a block can move. A block can not upwards, only to the left, the right, or down. Blocks which are already down, do not have a direction anymore.

\begin{code}
enum DIRECTION
{
   D_LEFT ,
   D_RIGHT,
   D_DOWN ,
   D_NONE ,
}
\end{code}

\chapter{World Objects Manipuleren}
\section{Object Classes en Code}
\subsection{Object Class}
Standaard is elk object in de game wereld een onderdeel van het terein. Dat betekent dat dit object op het scherm getoond zal worden, maar dat interactie met dat object niet mogelijk is. Wanneer je met een object meer wil doen dan het simpelweg tonen, dan is er een connectie nodig tussen het object en je code. Die connectie zag je eigenlijk al in het vorige hoofdstuk: daar maakten we een koppeling tussen een \texttt{OBJ\_PLAYER} en een de class \eeClass{player}.

Je gaat steeds op de volgende manier te werk:
\begin{enumerate}
	\item Maak een Object Class in de editor.
	\item Open je object en kies de tab Params. Daar kan je de Object Class instellen. (Zie afbeelding \ref{fig:objClass3})
	\item Schrijf de code voor je nieuwe class.
	\item Maak een \eeClass{Game.ObjMap} voor deze class.
	\item Link de Object Map aan de World.
\end{enumerate}
	
\begin{figure}[h]
\centering
\includegraphics[width=0.8\linewidth]{../images/objClass3.png}
\caption[]{Een Object Class toewijzen.}
\label{fig:objClass3}
\end{figure}	

De eerste twee stappen spreken voor zich, dus we springen direct naar de derde stap.

\begin{note}
Start in dit hoofdstuk met de applicatie `3D World - stage 2'.
\end{note}

\subsection{Een custom object toevoegen}
\label{subsection:addObject}
Wanneer je de object class in de editor instelt, zal je zien dat het object niet meer in de wereld verschijnt als je je game start. Het object is immers geen onderdeel meer van het terrein, maar je hebt nog geen code geschreven om aan te geven wat er dan wel moet gebeuren. 

\begin{note}
Bij de game assets, in de map `other', vind je een object `magic lamp'. Er is ook al een object aanwezig in de game world. Dit object heeft de Object Class \texttt{OBJ\_MAGIC\_LAMP}. In deze oefening gaan we dat object rond zijn as laten draaien.
\end{note}

Wanneer je een class maakt voor een world object, dan moet je een speciale class als base class kiezen. Je hebt de keuze uit de volgende mogelijkheden:

\begin{itemize}
	\item \eeClass{Game.Animatable} gebruik je voor objecten met een animatie, zoals een kist die open en dicht kan.
	\item \eeClass{Game.Chr} gebruik je voor avatars, zoals de speler en mobs.
	\item \eeClass{Game.Destructible} gebruik je voor objecten die uiteen kunnen vallen in delen. (Op dit moment ontbreekt in de Editor een mogelijkheid om een dergelijk object te maken. Het kan wel via code, maar eenvoudig is dat niet.)
	\item \eeClass{Game.Door} gebruik je voor deuren.
	\item \eeClass{Game.Item} gebruik je voor voorwerpen waarmee interactie mogelijk is.
	\item \eeClass{Game.Static} gebruik je voor voorwerpen waarmee interactie mogelijk is, maar die op een vaste plaats blijven staan.
	\item \eeClass{Game.Kinematic} gebruik je voor voorwerpen waarmee interactie mogelijk is, maar die enkel via code verplaatst kunnen worden.
\end{itemize}

In deze oefening gebruiken we \eeClass{Game.Static}. Voorzie alvast de volgende class:

\begin{code}
class magicLamp : Game.Static {
	
}
Game.ObjMap<magicLamp> MagicLamp;
\end{code}

Pass in het bestand `main' de functie \eeFunc{Init()} aan:

\begin{code}
bool Init()
{
   Physics.create(EE_PHYSX_DLL_PATH);
   
   Game.World.activeRange(D.viewRange());
   Game.World.setObjType(Players, OBJ_PLAYER);
   Game.World.setObjType(MagicLamp, OBJ_MAGIC_LAMP); // <- New Code!
   Game.World.New(UID(2458107509, 1250513895, 1140948412, 2823156334));
   if(Game.World.settings().environment)Game.World.settings().environment->set();
   
   return true;
}
\end{code}

Wanneer je nu het programma uitvoert, dan zal je zien dat de lamp zichtbaar is. 

\subsection{Virtuele functies}
De reden waarom je een object class moet baseren op een van de beschikbare \eeClass{Game} classes, is dat deze classes virtuele functies bevatten (met het keyword \eeFunc{virtual}). Wanneer je de header file van \eeClass{Game.Static} bekijkt, dan zie je dat alle functies in de class virtueel zijn. Bovendien is \eeClass{Game.Static} gebaseerd op \eeClass{Game.Obj}, die ook nog eens heel wat virtuele functies bevat.

Deze functies zijn nodig omdat de engine je eigen class niet kent. Door je class te koppelen aan de game wereld in de Init functie hierboven, link je je class aan de World Manager. Maar die kan niet weten wat voor functies jij aan die class toevoegt. De World Manager zal daarom functies uitvoeren die w\'el bekend zijn, zoals \eeFunc{update()} en \eeFunc{drawPrepare()}.

Aangezien deze functies virtueel zijn, kan je ze overschrijven in een afgeleide class. Met andere woorden: de World Manager voert de \eeFunc{update()} functie van \eeClass{Game.Static} uit, tenzij je een functie met dezelfde naam voorziet in de afgeleide class. In dat geval wordt je eigen functie uitgevoerd.

\subsection{Rotatie}
We willen in deze oefening de lamp roteren rond zijn Y-as. Dat gebeurt door de rotatie aan te passen in de update functie. Je moet dus een eigen update functie toevoegen:

\begin{code}
class magicLamp : Game.Static
{
   virtual bool update()
   {
      return super.update();
   }
}
\end{code}

De functie verwacht een \eeClass{bool} als resultaat. We voeren in dit geval de \eeFunc{update()} functie van \eeClass{Game.Static} uit en geven dat resultaat als functieresultaat.

Om het object te roteren maken we gebruik van zijn \eeClass{Matrix}. Een \eeClass{Matrix} voor een object is een combinatie van drie vectoren: x, y en z om de rotatie van een object in 3D te bepalen. Daarnaast bevat de \eeClass{Matrix} ook nog een vector voor de positie. De schaal van het object kan afgeleid worden uit de grootte van de eerste drie vectoren.

Om een \eeClass{Matrix} te roteren, kan je gebruik maken van \eeFunc{rotatie} functies. In dit geval willen we een rotatie rond de Y-as, dus kan je de volgende code gebruiken:

\begin{code}
virtual bool update()
{
	Matrix m = matrix();
	m.rotateY(1 * Time.d());
	matrix(m);
	return super.update();
}
\end{code}

Deze code doet het volgende:
\begin{itemize}
	\item Maak een nieuw object van de class \eeClass{Matrix}, met de waarden van de huidige object matrix.
	\item roteer de matrix, rekening houdend met de tijdsdelta.
	\item Geef de object matrix de waarde van de geroteerde matrix.
\end{itemize}

Start je programma en controleer het resultaat. Je zal zien dat de lamp elke 5 seconden voorbijvliegt. Wat gaat er mis?

Als je een bewerking uitvoert op een matrix, dan is die van toepassing op alle onderdelen van die matrix, dus ook op zijn positie! We roteren dus ook de positie van het object in de 3D wereld. Ik heb je deze fout laten maken omdat je dit goed moet onthouden.

De oplossing bestaat er in de Matrix eerst te verplaatsen naar zijn nulpunt. Zowel schalen als roteren dient steeds op het nulpunt te gebeuren. Daarna verplaats je de matrix terug naar de oorspronkelijke positie:

\begin{code}
virtual bool update()
{
	Matrix m = matrix();
	m.move(-pos());
	m.rotateY(1 * Time.d());
	m.move(pos());
	matrix(m);
	return super.update();
}
\end{code}

De nodige stappen voor een rotatie zijn dus:
\begin{itemize}
	\item Maak een nieuw object van de class \eeClass{Matrix}, met de waarden van de huidige object matrix.
	\item Verplaats het object naar zijn negatieve positie.
	\item Roteer de matrix, rekening houdend met de tijdsdelta.
	\item Verplaats het object naar zijn oorspronkelijke positie.
	\item Geef de object matrix de waarde van de geroteerde matrix.
\end{itemize}

\begin{exercise}
\begin{enumerate}
	\item Zoek een functie om de matrix zowel op de X als de Y-as te roteren. Gebruik voor beide assen een andere rotatiesnelheid.
	\item Zoek een functie om de matrix te schalen. Laat het object langzaam groter en kleiner worden. Hint: je kan de functies \eeFunc{Sin()}, \eeFunc{Time.appTime()} en \eeFunc{Time.d()} gebruiken om de schaal te berekenen.
\end{enumerate}
\end{exercise}

\section{Draw Functies}
\subsection{De Renderer}

Om een 3D wereld te renderen, gebruiken we de volgende code:

\begin{code}
void Render() {
   Game.World.draw();
}

void Draw() {
   Renderer(Render);
}
\end{code}

De functie \eeFunc{Draw()} is je wel bekend. Maar waar we vroeger zelf objecten op het scherm tekenden, laten we nu de \eeFunc{Renderer()} zijn werk doen. (Je kan natuurlijk later nog steeds een Gui en 2D elementen over het resultaat van de Renderer tekenen.

Die renderer heeft een functie nodig om hem te vertellen wat hij moet doen. In dit geval is dat de functie \eeFunc{Render()}. Daarin geven we de renderer de opdracht om de game wereld te renderen. Ook hier kan je later extra code toevoegen als je game complexer wordt.

Een begrijpelijke misvatting is dat je er van uit gaat dat die wereld in \'e\'en keer wordt gerenderd. De \eeFunc{Draw()} functie roept de \eeFunc{Renderer()} aan, de \eeFunc{Renderer} voert de functie \eeFunc{Render()} uit, en die voert \eeFunc{Game.World.draw()} uit, niet? 	

Maar zo eenvoudig is het niet. Het renderen van een 3D beeld gebeurt in verschillende fasen, die allemaal hun eigen doel hebben. Zo zijn er fasen voor het renderen van alle objecten, het toevoegen van licht, het toevoegen van schaduw en nog veel meer. Een volledig overzicht vind je in de engine header Graphics\\Renderer:

\begin{code}
enum RENDER_MODE // Rendering Mode, rendering phase of the rendering process
{
   RM_SIMPLE        , // simple
   RM_EARLY_Z       , // early z
   RM_SOLID         , // solid
   RM_SOLID_M       , // solid in mirrors/water reflections
   RM_AMBIENT       , // ambient
   RM_OVERLAY       , // overlay mode for rendering semi transparent surfaces onto solid meshes (like bullet holes)
   RM_OUTLINE       , // here you can optionally draw outlines of meshes using 'Mesh::drawOutline'
   RM_BEHIND        , // here you can optionally draw meshes which are behind the visible meshes using 'Mesh::drawBehind'
   RM_FUR           , // fur
   RM_BLEND         , // alpha blending
   RM_SHADOW        , // shadow map    , render all shadow casting objects here using 'Mesh::drawShadow', if objects will not be rendered in this phase they will not cast shadows
   RM_STENCIL_SHADOW, // shadow stencil, render all shadow casting objects here using 'Mesh::drawStencilShadow', if objects will not be rendered in this phase they will not cast shadows
   RM_CLOUD         , // clouds
   RM_WATER         , // water surfaces
   RM_PALETTE       , // color palette #0 (rendering is performed using 'Renderer.color_palette'  texture)
   RM_PALETTE1      , // color palette #1 (rendering is performed using 'Renderer.color_palette1' texture)
   RM_PREPARE       , // render all objects here using 'Mesh::draw', and add all lights to the scene using 'Light*::add'

   RM_SHADER_NUM=RM_SHADOW+1, // all modes from RM_SIMPLE to RM_SHADOW are included in the 'MeshPart' shader technique lookup list
};
\end{code}

In elke fase zal de renderer alle objecten in de World Manager overlopen en kijken of er iets moet gebeuren voor dat object. Voor de objecten waar je zelf een class voor maakt, kan je de renderer vertellen wat er moet gebeuren.

Om dat te doen, moet je eerst de virtuele functie \eeFunc{drawPrepare} overschrijven. Deze functie bepaalt welke extra fases dat op dit object van toepassing zijn. Het resultaat van deze functie is een \eeClass{uint}, een unsigned integer. Omdat elke game class zelf al enkele fases voorziet, roep je eerst de \eeFunc{drawPrepare()} functie van de base class aan.

\subsection{Outline Rendering}
We zullen nu outline rendering toevoegen aan de lamp uit de vorige sectie. Voeg alvast deze functie toe aan de class \eeClass{magicLamp}:

\begin{code}
virtual uint drawPrepare()
{
	uint result = super.drawPrepare();
	\\ add your own rendering phases here
	return result;
}
\end{code}

\begin{exercise}
Voeg aan de class magicLamp ook een bool `selected' toe. In de update functie schrijf je code om te detecteren of de L-toets ingedrukt is. Als dat zo is, dan wordt selected true, in het andere geval is selected false.
\end{exercise}

Om outline rendering uit te voeren is het nodig om deze mode aan het resultaat toe te voegen. Dat kan door de volgende code aan drawprepare toe te voegen:

\begin{code}
if(selected) {
	 result |= IndexToFlag(RM_OUTLINE);
}
\end{code}

Maar dat is niet genoeg. We moeten nu ook de outline draw functie van \eeClass{Game.Static} overschrijven. Die functie bestaat, maar is leeg. Het is dus niet nodig om \eeFunc{super.drawOutline()} uit te voeren.

\begin{code}
virtual void drawOutline()
{
	mesh->drawOutline(BLUE, matrixScaled());
}
\end{code} 

\begin{exercise}
\begin{itemize}
\item Wat gebeurt er als je hierboven de functie \eeFunc{matrixScaled()} door \eeFunc{matrix()} vervangt? Waarom?
\item Maak gebruik van een \eeClass{Color} om de alpha waarde van de kleur te varie\"eren met de tijd. Hint: gebruik terug de functie \eeFunc{Sin()} in combinatie met \eeFunc{Time.appTime()}.
\end{itemize}
\end{exercise}

\subsection{Draw Behind}
Een andere render mode is \eeClass{RM\_BEHIND}. In deze sectie ga je zelf aan de slag om deze mode toe te passen op de Object Class \texttt{OBJ\_CHEST}. De stappenlijst vermeldt de hoofdzaken. De details kan je vinden in de secties hierboven.

\begin{enumerate}
	\item Controleer dat je assets de Object Class \texttt{OBJ\_CHEST} bevat.
	\item Open Assets $\Rightarrow$ other $\Rightarrow$ chest en kijk in de tab Params of de class correct ingesteld is.
	\item Open de World en zorg dat er ten minste \'e\'en chest in de wereld staat.
	\item Maak een class chest, gebaseerd op \eeClass{Game.Static}.
	\item Voorzie een \eeClass{Game.ObjMap} voor deze class.
	\item Link in de \eeFunc{Init()} functie deze object map aan \texttt{OBJ\_CHEST}.
	\item Voeg aan je class een bool selected toe.
	\item Voeg een update functie toe, waarin je selected true of false maakt naargelang de status van de C toets.
	\item Voeg een functie \eeFunc{drawPrepare()} toe, waarin je de render mode \texttt{RM\_BEHIND} toevoegt als selected true is.
	\item Voeg de functie \eeFunc{drawBehind()} toe. Daarin schrijf je de volgende code:
	
	\begin{code}
		mesh->drawBehind(Color(64, 128, 255, 255), Color(255, 255, 255, 0), matrixScaled()); 
	\end{code}
	
	\item Experimenteer met verschillende kleuren tot je een origineel resultaat krijgt.
\end{enumerate}

\section{Mousepointer to 3D}
\subsection{Voorbeeldcode}
hierboven `selecteerden' we een object door een toets in te drukken. Daarbij zag je dat alle objecten van een type geselecteerd werden. Wanneer je slechts \'e\'en object van een bepaalde class wil selecteren, dan zal je op een andere manier te werk moeten gaan. Je zou bijvoorbeeld de afstand tot de speler kunnen controleren, maar in dit deel bekijken we hoe je een object kan selecteren via de mouse pointer.

Om dat te doen moeten we als uitgangspunt de positie van de muis op het scherm nemen en een denkbeeldige lijn in de diepte van het scherm trekken. Wanneer een object deze lijn raakt, dan is er een selectie mogelijk. Deze methode is behoorlijk rekenintensief. Daarom is het geen goed idee om ze uit te voeren in de class van het object, want dan moet die lijn voor elk object van die class opnieuw gemaakt en gecontroleerd worden.

\begin{exercise}
\begin{enumerate}
	\item Open de applicatie `3D World - stage 3'. 
	\item Pas alvast de functie \eeFunc{update()} van de class \eeClass{magicLamp} aan. De bool `selected' dient tijdens elke update false te worden.
	\item Maak een class \eeClass{inputControl}, en daaronder een object van deze class.
	\item Voeg aan de class een functie \eeFunc{void update()} toe.
	\item Voer deze update functie uit in de main \eeFunc{Update()} functie. Het maakt niet zoveel uit waar je de functie plaatst, maar het moet in ieder geval na \eeFunc{Game.World.update()} gebeuren. (Die voert immers de update functie van elk object uit, en daar zet je selected steeds op false.)
\end{enumerate}
\end{exercise}

We beginnen met een `mouseover' effect. Dat wil zeggen dat we eenvoudigweg controleren of een muis over een lamp beweegt, zonder extra controles. Voeg in de update functie van inputControl de volgende code toe:

\begin{code}
   void update()
   {      
      // stop if a gui element is clicked
      if ((Gui.msLit() != null) && (Gui.msLit()->type() != GO_DESKTOP)) return;

      Vec pos, dir;      
      ScreenToPosDir(Ms.pos(), pos, dir);
			
      PhysHit selector;      
      if (Physics.ray(pos, dir * D.viewRange(), &selector))
      {
         if (magicLamp * lamp = CAST(magicLamp, selector.obj))
         {
            lamp.selected = true;
         }
      }
      
   }
\end{code}

Wat betekent dit allemaal? Eerst staat er deze regel:

\begin{code}
if ((Gui.msLit() != null) && (Gui.msLit()->type() != GO_DESKTOP)) return;
\end{code}

Op dit moment heeft je project nog geen Gui. Maar als dat wel zo zou zijn, dan wil je geen objecten selecteren als je bijvoorbeeld op een button klikt. Het probleem is dat onder die gui nog altijd je wereld zit, dus een mouseclick zou zonder deze regel zowel de gui als de selectie in de wereld be\"invloeden. Deze regel controleert eerst of er een Gui element actief is. Wanneer het type van dat element gelijk is aan \texttt{GO\_DESKTOP} dan klik je wat de gui betreft op een leeg element, met andere woorden de wereld. De regel hierboven zorgt ervoor dat de rest van de functie niet uitgevoerd wordt wanneer je niet op die desktop klikt.

\begin{code}
Vec pos, dir;      
ScreenToPosDir(Ms.pos(), pos, dir);
\end{code}

Hier maken we twee vectoren die nodig zijn om een denkbeeldige lijn door de wereld te trekken. We gebruiken de functie \eeFunc{ScreenToPosDir()} om de 2D positie van de muis om te zetten in een 3D startpositie en een richting waarin de lijn moet lopen.

\begin{code}
PhysHit selector;
\end{code}

De functie die de denkbeeldige lijn controleert, moet je laten weten wat er geraakt werd. Dat gebeurt via dit object.

\begin{code}
if (Physics.ray(pos, dir * D.viewRange(), &selector))
\end{code}

Hier vraag je aan de Physics engine om een lijn (ray) te trekken. Deze functie heeft 3 argumenten:

\begin{description}
	\item[pos] de startpositie.
	\item[dir] de richting waarin de lijn moet lopen. We willen geen oneindig lange lijn. Dat zou betekenen dat Physics oneindig ver moet blijven controleren op objecten wanneer er geen object op de lijn zit. Daarom vermenigvuldigen we de richting met de viewRange. De viewRange is de afstand tot waar objecten in je game zichtbaar zijn.
	\item[selector] dit is een referentie naar het \eeClass{PhysHit} object dat je hierboven maakte. Physics zal het eerste object dat op deze lijn staat doorgeven aan dit object.
\end{description}

\begin{code}
if (magicLamp * lamp = CAST(magicLamp, selector.obj))
\end{code}

Wanneer Physics iets gevonden heeft, dan zit er een object in de selector. Maar we weten nog niet welk object. Het zou kunnen dat het terrein werd gevonden, of eender welk object in je game. Daarom proberen we het object om te zetten (te `casten') naar een (pointer naar een) object van het de class \eeClass{magicLamp}. Als dat lukt, dan weet je zeker dat je met een object van het juiste type te maken hebt.

\begin{code}
lamp.selected = true;
\end{code}

Wanneer een lamp gevonden werd, dan wijzigen we de waarde van selected. Hierboven heb je tijdens elke update alle lampen de waarde \texttt{selected = false} gegeven. Door deze functie na de wereld update uit te voeren, wordt enkel de lamp waar nu de muis over zit, true.

\subsection{Het werkt niet!}
Wanneer je nu je app uitvoert, dan zal je merken dat de selectie niet werkt. Om te onderzoeken wat er fout gaat, kijk je in zo'n geval best de Physics na. Dat kan eenvoudig door een regel toe te voegen aan de \eeFunc{Draw()} functie:

\begin{code}
void Draw() {
   Renderer(Render);
	 if(Kb.b(KB_EQUAL)) Physics.draw();
}
\end{code}

Als je nu opnieuw je programma uitvoert en door de wereld loopt, dan zie je dat de contouren van alle Physics objecten zichtbaar zijn wanneer je op de = toets drukt. Deze code wil je niet in je uiteindelijke game, maar tijdens de ontwikkeling kan dit wel erg handig zijn. Loop nu opnieuw naar de lamp, en je zal zien dat de Physics ontbreken. 

Wanneer je een object importeert in Esenthel, dan wordt er niet vanzelf een Physics object gemaakt. Je kan dat wel eenvoudig zelf doen. Je opent het object in de editor en kiest de tab `Physics'. Rechts zie je de verschillende opties. Je kan zelf de meest geschikte vorm kiezen, maar denk er wel aan dat meer detail altijd meer rekenkracht vergt.

Eens je object van physics voorzien is, zal de bovenstaande code zonder problemen werken.

\begin{exercise}
	\begin{itemize}
		\item Voer de Physics ray enkel uit wanneer de L-toets ingedrukt werd.
		\item Pas de zoekafstand aan, zodat een lamp enkel gevonden wordt tot op 10 meter afstand.
	\end{itemize}
\end{exercise}







\chapter{Object Parameters}
De game world bevat ook enkele runestones. Tot hier toe zijn die niet zichtbaar in de game. Selecteer in de game world een runestone en inspecteer de Object Class. Een runestone heeft als class \eeClass{OBJ\_INTERACTIVE}. Ondertussen weet je dat je in je code een class moet voorzien om deze objecten te laden.

\begin{exercise}
	Begin deze oefening met de applicatie `3D World - stage 4'. Zorg dat de runestones zichtbaar zijn in je game. Je voegt de class \eeClass{interactiveObject} toe, samen met een \eeClass{ObjMap} en een extra regel bij het laden van de game world. Als je niet meer precies weet wat je moet doen, kijk dan in hoofdstuk \ref{subsection:addObject}.

	Open daarna de game en controleer of de runestones zichtbaar zijn.
\end{exercise}

Object classes in code zijn een enumeratie. Dat betekent dat je maximum 256 verschillende waarden kan gebruiken. Als je game wat groter wordt, dan is dat waarschijnlijk te weinig om alle objecten in je game een eigen object class te geven. Daarom zal je dikwijls een hele groep van objecten van dezelfde object class voorzien. Het onderscheid maak je dan met parameters.

\begin{exercise}
	\begin{enumerate}
		\item Voeg in de editor een nieuwe enumeratie toe in de folder \texttt{enums}, met de naam \eeFunc{INTERACTIVE\_OBJECT}.
		\item Voeg aan de enumeratie de waarde \eeFunc{RUNESTONE} toe.
		\item Open de object class \eeFunc{OBJ\_INTERACTIVE}. 
		\item Druk rechts op `New Param'.
		\item Kies als Type Enum$\Rightarrow$INTERACTIVE\_OBJECT
		\item Geef als Name `Type' in.
		\item Kies als Value `RUNESTONE'
	\end{enumerate}
\end{exercise}

\begin{figure}[h]
\centering
\includegraphics[width=0.8\linewidth]{../images/objClassParams}
\caption[]{Custom Parameters for an object class.}
\label{fig:objClassParams}
\end{figure}	

Je hebt nu een parameter toegevoegd aan je object class. In dit geval was dat een enumeratie, maar je ziet dat je eender welke variabele kan maken. In je code heb je toegang tot deze waarden, dus je kan ze gebruiken om het gedrag van je object aan te passen.

\begin{exercise}
	\begin{enumerate}
		\item Open Assets $\Rightarrow$ other $\Rightarrow$ runestone en kies de tab `Params'. De parameter type werd overgenomen van de object class.
		\item Voeg nog een parameter toe aan je runestone. Deze keer kies je het type `Int', Name `ID' en Value `0'.
		\item Open de world editor en kies het tabblad `Object'. Selecteer daarna een runestone.
		\item Controleer of elke runestone de gewenste parameters bevat.
		\item Pas de waarde van ID voor elke runestone aan, zodat ze de ID's van 0 tot 3 bevatten.
	\end{enumerate}
\end{exercise}

\begin{note}
Je ziet dat er een vinkje verschijnt voor een parameter wanneer je die een nieuwe waarde geeft. Daaraan zie je dat deze waarde afwijkt van de oorspronkelijke waarde. Het gebruik van parameters is ietwat gelijk aan het principe van overerving. Je kan parameters ingeven op het niveau van de object class, op het niveau van het object, of op het niveau van de instantie van het object. Elk niveau kan het vorige niveau overschrijven.

Het is ook mogelijk om alle parameters op het niveau van de instantie te schrijven. Dat betekent echter meer werk en een grotere kans op fouten.
\end{note}

\section{Retrieving Object Params in Code}

Om de object parameters in code te gebruiken, dien je de functie \eeFunc{create} van de base class \eeClass{Game.Static} te overschrijven. Daarin voer je eerst die base class functie uit. Om de parameters in code op te slaan voeg je ook twee variabelen toe. Je class ziet er nu zo uit.

\begin{code}
class interactiveObject : Game.Static
{
private:
   INTERACTIVE_OBJECT iObjType = IO_NONE;
   int ID = 0;
   
public:   
   virtual void create(Object &obj)
   {
      super.create(obj);
      
      // every interactive object has a Type parameter
      iObjType = obj.getParam("Type").asEnum();
      
      if(iObjType == IO_RUNESTONE)
      {
         // every runestone has an ID parameter
         ID = obj.getParam("ID").asInt();
      }
   }
}

Game.ObjMap<interactiveObject> InteractiveObjects;
\end{code}

Zoals je ziet heeft \eeClass{Object} een lidfunctie \eeFunc{getParam} die je toegang geeft tot de parameters. Aangezien de code niet weet welk type je parameter heeft, moet je zelf aangeven hoe je de parameter aan je code wil doorgeven. Dat kan via functies zoals \eeFunc{asEnum()} of \eeFunc{asInt()}.

\section{Particles}
Om de runestones wat meer uitstraling te geven voegen we aan elke runestone een particle effect toe. Voeg eerst de volgende variabelen toe aan de \eeClass{interactiveObject} class:

\begin{code}
Game.ObjParticles fx;
Vec fxPos;
\end{code}

Particles zijn kleine afbeeldingen waarvan je de beweging en de levensduur kan instellen. In een game worden ze vaak gebruikt om een vuur te tonen, of als visualisatie van een spell. In Esenthel is een particle een game object met als base class \eeClass{OBJ\_PARTICLE}. Wanneer je bij de parameters van een object deze base class instelt, krijg je automatisch toegang tot alle parameters van een particle.

\begin{note}
De object class \eeClass{OBJ\_PARTICLE} moet dan wel al die parameters bevatten. Als je aan een eigen project werkt, dan kan je deze class best copi\"eren van een voorbeeldproject. 
\end{note}

Het oefenproject bij dit hoofdstuk bevat al een uitgewerkte particle in Assets$\Rightarrow$other$\Rightarrow$runestone$\Rightarrow$particle. We zullen deze particle in het object fx laden. Voor we dat doen zullen we eerst een geschikte positie moeten vinden. Het is de bedoeling dat de particles zichtbaar zijn in de holle opening van de runestone. 

Met die reden is aan het runestone object een `slot' toegevoegd. Je kan dat nakijken door het object in de editor te openen en de tab `slots' te kiezen. Selecteer de optie om een positie aan te passen en hover over het aanwezige slot. Je zal zien dat dit slot een naam heeft: `particle'. Via deze naam kunnen we de positie van het slot opvragen. Aangezien we in de world editor ook elke runestone een andere schaal zouden kunnen geven, is het belangrijk dat deze positie geschaald wordt.

Ook belangrijk om te onthouden is dat het om een relatieve positie gaat. Het slot heeft geen positie in de wereld, maar een offset ten opzichte van de positie van de runestone. In dit geval komt dat goed uit. We geven fx ook de positie van de runestone, maar de bron van de particles moet relatief zijn ten opzichte van die positie. Als bron stellen we een \eeClass{Ball} in die de positie van het slot krijgt.

De volledige create functie ziet er dan zo uit:
\begin{code}
virtual void create(Object &obj)
{
  super.create(obj);
  
  iObjType = (INTERACTIVE_OBJECT)obj.getParam("Type").asEnum(); 
  if(iObjType == IO_RUNESTONE)
  {
     ID = obj.getParam("ID").asInt();
     
     if(mesh->skeleton().findPoint(8"particle") != null)
     {
        fxPos = mesh->skeleton().findPoint(8"particle").pos * scale;
        fx.create(*Objects(=== drop particle object here ===));
        fx.pos(pos());
        fx.particles.source(Ball(0.1, fxPos));
     } 
  }
}
\end{code}

\begin{exercise}
Voeg deze code toe aan je project.
\end{exercise}

Particles zijn bewegende afbeeldingen en moeten ook geupdated worden. Daarvoor heeft de class \eeClass{Game.ObjParticles} een update functie. Het volstaat om deze functie uit te voeren. Maar omdat de \eeClass{interactiveObject} class niet enkel voor runestones dient, doen we dit enkel wanneer \eeClass{iObjType} een runestone is:

\begin{code}
virtual Bool update()
{
  if(iObjType == IO_RUNESTONE)
  {
     // update the particles
     fx.update();
  }
  
  return super.update();
}
\end{code}

Particles worden gerendered in een afzonderlijke draw functie. Afhankelijk van het soort particles heb je \'e\'en van de volgende render modes nodig:

\begin{itemize}
	\item RM\_BLEND
	\item RM\_PALETTE
	\item RM\_PALETTE1
\end{itemize}

In de \eeFunc{drawPrepare} functie moet je daarom de gewenste render mode toevoegen. In dit geval gebruiken we RM\_PALETTE.

\begin{code}
virtual UInt drawPrepare()
{     
  uint result = super.drawPrepare();  
  if(iObjType == IO_RUNESTONE)
  {
     result |= IndexToFlag(RM_PALETTE);
  } 
  return result;
}
\end{code}

Tot slot moet je ook de functie \eeFunc{drawPalette} overschrijven:
\begin{code}
virtual void drawPalette()
{
   fx.drawPalette();
}
\end{code}

\begin{exercise} 
Voeg alle bovenstaande code toe aan je project. Voer je game uit en controleer of je de particles op het scherm ziet.
\end{exercise}

\section{Licht}
Een andere techniek die de game wat meer sfeer kan geven is het toevoegen van lichtbronnen. Hier moet je wel voorzichtig mee omgaan. Voor elke lichtbron moeten immers ook schaduwen berekend worden. Een scene met veel lichtbronnen kan bijzonder zwaar zijn voor een oudere computer. Ook mobile games zijn meestal beperkt tot \'e\'en enkele bron.

Een lichtbron toevoegen is alvast heel eenvoudig. Je hebt enkel een positie en een kleur nodig. (Al zijn er extra argumenten om ook de intensiteit en de maximum afstand te bepalen, maar hier gebruik je de defaults.)

Je voegt deze code toe aan de \eeFunc{drawPrepare} functie:
\begin{code}
if(iObjType == IO_RUNESTONE)
{
   LightPoint(1, pos() + fxPos, GREEN.asVec()).add();
   result |= IndexToFlag(RM_PALETTE);
}
\end{code}

\begin{exercise}
Experimenteer met de instellingen van de lichtbron. Wijzig ook de parameters van het particle object, zodat je een beter zicht krijgt op de betekenis van de parameters.
\end{exercise}

\section{Toggle the Runes}
Op dit moment zijn de particles en het licht steeds actief. Nu ga je er voor zorgen dat je ze in en uit kan schakelen door er op te klikken. Een groot deel van deze code ken je al uit de vorige secties. Zo zullen we eerst een outline toevoegen wanneer je muis over een runestone hovert.

\begin{enumerate}
	\item Voeg eerst deze variabelen toe aan de \eeClass{interactiveObject} class:

\begin{code}
bool belowMouse = false;
bool active = false;
\end{code}
	\item In de \eeFunc{update} functie zorg je er voor dat \eeFunc{belowMouse} steeds false wordt.
	\item Je voegt daarna de outline render mode toe in \eeFunc{drawPrepare}, maar enkel waneer \eeFunc{belowMouse} de waarde true heeft.
	\item zorg er ook voor dat het \eeClass{LightPoint} en de particles enkel getoond worden wanneer `active' true is.
	\item Nu overschrijf je de \eeFunc{drawOutline} functie. Als je niet zeker bent, kan je spieken in de \eeClass{magicLamp} class.
	\item Tot slot voeg je enkele eenvoudige functies toe die je later van pas zullen komen. Je ziet hieronder de declaraties. Je kan deze functies (telkens precies \'e\'en statement) zeker zelf uitwerken.

\begin{code}
   bool isRuneStone   () { ???; }  
   void setBelowMouse () { ???; }  
   void toggleActive  () { ???; }
   void deactivate    () { ???; }
   bool isActive      () { ???; } 
   int  getID         () { ???; }   
   Vec  getParticlePos() { ???; }
\end{code}

\end{enumerate}

In de class \eeClass{inputControl} kan je nu een extra controle op de physics selector uitvoeren:

\begin{code}
if (interactiveObject  * obj = CAST(interactiveObject, selector.obj))
{
	if(obj.isRuneStone())
	{
	   obj.setBelowMouse();
	   if(Ms.bp(0))
	   {
	      obj.toggleActive();
	   }
	}
}
\end{code}

\begin{exercise}
Wanneer je alle code toegevoegd hebt, voer je de game uit en controleer je of alles werkt.
\end{exercise}


\section{A Puzzle}
In het laatste deel van dit hoofdstuk maak je een puzzel. Het is de bedoeling de 4 runestones in de juiste volgorde activeert. Als ze alle vier actief zijn, komt het hoofd in het midden naar boven. Ook toon je een laser effect tussen de actieve runestones. Je start met een nieuwe class \eeClass{puzzle}:

\begin{code}
class puzzle {
private:
	Mems<interactiveObject *> objects;
	Mems<Vec> laserPoints;
	Sound sound;

public:

}
puzzle Puzzle;
\end{code}

De \eeClass{objects} container bevat pointers naar de aanwezige runestones in de game wereld. \eeClass{laserPoints} zal de punten bevatten waartussen en laser getoond moet worden. En tot slot is er een \eeClass{sound} om alles wat spannender te maken.

Je voegt nu in het private deel van de class de volgende functies toe:

\begin{code}
interactiveObject * getObjectWithID(int ID)
{
  REPA(objects)
  {
     if(objects[i].getID() == ID) return objects[i];
  }
  
  return null;
}

void turnOffStones()
{
  REPA(objects) objects[i].deactivate();
}
\end{code}

Deze functies maken de volgende code eenvoudiger:

\begin{description}
	\item[getObjectWithID] zoek de runestone met het gevraagde ID. De volgorde in de container is immers niet gelijk aan de volgorde van de ID's.
	\item[turnOffStones] deactiveert alle runestones in de container.
\end{description}

De volgende functies die je aanmaakt zorgen er voor dat je runestones kan toevoegen en verwijderen uit de container. Je moet er namelijk rekening mee houden dat Esenthel oneindig grote werelden ondersteunt. Dat betekent dat niet de hele wereld steeds in het geheugen zit. World Objects kunnen geladen en terug verwijderd worden terwijl je de speler verplaatst doorheen de game world.

\begin{code}
void registerStone(interactiveObject * obj)
{
  objects.add(obj);
}

void unregisterStone(interactiveObject * obj)
{
  REPA(objects)
  {
     if(objects[i] == obj)
     {
        objects.remove(i);
        return;
     }
  }
}
\end{code}

Nu komt het er op aan om deze functies op de juiste plaats te gebruiken. We laten elke runestone zichzelf registeren bij de puzzel. Dat kan het best in de \eeFunc{create} functie van interactiveObject. Zoek zelf even naar de meest geschikte plaats binnen deze functie en voeg de volgende code toe:

\begin{code}
Puzzle.registerStone(this);
\end{code}

Het object moet zich ook verwijderen uit de puzzel wanneer het niet langer actief is. Nu bestaat er geen functie binnen Game.Static die uitgevoerd wordt wanneer dit het geval is. Maar je kan wel kijken wat het resultaat is van update. Bij `false' zal de game engine het object verwijderen. Het komt er dus op aan om net ervoor de runestone uit de puzzel te verwijderen. Op het eind van de update functie schreven we tot hiertoe steeds:

\begin{code}
return super.update();
\end{code}

\begin{exercise}
Herschrijf de bovenstaande regel zo dat je het resultaat van \eeFunc{super.update()} kan gebruiken om het object te verwijderen als dat nodig is. 
\end{exercise}

De echte logica van de puzzel zit in de update functie. We overlopen stap voor stap wat hier moet gebeuren. Eerst wordt de container met punten voor de laser leeggemaakt. En wanneer er geen 4 runestones geladen zijn heeft het in elk geval geen zin om de puzzel te controleren.

\begin{code} 
void update() {
  laserPoints.clear();
  if(objects.elms() != 4) return;
}
\end{code}

Vervolgens controleren we de huidige stand van zaken. We kunnen er van uit gaan dat je nooit twee runestones op hetzelfde moment kan inschakelen. Daarom is het mogelijk om de volgende logica toe te passen:
\begin{enumerate}
	\item Stel de hoogst actieve ID gelijk aan nul.
	\item Ga alle runes af, beginnend met het hoogste ID.
	\item Indien dat ID actief is en de huidige hoogst actieve ID kleiner dan deze, wordt dit de hoogst actieve ID.
	\item Indien de ID kleiner is dan de hoogst actieve ID en niet actief is, dan is de puzzel ongeldig en stop je de loop.
\end{enumerate}

De code ziet er zo uit:
\begin{code}
int highestActive = 0;
bool invalid = false;
for(int i = 3; i >= 0;  i--)
{
  if(highestActive < i && getObjectWithID(i).isActive())
  {
     highestActive = i;
  } else if(highestActive > i && !getObjectWithID(i).isActive())
  {
     invalid = true;
     break;
  }
}
\end{code}

In de volgende stap zijn er twee mogelijkheden: Invalid kan true of false zijn. Wanneer de puzzel ongeldig is, dan kunnen we eenvoudigweg alle stenen uitschakelen. Dat zorgt er voor dat de speler terug opnieuw moet beginnen.

In het andere geval moet je de posities van de actieve stenen toevoegen aan de laserPoints container, maar enkel wanneer de hoogst actieve positie groter is dan 0. (Met slechts 1 actieve positie kan je toch geen lijn maken.)

Wanneer alle stenen actief zijn, dan voeg je op het eind nogmaals de eerste positie toe, zodat de `cirkel' gesloten is.

\begin{code}
if(invalid)
{
  turnOffStones();
} else if(highestActive > 0)
{
  // add points to laser
  for(int i = 0; i <= highestActive; i++)
  {
    laserPoints.add(getObjectWithID(i).getParticlePos());
  }
 
  // close trajectory if all stones are on
  if(highestActive == 3)
  {
    laserPoints.add(getObjectWithID(0).getParticlePos());
  }
}
\end{code}

Op het eind van de update functie kan je zelf code toevoegen de nodige geluiden te spelen. Je vertrekt van de volgende regels:

\begin{itemize}
	\item Wanneer laserPoints ten minste \'e\'en element bevat en \eeClass{sound} nog niet speelt, dan start je het geluid `erie\_ring' in een loop.
	\item Wanneer laserPoints leeg is en het \eeClass{sound} w\'el speelt, dan stop je het geluid met een fadeout, en je speelt het geluid `roar' (zonder loop, via de functie \eeFunc{SoundPlay}).
\end{itemize}

De laatste twee functies in deze class zijn weer eenvoudig. De \eeFunc{draw} functie toont de laser op het scherm en de \eeFunc{solved} functie laat weten of de puzzel al dan niet opgelost is.

\begin{code}
void draw()
{
  if(laserPoints.elms() > 0)
  {
     DrawLaser(GREEN, WHITE, 0.01, 0.03, false, laserPoints);
  }     
}

bool solved()
{
  return laserPoints.elms() == 5;
}
\end{code}

Omdat Puzzle geen deel uitmaakt van de Game World en toch een draw functie heeft, moet je die zelf uitvoeren. De declaratie van DrawLaser vertelt je dat er twee render modes nodig zijn: RM\_SOLID en RM\_AMBIENT. Je voegt die toe aan de \eeFunc{Render} functie in het bestand `main'.

\begin{code}
void Render()
{
   Game.World.draw();
   
   switch(Renderer())
   {
      case RM_SOLID:
      {
         Puzzle.draw();
         break;
      }
      
      case RM_AMBIENT:
      {
         Puzzle.draw();
         break;
      }
   }
}
\end{code}

Tot slot is het de bedoeling om het hoofd tussen de runes te tonen wanneer de puzzel opgelost is. Het hoofd wordt ook geladen als \eeClass{interactiveObject}, maar heeft als \eeClass{iObjType} IO\_HEAD.

Voeg eerst een variabele toe aan \eeClass{interactiveObject}.

\begin{code}
float origHeadPosY;
\end{code}

Nu kan je die oorspronkelijke Y positie van het hoofd onthouden in de \eeFunc{create} functie, om daarna het hoofd 2 units naar beneden te plaatsen:

\begin{code}
if(iObjType == IO_HEAD)
{
  origHeadPosY = pos().y;
  Vec newPos = pos();
  newPos.y -= 2;
  pos(newPos);
}
\end{code}

Daarna kan je in de update functie het hoofd verplaatsen wanneer nodig:

\begin{code}
if(iObjType == IO_HEAD)
{
  if(Puzzle.solved() && pos().y < origHeadPosY)
  {
     your code here!
  }
  else if(!Puzzle.solved() && pos().y > origHeadPosY - 2)
  {
	 .. and here!
  }
}
\end{code}

\begin{exercise}
Vul de bovenstaande code verder aan en test of alles werkt.
\end{exercise}

\chapter{Animations}
De class \eeClass{Game.Chr} voorziet een aantal standaard animaties. Die zijn echter zelden genoeg. In dit hoofdstuk zie je hoe je deze animaties aanpast, en hoe je nieuwe animaties toevoegt.

Esenthel maakt een onderscheid tussen standaard animaties and custom animaties. De standaard animaties worden automatisch toegepast aan de hand van enkele variabelen in \eeClass{Game.Chr}. Als je de header file van deze class opent, dan zie je binnen de class een struct \eeClass{Input}. Deze class regelt hoe je avatar over het scherm beweegt. Je ziet properties zoals crouch, walk, jump en dodge.

Het is eenvoudig om deze waarden aan te passen. Je hebt dat trouwens al gedaan in een van de vorige hoofdstukken. Denk aan code zoals:

\begin{code}
input.turn.x = Kb.b(KB_Q) - Kb.b(KB_E);
input.turn.y = Kb.b(KB_T) - Kb.b(KB_G);
input.move.x = Kb.b(KB_D) - Kb.b(KB_A);
\end{code}

\begin{exercise}
De avatar heeft, naast een run animatie, ook een walk animatie. Die wordt op dit moment niet gebruikt. Pas de code in \eeFunc{player.update} aan zodat je avatar standaard wandelt, maar wel loopt wanneer de linker ctrl toets ingedrukt is. Zoek in de struct \eeClass{Input} op welke property je hier voor aanpast.
\end{exercise}

\section{Jump}
Vreemd genoeg bevat \eeClass{Input} wel een property jump die de avatar tijdelijk omhoog beweegt, maar bestaat er geen standaard animatie voor jump. Dat wil zeggen dat je een custom animatie moet gebruiken. Om dit te doen heb je een \eeClass{Motion} object nodig. Voeg alvast dit object toe aan de class \eeClass{Player}.

\begin{code}
Motion jumpMotion;
\end{code}

In de functie \eeFunc{update} kan je de huidige `jump' code aanvullen:

\begin{code}
// jumping
input.jump = Kb.bp(KB_SPACE) ?  3.5 :  0;
if(Kb.bp(KB_SPACE))
{
  jumpMotion.set(skel, === drop jump animation ===);
}
jumpMotion.updateAuto(5, 5, 1);
\end{code}

Het eerste statement stond al in je code. Dat bepaalt of je avatar al dan niet even omhoog gaat. Daarna stel je de animatie in wanneer de speler op de spatiebalk drukt. Het eerste argument is het skeleton waarop de animatie van toepassing is. De waarde `skel' is skeleton dat al aanwezig is in \eeClass{Game.Chr}. Het tweede argument is de verwijzing naar de animatie. Drop daar de jump animatie die je vindt in Assets $\Rightarrow$ characters  $\Rightarrow$ samurai.

Tijdens elke update moeten ook alle \eeClass{Motion} object geupdated worden. In dit geval is dat jumpMotion. Je kan de \eeFunc{updateAuto} functie gebruiken om het eenvoudig te houden. De eerste twee argumenten bepalen hoe snel je overschakelt van de standaard animatie naar de jump animatie en omgekeerd. Het derde argument is de algemene snelheid waarmee de animatie getoond wordt.

Wanneer je nu de game uitvoert, zal je zien dat de jump animatie niet gebruikt wordt. We moeten eerst de aanwezige functie \eeFunc{animate} van \eeClass{Game.Chr} overschrijven. Die functie voert nu enkel de standaard animaties uit. Omdat dit een virtuele functie is (kijk in de header file!) moet je daarin ook de oorspronkelijke functie uitvoeren. \textit{(Tenzij je echt niet wil dat de standaard animaties gebruikt worden.)} Voeg daarom de volgende functie toe aan je player class:

\begin{code}
virtual void animate()
{
  super.animate();
  skel.animate(jumpMotion, true);
}
\end{code}

Deze functie voert eerst de standaard animaties uit. Daarna wordt de \eeClass{jumpMotion} toegevoegd. Het extra argument `replace' bepaalt dat de vorige animaties overschreven moeten worden. Wanneer je `false' gebruikt, zal de impact van de vorige animaties veel groter zijn.

\begin{exercise}
Pas de waarden in \eeFunc{updateAuto} aan een bekijk het resultaat. Bekijk ook hoe anders de animatie is wanneer je \eeFunc{skel.animate(jumpMotion, false)} gebruikt.
\end{exercise}


\chapter{Dynamic Objects}
In dit hoofdstuk leer je hoe je met tijdelijke items in je game world omgaat. In het algemeen heb je hier twee mogelijkheden:

\begin{enumerate}
	\item items die altijd aanwezig, maar slechts af en toe zichtbaar zijn.
	\item items die je zelf aan de game world toevoegt via code en nadien ook terug verwijdert.
\end{enumerate}

Welk van deze mogelijkheden je gebruikt, hangt af van de situatie. Wil je bijvoorbeeld een power-up op een bepaalde locatie die verdwijnt wanneer de speler hem gebruikt, maar enkele minuten later terug verschijnt, dan gebruik je de eerste optie. Maar wanneer je een voorwerp op eender welke locatie kan opnemen en terug achterlaten, dan is de tweede optie beter.


\input{nl/3DWorlds/inventory}

\printindex

\end{document}

