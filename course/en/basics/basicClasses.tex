\chapter{Classes: the basics}

This course assumes you have some basic knowledge about object oriented programming, and classes are a crucial part of that. In this chapter, the basics of classes will be reviewed, but a little advance knowledge might be required if you're a slow learner.

If data belongs together, this should be reflected in your code. For example: the class \eeClass{Vec2} has two float variables to remember a position: x and y. Without this class it would still be possible to work with 2D coordinates, but you would have to declare two float variables for one position. 

The designer of the game engine knew that 2D positions would be used rather often. So it made sense to provide a class for that. A very simple \eeClass{Vec2} class could look like this:

\begin{code}
class Vec2 {
	float x;
	float y;
}  
\end{code}

When this class exists, it can be used anywhere in your project:

\begin{code}
Vec2 pos;
Vec2 pos2 = pos; // Assign the values of pos to pos2
pos.x = 3;       // Adjust the x value of pos
pos.y = pos2.x;  // Assign the x values of pos2 to the y value of pos
\end{code}

The same can be done with your own classes. In fact, almost every line of code in your project should be part of a class.

\begin{note}
When two or more objects or variables belong together, they should be defined as a class.
\end{note}

There is a difference between a class and an object. In the example above, \eeClass{Vec2} is a class, but pos and pos2 are objects of the class \eeClass{Vec2}. In other words: you cannot 'use' a class directly, but you can create objects with it and use those.

\begin{code}
Vec2.x = 3; // Very wrong!
Vec2 pos;
pos.x = 3; // Correct.
\end{code}

\section{Roll your own Class}
Suppose you need a moving circle in your project. To have this circle move at a fixed speed, you need a variable. For instance a \eeClass{Vec2} called 'speed'. This \eeClass{Vec2} obviously belongs with the circle, but the class \eeClass{Circle} does not provide such one. And it should not, because not every circle needs to move. 

This is a very good reason to create your own class for moving circles, containing a \eeClass{Circle} and a \eeClass{Vec2}:

\begin{code}
class movingCircle {
  Circle circle;
  Vec2 speed;
}

movingCirle mc;

// in the Init function
mc.circle.set(0.1, Vec2(-0.4, -0.3));
mc.speed = Vec2(0.3, 0.5);

// in the Update function
mc.circle.pos += mc.speed * Time.d();
\end{code}

Once you've created this class \eeClass{movingCirle}, you'll be able to create as many moving circles as you like, all with their own speed. (But using them looks rather complicated, right? You'll soon learn how to simplify this.)

\begin{exercise}
\begin{enumerate}
  \item Create a container for the class \eeClass{movingCircle}. Every time the space bar is pressed, add an object to this container. The new object should have a random position and a random speed.
	\item In the Update function, you must update the position for every circle. When a circle move outside of the screen, it should be put back in the middle.
	\item Extend the class \eeClass{movingCirle} with a color. Every object should have a random color. Be sure to draw every circle with its custom color.
	\end{enumerate}
\end{exercise}

\begin{note}
More work will be done on this exercise in the next chapter. Save it!
\end{note}
