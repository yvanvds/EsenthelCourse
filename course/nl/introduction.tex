\chapter{Inleiding}

Tegenwoordig hoor je veel over Java, C\#, php en andere moderne programmeertalen. Dikwijls met de belofte dat ze de productiviteit verhogen of het de programmeur op een of andere manier eenvoudiger maken. Toch blijft C++ na al die jaren \'e\'en van de belangrijkste programmeertalen, en met reden: de snelheid van de gecompileerde code is een veelvoud van zelfs de meest geoptimaliseerde java of php code; er bestaan libraries voor zowat alles dat je kan bedenken; source code voor nieuwe hardware is steevast een combinatie van C/C++. 

Maar zelfs al zou je uiteindelijk in een andere taal werken, kennis van C++ is nooit verloren. De syntax van de meeste andere programmeertalen is sterk gebaseerd op het taalidioom van C. Je kan met andere woorden je C++ kennis ook gebruiken om met andere talen te werken.

In deze cursus gaan we er van uit dat je de basis van de taal al beheerst. Je weet wat functies, pointers en classes zijn. Je kan een eenvoudig programma schrijven dat tekst in een console venster toont. Al is dat natuurlijk niet dadelijk het soort applicatie waar je echt zinvolle dingen mee doet.

De taal C++ geeft je de structuur die je nodig hebt om je ide\"en te formuleren. Maar om je programma ook werkelijk zinvolle dingen te laten doen heb je libraries nodig: bibliotheken met code die al ontwikkeld is door anderen en die je toelaten om bijvoorbeeld een afbeelding te genereren, een GUI te tonen, verbinding te maken met een netwerk, etc. 

Als softwareontwikkelaar is het belangrijk dat je je een library eigen kan maken. Deze cursus neemt als voorbeeld een library voor game development, maar daar gaat het eigenlijk niet om. Veel belangrijker is dat je binnen die library zelfstandig kan vinden wat je nodig hebt voor de uitwerking van je software. Die vaardigheid kan je dan later toepassen op eender welke library die je nodig zou hebben.

\section{Esenthel Installeren}

Esenthel is geen library in de enge zin van het woord. Het pakket bevat onder meer een library, een editor, convertors voor graphics en andere tools die je kunnen helpen bij de ontwikkeling van een game. Ook bevat het pakket een pre-compiler, die het programmeren van C++ code een beetje eenvoudiger maakt. Zo moet je geen header files schrijven, zijn includes overbodig en worden references automatisch omgezet naar pointers als dat nodig is.

Om source code om te zetten naar een binair formaat gebruiken we een compiler. Esenthel werkt met MSVC++ (Microsoft Visual C++)